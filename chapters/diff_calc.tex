%
% (c) 2025 Autor, ETH Zürich
%
% !TEX root = main.tex
% !TEX encoding = UTF-8
%

\section{Differential Calculus}
In this chapter we deal with differential calculus in one variable. This is of fundamental importance for understanding functions on $\mathbb{R}$.

\subsection{The Derivative}

\subsubsection{Definition and Geometrical Interpretation}
In this section $D \subseteq \mathbb{R}$ denotes a non-empty set with no isolated points, i.e., every $x \in D$ is an accumulation point of $D \setminus \{x\}$. A typical example is a non-empty interval containing more than one point.

\begin{definition}{Derivative}{derivative}
	Let $f:D \to \mathbb{R}$ be a function and $x_0 \in D$. We say that $f$ is \textbf{differentiable} at $x_0$ if the limit
	\begin{equation}
		\label{eq:diff_quot}
		f'(x_0) = \lim_{\underset{x \neq x_0}{x \to x_0}} \frac{f(x) - f(x_0)}{x - x_0} = \lim_{\underset{h \neq 0}{h \to 0}} \frac{f(x_0 + h) - f(x_0)}{h}
	\end{equation}
	exists. In this case we call $f'(x_0)$ the \textbf{derivative} of $f$ at $x_0$. If $f$ is differentiable at every point of $D$, we say that $f$ is \textbf{differentiable} on $D$, and we call the resulting function $f':D \to \mathbb{R}$ the \textbf{derivative} of $f$.
\end{definition}

To simplify notation, we will often write
\[
	f'(x_0) = \lim_{x \to x_0} \frac{f(x) - f(x_0)}{x - x_0} = \lim_{h \to 0} \frac{f(x_0 + h) - f(x_0)}{h}	
\]
without explicitly mentioning that $x \neq x_0$ and $h \neq 0$.
Note that the condition
\[
	f'(x_0) = \lim_{x \to x_0} \frac{f(x) - f(x_0)}{x - x_0}
\]
can be rewritten as
\[
	\lim_{x \to x_0} \frac{f(x) - f(x_0) - f'(x_0)(x - x_0)}{x - x_0} = 0,
\]
or equivalently, using the little-$o$ notation from Definition \ref{def*lil_o},
\begin{equation}
	\label{eq:derivative_lil_o}
	f(x) - f(x_0) - f'(x_0)(x - x_0) = o(x - x_0).
\end{equation}

\begin{remark}
	\label{rmk:diff_imp_cont}
	If $f:D \to \mathbb{R}$ is differentiable at $x_0$, then $f$ is also continuous at $x_0$. Indeed, using Equation \ref{eq:derivative_lil_o},
	\[
		\lim_{x \to x_0} f(x) = \lim_{x \to x_0} (f(x_0) + f'(x_0)(x - x_0) + o(x - x_0)) = f(x_0),
	\]
	hence $f$ is continuous at $x_0$.
\end{remark}

An alternative notation for the derivative of $f$ is $\frac{df}{dx}$. If $x_0 \in D$ is a right accumulation point of $D$, then $f$ is \textbf{differentiable from the right} at $x_0$ if the \textbf{right derivative}
\[
	f'_{+}(x_0) = \lim_{x \to x_0^{+}} \frac{f(x) - f(x_0)}{x - x_0} = \lim_{h \to 0^{+}} \frac{f(x_0 + h) - f(x_0)}{h}
\]
exists. \textbf{Differentiablility from the left} and the \textbf{left derivative} $f'_{-}(x_0)$ are defined analogously using the limit $x \to x_0^{-}$.

If $f:D \to \mathbb{R}$ is differentiable at $x_0 \in D$, the function $x \mapsto f(x_0) + f'(x_0)(x - x_0)$ is called the \textbf{affine approximation} of $f$ at $x_0$.

\begin{definition}{Higher Derivatives}{high_derivatives}
	Let $f:D \to \mathbb{R}$ be a function. We define the \textbf{higher derivatives} of $f$, if they exist, by
	\[
		f^{(0)} = f, \qquad f^{(1)} = f', \qquad f^{(2)} = f'',\; \hdots \;, f^{(n+1)} = (f^{(n)})'
	\]
	for all $n \in \mathbb{N}$. If $f^{(n)}$ exists, we say that $f$ is $n$\textbf{-times differentiable}. If the $n$-th derivative is also continuous, we say that $f$ is $n$\textbf{-times continuously differentiable}. We denote the set of $n$-times continuously differentiable functions on $D$ by $C^n(D)$.
\end{definition}

Equivalently, $C^0(D)$ is the set of real-valued continuous function on $D$, and $C^1(D)$ is the set of all differentiable functions  whose derivative is continuous (these are called \textbf{continuously differentiable} or of \textbf{class} $C^1$). Recursively, for $n \geq 1$, we define
\[
	C^n(D) = \{f:D \to \mathbb{R} \;|\; f \text{ is differentiable and } f' \in C^{n - 1}(D)\},
\]
and we say that $f \in C^{n}(D)$ is of \textbf{class} $C^n$.

\begin{definition}{Smooth Functions}{smooth_func}
	We define
	\[
		C^{\infty}(D) = \bigcap_{n = 0}^{\infty} C^n(D) = \{f:D \to \mathbb{R} \;|\; f \text{ is differentiable infinitely many times}\},
	\]
	and call functions $f \in C^{\infty}(D)$ \textbf{smooth} or of \textbf{class} $C^{\infty}$.
\end{definition}

The exponential function $\exp: \mathbb{R} \to \mathbb{R}$ is smooth.

\subsubsection{Differentiation Rules}
As with continuous functions, we rarely reprove differentiability from first principles for each new example. Instead, we use general rules that reduce differentiability of compound expressions to that of simpler ones.

\begin{proposition}{Derivative of Sum and Product}{derivative_sum_prod}
	Let $D \subseteq \mathbb{R}$ and $x_0 \in D$ be an accumulation point of $D \setminus \{x_0\}$. Let $f, g:D \to \mathbb{R}$ be differentiable at $x_0$. Then $f + g$ and $f \cdot g$ are differentiable at $x_0$, and
	\begin{subequations}
		\begin{align}
			\label{eq:sum_rule}
			(f + g)'(x_0) &= f'(x_0) + g'(x_0),\\
			\label{eq:prod_rule}
			(f \cdot g)'(x_0) &= f'(x_0)g(x_0) + f(x_0)g'(x_0).
		\end{align}
	\end{subequations}
	In particular, for any $\alpha \in \mathbb{R}$, the scalar multiple $\alpha f$ is differentiable at $x_0$ and $(\alpha f)'(x_0) = \alpha f'(x_0)$.
\end{proposition}

\begin{proof}
	Using the properties of limit discussed in Section \ref{sec:lim_vicinity_pt}, we have
	\begin{align*}
		\lim_{x \to x_0} \frac{(f + g)(x) - (f + g)(x_0)}{x - x_0} &= \lim_{x \to x_0} \left(\frac{f(x) - f(x_0)}{x - x_0} + \frac{g(x) g(x_0)}{x -x_0}\right)\\
		&=\lim_{x \to x_0} \frac{f(x) - f(x_0)}{x - x_0} + \lim_{x \to x_0} \frac{g(x) - g(x_0)}{x - x_0}\\
		&= f'(x_0) + g'(x_0),
	\end{align*}
	and
	\begin{align*}
		\lim_{x \to x_0} \frac{(fg)(x) - (fg)(x_0)}{x - x_0} &= \lim_{x \to x_0} \frac{(f(x) - f(x_0))g(x) + f(x_0)(g(x) - g(x_0)) }{x - x_0}\\
		&= \lim_{x \to x_0} \left(\frac{f(x) - f(x_0)}{x - x_0} g(x)\right) + f(x_0)\, \lim_{x \to x_0} \frac{g(x) - g(x_0)}{x - x_0}\\
		&= \lim_{x \to x_0} \left(\frac{f(x) - f(x_0)}{x - x_0}\right) \cdot \left( \lim_{x \to x_0} g(x) \right) + f(x_0)\, \lim_{x \to x_0} \frac{g(x) - g(x_0)}{x - x_0}\\
		&=
		f'(x_0)g(x_0) + f(x_0)g'(x_0),
	\end{align*}
	where we used that $g$ is continuous at $x_0$ (see Remark \ref{rmk:diff_imp_cont}) to conclude $\lim_{x \to x_0} g(x) = g(x_0)$. \qedhere
\end{proof}

\begin{corollary}{Higher Order Derivatives of the Sum and Product}{higher_derivative_prod}
	Let $f,g:D \to \mathbb{R}$ be $n$-times differentiable. Then $f + g$ and $f \cdot g$ are also $n$-times differentiable, and 
	\begin{align*}
		(f + g)^{(n)} &= f^{(n)} + g^{(n)},\\
		(fg)^{(n)} &= \sum_{k = 0}^{n} \binom{n}{k} f^{(k)} g^{(n - k)}.
	\end{align*}
	In particular, for every $\alpha \in \mathbb{R}$, $(\alpha f)^{(n)} = \alpha f^{(n)}$.
\end{corollary}

\begin{proof}
	For $n = 1$ this is Proposition \ref{prop*derivative_sum_prod}. The general case for $n \geq 1$ follows by induction. \qedhere
\end{proof}

\begin{corollary}{Derivatives of Polynomials}{derivative_poly}
	Polynomial functions are differentiable on all of $\mathbb{R}$. Moreover $(1)' = 0$ and $(x^n)' = nx^{n-1}$ for all $n \geq 1$.
\end{corollary}

\begin{proof}
	We argue by induction. For $n = 0$, we have
	\[
		(1)' = \lim_{x \to x_0} = \frac{1 - 1}{x - x_0} = 0.
	\]
	For $n = 1$, we have
	\[
		(x)' = \lim_{x \to x_0} \frac{x - x_0}{x - x_0} = 1.
	\]
	
	For $n > 1$, assume $(x^n)' = nx^{n-1}$. Then by Equation \eqref{eq:prod_rule}, since $x^{n+1} = x\cdot x^n$, we have
	\[
		(x^{n+1})' = (x \cdot x^n)' = 1 \cdot x^n + x \cdot n x^{n-1} = (n + 1)x^n.
	\]
	This proves the inductive step and establishes the result. Finally, the linearity of the derivative (see Equation \eqref{eq:sum_rule}) yields the differentiability of any polynomial. \qedhere
\end{proof}

\subsubsection*{Example}
With $\alpha = \pm 1$ and $\alpha = \pm i$, we have
\[
	(e^x)' = e^x, \qquad (e^{-x})' = -e^{-x}, \qquad (e^{ix})' = ie^{ix}, \qquad (e^{-ix})' = -ie^{-ix}.
\]
By Theorem \ref{theo*cmplx_exp_sin_cos}, we have
\[
	\sin'(x) = \frac{(e^{ix})' - (e^{-ix})'}{2i} = \frac{e^{ix} + e^{ix}}{2} = \cos(x),
\]
and analogously, $\cos'(x) = -\sin(x)$. Similarly $\sinh'(x) = \cosh(x)$ and $\cosh'(x) = \sinh(x)$.

\begin{theorem}{Chain Rule}{chain_rule}
	Let $D, E \subseteq \mathbb{R}$ and let $x_0 \in D$ be an accumulation point of $D \setminus \{x_0\}$. Let $f: D \to E$ be differentiable at $x_0$ such that $y_0 = f(x_0)$ is an accumulation point of $E \setminus \{y_0\}$, and let $g:E \to \mathbb{R}$ be differentiable at $y_0$. Then $g \circ f:D \to \mathbb{R}$ is differentiable at $x_0$ and 
	\[
		(g \circ f)'(x_0) = g'(f(x_0))f'(x_0).
	\]
\end{theorem}

\begin{proof}
	Observer that we can write
	\begin{align*}
		g(y) &= g(y_0) [g(y) - g(y_0)]\\
		&= g(y_0) + g'(y_0)(y - y_0) + [g(y) - g(y_0) - g'(y_0)(y - y_0)]\\
		&= g(y_0) + g'(y_0)(y - y_0) + \omega(y)(y - y_0),
	\end{align*}
	where $\omega: E \to \mathbb{R}$ is defined as
	\[
		\omega(y) = \begin{cases}
			\dfrac{g(y) - g(y_0)}{y - y_0} - g'(y_0) \qquad &\text{for }y \in E \setminus \{y_0\},\\
			0 \qquad &\text{for } y = y_0.
		\end{cases}
	\]
	Since $g$ is continuous at $y_0$, it follows that $\omega(y) \to 0$ as $y \to y_0$; hence, the function $\omega$ is continuous at $y_0$. Substituting $y = f(x)$ and using $y_0 = f(x_0)$, we get
	\[
		g(f(x)) = g(f(x_0)) + g'(f(x_0))[f(x) - f(x_0)] + \omega(f(x))[f(x) - f(x_0)],
	\]
	therefore
	\begin{align*}
		 \lim_{x \to x_0} \frac{g(g(x)) - g(f(x_0))}{x - x_0} &= \lim_{x \to x_0} \bigg( g'(f(x_0))\frac{f(x) - f(x_0)}{x - x_0} + \omega(f(x)) \frac{f(x) - f(x_0)}{x - x_0} \bigg)\\
		 &= g'(f(x_0)) f'(x_0) + \underbrace{\omega(f(x_0))}_{= 0} f'(x_0) = g'(f(x_0)) f'(x_0),
	\end{align*}
	where we used the continuity of $\omega$ at $y_0 = f(x_0)$ to deduce that $\omega(f(x)) \to \omega(f(x_0))$ as $x \to x_0$. \qedhere
\end{proof}

\begin{corollary}{Quotient Rule}{quot_rule}
	Let $D \subseteq \mathbb{R}$. let $x_0$ be an accumulation point of $D \setminus \{x_0\}$, and let $f,g:D \to \mathbb{R}$ be differentiable at $x_0$. If $g(x_0) \neq 0$, then $\frac{f}{g}$ is differentiable at $x_0$, and
	\[
		\left(\frac{f}{g}\right)'(x_0) = \frac{f'(x_0)g(x_0) - f(x_0)g'(x_0)}{g(x_0)^2}.
	\]
\end{corollary}

\begin{proof}
	Consider the function $\psi: \mathbb{R} \setminus \{0\} \to \mathbb{R}$ given by $\psi(y) = \frac{1}{y}$. This function is differentiable, with $\phi'(y) = -\frac{1}{y^2}$. Then, by the chain rule (Theorem \ref{theo*chain_rule}), $\frac{1}{g} = \psi \circ g$ is differentiable at $x_0$, with
	\[
		\bigg( \frac{1}{g}\bigg)'(x_0) = \psi'(g(x_0)) g'(x_0) = -\frac{g'(x_0)}{g(x_0)^2}.
	\]
	Applying now the product rule (Proposition \ref{prop*derivative_sum_prod}), $\frac{f}{g} = f \cdot \frac{1}{g}$ is differentiable at $x_0$, and
	\[
		\left(\frac{f}{g}\right)'(x_0) = \left(f \cdot \frac{1}{g}\right)'(x_0) = f'(x_0)\frac{1}{g(x_0)} - f(x_0) \frac{g'(x_0)}{g(x_0)^2} = \frac{f'(x_0)g(x_0) - f(x_0)g'(x_0)}{g^(x_0^2)}. \qedhere 
	\]
\end{proof}

\begin{theorem}{Derivative of the Inverse}{derivative_inv}
	Let $D, E \subseteq\mathbb{R}$, and let $f:D \to E$ be a continuous bijection whose inverse $f^{-1}:E \to D$ is also continuous. Let $\bar{x} \in D$ be an accumulation point of $D \setminus \{\bar{x}\}$, and assume $f$ is differentiable at $\bar{x}$ with $f'(\bar{x}) \neq 0$. Then $f^{-1}$ is differentiable at $\bar{y} = f(\bar{x})$ and
	\[
		(f^{-1})'(\bar{y}) = \frac{1}{f'(\bar{x})} = \frac{1}{f'(f^{-1}(\bar{y}))}.
	\] 
\end{theorem}

\begin{proof}
	To compute $(f^{-1})'(\bar{y})$, take a sequence $(y_n)_{n=0}^{\infty} \subseteq E \setminus \{\bar{y}\}$ with $y_n \to \bar{y}$ and set $x_n = f^{-1}(y_n)$. Then
	\[
		\frac{f^{-1}(y_n) - f^{-1}(\bar{y})}{y_n - \bar{y}} = \frac{x_n - \bar{x}}{f(x_n) - f(\bar{x})} = \left(\frac{f(x_n) - f(\bar{x})}{x_n - \bar{x}}\right)^{-1}.
	\]
	By the continuity of $f^{-1}$ we have $f^{-1}(y_n) = x_n \to \bar{x} = f^{-1}(\bar{y})$ as $y_n \to \bar{y}$. Thus, since $f$ is differentiable at $\bar{x}$ with $f'(\bar{x}) \neq 0$, Proposition \ref{prop*lim_op}(4) implies
	\[
		\left(\frac{f(x_n) - f(\bar{x})}{x_n - \bar{x}}\right)^{-1} \longrightarrow \frac{1}{f'(\bar{x})},
	\]
	proving that
	\[
		\lim_{n \to \infty} \frac{f^{-1}(y_n) - f^{-1}(\bar{y})}{y_n - \bar{y}} = \frac{1}{f'(\bar{x})}.
	\]
	Since the sequence $(y_n)_{n=0}^{\infty}$ was arbitrary, Lemma \ref{lem*lim_and_seq} gives
	\[
		\lim_{y \to \bar{y}} \frac{f^{-1}(y) - f^{-1}(\bar{y})}{y - \bar{y}} = \frac{1}{f'(\bar{x})},
	\]
	as desired. \qedhere
\end{proof}

\subsection{Main Theorems of Differential Calculus}

\subsubsection{Local Extrema}

\begin{definition}{Local Extrema}{local_extrema}
	Let $D \subseteq \mathbb{R}$ and $x_0 \in D$. We say that a function $f:D \to \mathbb{R}$ has a \textbf{local maximum} at $x_0$, if there exists $\delta > 0$ such that
	\[
		f(x) \leq f(x_0) \qquad \forall x \in D \cap (x_0 - \delta, x_0 + \delta).
	\]
	If the inequality is strict (i.e. $f(x) < f(x_0)$ for all $x \in (x_0 - \delta, x_0 + \delta) \setminus \{x_0\}$), then $f$ has a \textbf{strict local maximum} at $x_0$. A (\textbf{strict}) \textbf{local minimum} is defined analogously. We call $x_0$ a \textbf{local extremum} if $f$ has either a local maximum or a local minimum at $x_0$.
\end{definition}

\begin{proposition}{Local Extrema vs. First Derivative}{local_extrema_deriv}
	Let $D \subseteq \mathbb{R}$ and $f:D \to \mathbb{R}$. Suppose $x_0 \in D$ is a local extremum of $f$, that $f$ is differentiable at $x_0$ and that $x_0$ is both a right-hand and a left-hand accumulation point of $D$. Then
	\[
		f'(x_0) = 0.
	\]
\end{proposition}

\begin{proof}
	W.l.o.g, assume that $f$ has a local maximum at $x_0$ (otherwise replace $f$ by $-f$). We first note that for $x$ close to $x_0$ and to the right of it, we have $f(x) - f(x_0) \leq 0$ and $x - x_0 > 0$. Hence,
	\[
		f'_{+}(x_0) = \lim_{x \to 0^+} \frac{f(x) - f(x_0)}{x - x_0} \leq 0.
	\]
	Similarly for $x$ close to $x_0$ and to the left of it, we have $f(x) - f(x_0) \leq 0$ and $x - x_0 < 0$, so
	\[
		f'_{-}(x_0) = \lim_{x \to x_0^-} \frac{f(x) - f(x_0)}{x - x_0} \geq 0.
	\]
	Since $f$ is differentiable at $x_0$, the two one sided derivatives coincide, i.e.,
	\[
		 f'(x_0) = f'_{+}(x_0) = f'_{-}(x_0) = 0,
	\]
	and therefore $f'(x_0) = 0$. \qedhere
\end{proof}

\begin{corollary}{Local Extrema in an Inteval}{local_extrema_inteval}
	Let $I \subseteq \mathbb{R}$ be an interval and $f:I \to \mathbb{R}$. If $x_0 \in I$ is a local extrema, then at least one of the following statements holds:
	\begin{enumerate}
		\item $x_0$ is an endpoint of $I$,
		\item $f$ is not differentiable at $x_0$,
		\item $f$ is differentiable at $x_0$ and $f'(x_0) = 0$.
	\end{enumerate}
	In particular, all local extrema of a differentiable function on an open interval are zeros of the derivative.
\end{corollary}

\subsubsection{The Mean Value Theorem}

\begin{theorem}{Rolle's Theorem}{rolle_theo}
	Let $f:[a,b] \to \mathbb{R}$ be continuous on $[a,b]$ and differentiable on $(a,b)$. If $f(a) = f(b)$, then there exists $\xi \in (a,b)$ with $f'(\xi) = 0$.
\end{theorem}

\begin{proof}
	By Theorem \ref{theo*extreme_val_theo} $f$ attains both its maximum and minimum on $[a,b]$ at some points $x_0, x_1 \in [a,b]$. By Proposition \ref{prop*local_extrema_deriv} any interior extremum has zero derivative. Thus, we consider two cases:
	\begin{enumerate}
		\item[(i)] If either $x_0$ or $x_1$ lies in $(a,b)$ we are done.
		\item[(ii)] If both $x_0$ and $x_1$ are endpoints, since $f(a) = f(b)$, then $\min f = \max f = f(a) = f(b)$, therefore $f$ is constant. In particular, $f'(x) = 0$ for all $x \in (a,b)$ and the result follows also in this case. \qedhere
	\end{enumerate}
\end{proof}

\begin{corollary}{Non-Vanishing Derivative Implies Different Endpoint Values}{non_zero_deriv}
	Let $f:[a,b] \to \mathbb{R}$ be continuous on $[a,b]$ and differentiable on $(a,b)$. If $f'(x) \neq 0$ for all $x \in (a,b)$, then $f(a) \neq f(b)$.
\end{corollary}

\begin{proof}
	Assume, by contradiction, that $f(a) = f(b)$. Then by Rolle's Theorem, there would exists $\xi \in (a,b)$ such that $f'(\xi) = 0$, which would contradict the assumption that $f'$ never vanishes. \qedhere
\end{proof}

\begin{theorem}{Mean Value Theorem}{mean_val_theo}
	Let $f:[a,b] \to \mathbb{R}$ be continuous on $[a,b]$ and differentiable on $(a,b)$. Then there exists $\xi \in (a,b)$ such that
	\[
		f'(\xi) = \frac{f(b) - f(a)}{b - a}.
	\]
\end{theorem}

\begin{proof}
	Define $g:[a,b] \to \mathbb{R}$ by
	\[
		g(x) = f(x) - \frac{f(b) - f(a)}{b - a} (x - a).
	\]
	Then $g$ is continuous on $[a,b]$ and differentiable on $(a,b)$, and satisfies
	\[
		g(a) = f(a), \qquad g(b) = f(b) - (f(b) - f(a)) = f(a).
	\]
	By Rolle's Theorem \ref{theo*rolle_theo}, there exists $\xi \in (a,b)$ such that
	\[
		0 = g'(\xi) = f'(\xi) - \frac{f(b) - f(a)}{b - a},
	\]
	proving the result. \qedhere
\end{proof}

\begin{corollary}{Lipschitz Continuity vs. Bounded Derivative}
	Let $f:[a,b] \to \mathbb{R}$ be continuous on $[a,b]$ and differentiable on $(a,b)$. Then $f$ is Lipschitz continuous on $[a,b]$ if and only if $ f'$ is bounded on $(a,b)$.
\end{corollary}

\begin{proof}
	Suppose first that $f$ is Lipschitz continuous on $[a,b]$ with constant $L$. This implies that given $x, x_0 \in (a,b)$ with $x \neq x_0$,
	\[
		\left|\frac{f(x) - f(x_0)}{x - x_0}\right| \leq L.
	\]
	Taking the limit as $x \to x_0$ gives $|f'(x_0)| \leq L$, so $f'$ is bounded on $(a,b)$.
	
	Conversely, suppose that $f'$ is bounded on $(a,b)$, say $f'(z) \leq M$ for all $z \in (a,b)$. Then given $x, y \in [a,b]$ with $x < y$, the Mean Value Theorem \ref{theo*mean_val_theo} applied on the interval $[x, y]$ yields $\xi \in (x,y) \subseteq (a,b)$ such that
	\[
		f(y) - f(x) = f'(\xi)(y - x),
	\]
	therefore
	\[
		|f(y) - f(x)| = |f'(\xi)| |y - x| \leq M |y - x|.
	\]
	Since $x, y \in [a,b]$ are arbitrary, this shows that $f$ is Lipschitz continuous on $[a,b]$ with Lipschitz constant $M$. \qedhere
\end{proof}

\begin{theorem}{Cauchy Mean Value Theroem}{cauchy_mean_val}
	Let $f, g: [a,b] \to \mathbb{R}$ be continuous on $[a,b]$ and differentiable on $(a,b)$. Then there exists $\xi \in (a,b)$ such that
	\begin{equation}
		\label{eq:cauchy_mean_val}
		g'(\xi)(f(b) - f(a)) = f'(\xi)(g(b) - g(a)).
	\end{equation}
	If, in addition $g'(x) \neq 0$ for all $x \in (a,b)$, then $g(a) \neq g(b)$ and
	\[
		\frac{f'(\xi)}{g'(\xi)} = \frac{f(b) - f(a)}{g(b) - g(a)}.
	\]
\end{theorem}

\begin{proof}
	Define the function $F:[a,b] \to \mathbb{R}$ as
	\[
		F(x) = g(x)(f(b) - f(a)) - f(x)(g(b) - g(a)).
	\]
	Then
	\begin{align*}
		F(a) &= g(a)(f(b) - f(a)) - f(a)(g(b) - g(a)) = g(a)f(b) - f(a)g(b)\\
		F(b) &= g(b)(f(b) - f(a)) - f(b)(g(b) - g(a)) = g(a)f(b) - f(a)g(b).
	\end{align*}
	Thus, by Rolle's Theorem \ref{theo*rolle_theo}, there exists $\xi \in (a,b)$ such that
	\[
		F'(\xi) = 0 = g'(\xi)(f(b) - f(a)) - f'(\xi)(g(b) - g(a)),
	\]
	which is Equation \ref{eq:cauchy_mean_val}.
	
	If $g'(x) \neq 0$ for all $x \in (a,b)$, then Corollary \ref{cor*non_zero_deriv} yields $g(a) \neq g(b)$. Dividing Equation \ref{eq:cauchy_mean_val} accordingly gives the second formula. \qedhere
\end{proof}

\subsubsection{L'Hopital's Rule}
\begin{theorem}{L'Hopital's Rule}{hopital}
	Let $f, g: (a,b) \to \mathbb{R}$ be differentiable. Suppose:
	\begin{enumerate}
		\item $g(x) \neq 0$ and $g'(x) \neq 0$ for all $x \in (a,b)$,
		\item $\lim_{x \to a^{+}} f(x) = \lim_{x \to a^{+}} g(x) = 0$,
		\item the limit $L = \lim_{x \to a^{+}} \dfrac{f'(x)}{g'(x)}$ exists.
	\end{enumerate}
	Then $\lim_{x \to a^{+}} \dfrac{f(x)}{g(x)}$ exists and equals $L$.
\end{theorem}

\begin{proof}
	By (2), we can extend $f$ and $g$ continuously to $[a, b)$ by setting $f(a) = g(a) = 0$. Fix $\varepsilon > 0$. By (3), there exists $\delta > 0$ such that
	\[
		\frac{f'(\xi)}{g'(\xi)} \in (L - \varepsilon, L + \varepsilon) \qquad \forall x \in (a, a + \delta).
	\]
	Now, for any $x \in (a, a + \delta)$, we can apply Cauchy's Mean Value Theorem \ref{theo*cauchy_mean_val} to $f$ and $g$ on $[a, x]$ to find some $\xi_x \in (a,x)$ with
	\[
		\frac{f(x)}{g(x)} = \frac{f(x) - f(a)}{g(x) - g(a)} = \frac{f'(\xi_x)}{g'(\xi_x)}.
	\]
	Since $\xi_x \in (a, x) \subseteq (a, a + \delta)$, it follows that
	\[
		\frac{f(x)}{g(x)} = \frac{f'(\xi_x)}{g'(\xi_x)} \in (L - \varepsilon, L + \varepsilon) \qquad \forall x \in (a, a + \delta).
	\]
	Because $\varepsilon > 0$ is arbitrary, this proves that $\lim_{x \to a^+} \dfrac{f(x)}{g(x)} = L$.
\end{proof}

\begin{theorem}{L'Hopital's Rule for Improper Limits}{hopital_improper_lim}
	Let $f,g:(a,b) \to \mathbb{R}$ be differentiable. Suppose:
	\begin{enumerate}
		\item $g(x) \neq 0$ and $g'(x) \neq 0$ for all $x \in (a,b)$,
		\item $\lim_{x \to a^+} |f(x)| = \lim_{x \to a^{+}} |g(x)| = \infty$,
		\item the limit $L = \lim_{x \to a^{+}} \dfrac{f'(x)}{g'(x)}$ exists.
	\end{enumerate}
	Then $\lim_{x \to a^{+}} \dfrac{f(x)}{g(x)}$ exists and equals $L$.
\end{theorem}

\begin{theorem}{L'Hopital's Rule at Infinity}{hopital_infinity}
	Let $R > 0$ and $f,g:(R, \infty) \to \mathbb{R}$ be differentiable. Suppose:
	\begin{enumerate}
		\item $g(x) \neq 0$ and $g'(x) \neq 0$ for all $x \in (R, \infty)$,
		\item either $\lim_{x \to \infty} f(x) = \lim_{x \to \infty} g(x) = 0$ or $\lim_{x \to \infty} |f(x)| = \lim_{x \to \infty} |g(x)| = \infty$,
		\item the limit $L = \lim_{x \to \infty} \dfrac{f'(x)}{g'(x)}$ exists.
	\end{enumerate}
	Then $\lim_{x \to \infty} \dfrac{f(x)}{g(x)}$ exists and equals $L$.
\end{theorem}

\subsubsection{Monotonicity and Convexity via Differential Calculus}
Here $I$ always denotes a non-trivial interval (non-empty and not a single point).

\begin{proposition}{Monotonicity vs. First Derivative}{mono_deriv}
	Let $I \subseteq \mathbb{R}$ be an interval and let $f:I \to \mathbb{R}$ be differentiable. Then
	\[
		f' \geq 0 \quad \Leftrightarrow \quad f \text{ is increasing}.
	\]
\end{proposition}

\begin{proof}
	If $f$ is increasing $f(x + h) - f(x) \geq 0$ for $h > 0$, and $f(x + h) - f(x) \leq 0$ for $h < 0$. Hence, in both cases, $\frac{f(x+h) - f(x)}{h} \geq 0$, therefore
	\[
		f'(x) = \lim_{h \to 0} \frac{f(x + h) - f(x)}{h} \geq 0.
	\]
	Conversely, assume $f$ is not increasing. Then there exists $x_1 < x_2$ with $f(x_1) > f(x_2)$. By the Mean Value Theorem \ref{theo*mean_val_theo}, there exists $\xi \in (x_1, x_2)$ with
	\[
		f'(\xi) = \frac{f(x_2) - f(x_1)}{x_2 - x_1} < 0,
	\]
	so $f' \not \geq 0$ on $I$. \qedhere
\end{proof}

\begin{remark}
	If $f' > 0$, the same argument shows that $f$ is strictly increasing. However, the converse fails: the function $f(x) = x^3$is strictly increasing but $f'(0) = 0$.
\end{remark}

\begin{corollary}{Constant Functions vs. First Derivative}{const_deriv}
	Let $I \subseteq \mathbb{R}$ be an interval and $f:I \to \mathbb{R}$. Then $f$ is constant if and only if $f$ is differentiable and $f'(x) = 0$ for all $x \in I$.
\end{corollary}

\begin{proof}
	The derivative of a constant function is 0.
	
	Conversely, if $f' = 0$, then $f' \geq 0$ and $-f' \geq 0$, so by Proposition \ref{prop*mono_deriv} both $f$ and $-f$ are increasing, hence $f$ is constant. \qedhere
\end{proof}

\begin{definition}{Convex Functions}{convex_func}
	Let $I \subseteq \mathbb{R}$ and $f:I \to \mathbb{R}$. We call $f$ \textbf{convex}, if for all $a,b \in I$ with $a < b$ and all $t \in (0, 1)$, it holds that
	\begin{equation}
		\label{eq:convex_func}
		f((1-t)a + tb) \leq (1-t)f(a) + tf(b).
	\end{equation}
	We call $f$ \textbf{strictly convex} if the inequality in \eqref{eq:convex_func} is strict. A function $g:I \to \mathbb{R}$ is (\textbf{strictly}) \textbf{concave} if $-g$ is (strictly) convex.
	
	An equivalent definition of a convex function is the following: $f:I \to \mathbb{R}$ is convex if for all $a,b \in I$ with $a < b$ and all $x \in (a,b)$, we have
	\begin{equation}
		\label{eq:convex_func_2}
		\frac{f(x) - f(a)}{x - a} \leq \frac{f(b) - f(x)}{b - x},
	\end{equation}
	and strictly convex if the inequality is strict.
\end{definition}

\begin{proposition}{Convexity vs. Monotonicity of the First Derivative}{convex_deriv}
	Let $I \subseteq \mathbb{R}$ and let $f:I \to \mathbb{R}$ be differentiable. Then $f$ is convex if and only if $f'$ is increasing.
\end{proposition}

\begin{proof}
	Assume $f'$ is increasing. Then, for $a < b$ and $x \in (a, b)$, the Mean Value Theorem \ref{theo*mean_val_theo} applied on the intervals $[a, x]$ and $[x, b]$ yields $\xi \in (a, x)$ and $\zeta \in (x, b)$ such that
	\[
		f'(\xi) = \frac{f(x) - f(a)}{x - a}, \qquad f'(\zeta) = \frac{f(b) - f(x)}{b - x}.
	\]
	Since $f'$ is increasing, we have $f'(\xi) \leq f'(\zeta)$, so Equation \eqref{eq:convex_func_2} follows. Since $a < b \in I$ and $x \in (a, b)$ are arbitrary, $f$ is convex.
	
	Conversely, assume $f$ is convex. Given $a < b$, consider $h > 0$ small enough, such that $a + h < b - h$ and apply Equation \eqref{eq:convex_func_2} twice: first, applying it on the interval $(a, b - h)$ with $x = a + h$ we get
	\[
		\frac{f(a + h) - f(a)}{h} \leq \frac{f(b - h) - f(a + h)}{(b- h) - (a + h)};
	\]
	then, applying it on the interval $(a + h, b)$ with $x = b - h$ we get
	\[
		\frac{f(b - h) - f(a + h)}{(b - h) - (a + h)} \leq \frac{f(b) - f(b - h)}{h}.
	\]
	Combining these two inequalities, we deduce that, for small $h > 0$,
	\[
		\frac{f(a + h) - f(a)}{h} \leq \frac{f(b) - f(b - h)}{h}.
	\]
	Letting $h \to 0^+$ gives $f'(a) \leq f'(b)$. Since $a < b$ are arbitrary, $f'$ is increasing. \qedhere
\end{proof}

\begin{corollary}{Convexity vs. Second Derivative}{convex_second_deriv}
	Let $I \subseteq \mathbb{R}$ and let $f:I \to \mathbb{R}$ be twice differentiable. Then $f$ is convex if and only if $f'' \geq 0$.
\end{corollary}

\begin{proof}
	By Proposition \ref{prop*convex_deriv} $f$ is convex if and only if $f'$ is increasing. Applying Proposition \ref{prop*mono_deriv} to $f'$, we have that $f'$ is increasing if and only if $f'' \geq 0$. \qedhere
\end{proof}

\subsection{Example: Differentiation of Trigonometric Functions}
In this section we will list the derivatives of various trigonometric functions that can be looked up, when solving integrals, in order to find a certain primitive.
\subsubsection{Sine and Arc Sine}
The function $\sin:[-\frac{\pi}{2}, \frac{\pi}{2}] \to [-1, 1]$ satisfies
\[
	\sin'(x) = \cos(x), \qquad \cos'(x) = -\sin(x).
\]
Its inverse function $\arcsin:[-1, 1]\to [-\frac{\pi}{2}, \frac{\pi}{2}]$ has the derivative
\[
	\arcsin'(x) = \frac{1}{\cos(\arcsin(x))} = \frac{1}{\sqrt{1 - \sin^2(\arcsin(x))}} = \frac{1}{\sqrt{1 - x^2}},
\]
where we used the derivative of the inverse (Theorem \ref{theo*derivative_inv}) to calculate the above derivative.

\subsubsection{Cosine and Arc Cosine}
Similarly the function $\cos:[0, \pi] \to [-1, 1]$ has the inverse function $\arccos:[-1, 1] \to [0, \pi]$, which has the derivative
\[
	\arccos'(x) = \frac{1}{-\sin(\arccos(x))} = -\frac{1}{\sqrt{1 - \cos^2(\arccos(x))}} = -\frac{1}{\sqrt{1 - x^2}}.
\]

\subsubsection{Tangent and Arc Tangent}
The function $\tan:(-\frac{\pi}{2}, \frac{\pi}{2}) \to \mathbb{R}$ has the derivative
\[
	\tan'(x) = \left(\frac{\sin(x)}{\cos(x)}\right)' = \frac{\cos(x)\cos(x) - \sin(x)(-\sin(x))}{\cos^2(x)} = \frac{1}{\cos^2(x)}.
\]
Its inverse function $\arctan:\mathbb{R} \to (-\frac{\pi}{2}, \frac{\pi}{2})$ has the derivative
\[
	\arctan'(x) = \frac{1}{\dfrac{1}{\cos^2(\arctan(x))}} = \cos^2(\arctan(x)).
\]
Since 
\[
	\tan^2(x) = \frac{\sin^2(x)}{\cos^2(x)} = \frac{1 - \cos^2(x)}{\cos^2(x)} = \frac{1}{\cos^2(x)} - 1,
\]
it follows that
\[
	\arctan'(x) = \cos^2(\arctan(x)) = \frac{1}{1 + \tan^2(\arctan(x))} = \frac{1}{1 + x^2}.
\]

The cotangent and its inverse behave similarly, i.e. 
\[
	\cot'(x) = \left(\frac{\cos(x)}{\sin(x)}\right)' = \frac{-\sin(x)\sin(x) - \cos(x)\cos(x)}{\sin^2(x)} = \frac{-1}{\sin^2(x)},
\]
and
\[
	\operatorname{arccot}'(x) = -\frac{1}{1 + x^2}.
\]

\subsubsection{Hyperbolic Functions}
We now perform the analogous analysis for the hyperbolic trigonometric functions
\[
	\sinh(x) = \frac{e^x - e^{-x}}{2}, \qquad \cosh(x) = \frac{e^x + e^{-x}}{2}, \qquad \tanh(x) = \frac{\sinh(x)}{\cosh(x)}.
\]
The function $\sinh: \mathbb{R} \to \mathbb{R}$ is bijective and satisfies
\[
	\sinh'(x) = \cosh(x), \qquad \sinh''(x) = \sinh(x).
\] 
Its inverse $\operatorname{arsinh}:\mathbb{R} \to \mathbb{R}$ is called the inverse hyperbolic sine. Its derivative is given by
\[
	\operatorname{arsinh}'(x) = \frac{1}{\cosh(\operatorname{arsinh}(x))} = \frac{1}{\sqrt{1 + \sinh^2(\operatorname{arsinh}(x))}} = \frac{1}{\sqrt{1 + x^2}}.
\]

The hyperbolic cosine satisfies
\[
	\cosh'(x) = \sinh(x), \qquad \cosh''(x) = \cosh(x).
\]
Its inverse $\operatorname{arcosh}:[1,\infty] \to [0, \infty]$ has the derivative
\[
	\operatorname{arcosh}'(x) = \frac{1}{\sinh(\operatorname{arcosh}(x))} = \frac{1}{\sqrt{\cosh^2(\operatorname{arcosh}(x)) - 1}} = \frac{1}{\sqrt{x^2 - 1}}.
\]

The inverse hyperbolic tangent $\operatorname{artanh}:(-1, 1) \to \mathbb{R}$ has the derivative
\[
	\operatorname{artanh}'(x) = \frac{1}{1 - x^2}.
\]

Notice that the derivatives containing the inverse functions of either the tangent or hyperbolic tangent do not have a root in the denominator. One can memorize this by the sentence: ''You can't get tan under a roof''.



