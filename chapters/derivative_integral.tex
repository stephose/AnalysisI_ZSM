%
% (c) 2025 Autor, ETH Zürich
%
% !TEX root = main.tex
% !TEX encoding = UTF-8
%

\section{The Derivative and the Riemann Integral}

\subsection{The Fundamental Theorem of Calculus}
Throughout this section we fix a compact interval $I \subseteq\mathbb{R}$ that is non-emtpy and contains more than one point. For brevity, we write \textit{integrable} for \textit{Riemann integrable}.

\subsubsection{The Fundamental Theorem}
\begin{definition}{Primitive Function}{primitive_func}
	Let $I \subseteq \mathbb{R}$ be an interval and $f:I \to \mathbb{R}$ a function. Any differentiable function $F : I \to \mathbb{R}$ such that $F' = f$ is called a \textbf{primitive} (or \textbf{antiderivative}) of $f$.
\end{definition}

\begin{remark}
	A primitive may not always exist.
\end{remark}

The next result is known as the \textbf{Fundamental Theorem of (Integral and Differential) Calculus}, going back to Leibniz, Newton and Barrow.

\begin{theorem}{Fundamental Theorem of Calculus}{fund_thm_calc}
	Let $f:[a,b] \to \mathbb{R}$ be continuous. Then:
	\begin{enumerate}
		\item[(i)] For every $C \in \mathbb{R}$, the function $F:[a,b] \to \mathbb{R}$ defined by
		\begin{equation}
			\label{eq:primitive}
			F(x) = \int_{a}^{x} f(t) \, dt + C
		\end{equation}
		is a primitive of $f$.
		\item[(ii)] Every primitive $F:[a,b] \to \mathbb{R}$ of $f$ has the form \ref{eq:primitive} for some constant $C$.
	\end{enumerate}
\end{theorem}

\begin{proof}
	By Theorem \ref{theo*cont_func_int}, $f$ is integrable.
	
	Let $F$ be defined as in Equation \ref{eq:primitive}. To prove (i), we fix $x_0 \in [a,b]$ and we want to show that $F'(x_0) = f(x_0)$. To this aim, fix $\varepsilon > 0$. By continuity, there exists $\delta > 0$ such that
	\begin{equation}
		\label{eq:fund_cont}
		z \in [a,b], \quad |z - x_0| < \delta \quad \Rightarrow \quad |f(z) - f(x_0)| < \varepsilon.
	\end{equation}
	Now, given $x \in (x_0, x_0 + \delta) \cap [a,b]$, it follows from Remark \ref{rmk:int_seperate} that
	\begin{align*}
		\left|\frac{F(x) - F(x_0)}{x - x_0} - f(x_0)\right| &= \left|\frac{1}{x - x_0} \bigg(\int_{a}^{x} f(t) \, dt - \int_{a}^{x_0} f(t) \, dt\bigg) - f(x_0)\right|\\
		&= \left|\frac{1}{x- x_0} \int_{x_0}^{x} f(t) \, dt - f(x_0)\right|.
	\end{align*}
	Also,
	\[
		f(x_0) = f(x_0) \frac{1}{x - x_0} \int_{x_0}^{x} \, dt = \frac{1}{x - x_0} \int_{x_0}^{x} f(x_0) \, dt.
 	\]
 	Combining these two equations and using Theorem \ref{theo*triangle_ineq_int}, we get
 	\begin{align*}
 		\left|\frac{F(x) - F(x_0)}{x - x_0} - f(x_0)\right| &= \left|\frac{1}{x - x_0} \int_{x_0}^{x} f(t) \, dt - \frac{1}{x - x_0} \int_{x_0}^{x} f(x_0)\, dt\right|\\
 		&= \left|\frac{1}{x - x_0} \int_{x_0}^{x} (f(t) - f(x_0)) \, dt\right| \\
 		&\leq \frac{1}{x - x_0} \int_{x_0}^{x} |f(t) - f(x_0)| \, dt.
 	\end{align*}
 	Note now that, in the last integral, $t \in [x_0, x] \subseteq [x_0, x_0 + \delta) \cap [a,b]$. Hence, it follows form Equation \ref{eq:fund_cont} that $|f(t) - f(x_0)| < \varepsilon$, therefore
 	\[
 		\left|\frac{F(x) - F(x_0)}{x - x_0} - f(x_0)\right| < \frac{1}{x - x_0} \int_{x_0}^{x} \varepsilon \, dt = \varepsilon. 
 	\]
 	Similarly, if $x \in (x_0 - \delta, x_0) \cap [a,b]$, then
 	\[
 		\left|\frac{F(x) - F(x_0)}{x - x_0} - f(x_0)\right| = \left|\frac{1}{x_0 - x} \int_{x}^{x_0} (f(t) - f(x_0)) \, dt\right| \leq \frac{1}{x_0 - x} \int_{x}^{x_0} |f(t) - f(x_0)| \, dt < \varepsilon.
 	\]
 	In summary, we proved that
 	\[
 		x \in [a,b], \quad |x - x_0| < \delta \quad \Rightarrow \quad \left|\frac{F(x) - F(x_0)}{x - x_0} - f(x_0)\right| < \varepsilon,
 	\]
 	therefore
 	\[
 		F'(x_0) = \lim_{x \to x_0} \frac{F(x) - F(x_0)}{x - x_0} = f(x_0),
 	\]
 	as desired.
 	
 	We now prove (ii). Let $F$ be a primitive of $f$. Then, since $\left(\int_{a}^{x} f(t) \, dt\right)' = f(x)$ (by (i)), we have that
 	\[
 		\left(F(x) - \int_{a}^{x} f(t)\, dt\right)' = F'(x) - f(x) = f(x) - f(x) = 0 \qquad  \forall x \in (a,b).
 	\]
 	By Corollary \ref{cor*const_deriv}, this implies that $F(x) - \int_{a}^{x} f(t)\, dt$ is constant on $[a,b]$, concluding the proof of (ii). \qedhere
\end{proof}

\begin{corollary}{Integral vs. Derivative}{int_vs_deriv}
	If $F:[a,b] \to \mathbb{R}$ is continuously differentiable, then for all $x \in [a,b]$,
	\[
		F(x) = F(a) + \int_{a}^{x} F'(t)\, dt.
	\]
\end{corollary}

\begin{proof}
	Since $F$ is a primitive of $F'$, Theorem \ref{theo*fund_thm_calc} yields $F(x) = \int_{a}^{x} F'(t)\, dt + C$. Evaluating at $x = a$ gives $C = F(a)$. \qedhere
\end{proof}

\begin{corollary}{Riemann Integral and Primitives}{int_primitive}
	If $f:[a,b] \to \mathbb{R}$ is continuous and $F$ is a primitive of $f$, then
	\[
		\int_{a}^{b} f(t)\, dt = F(b) - F(a).
	\]
\end{corollary}

\begin{proof}
	Apply Corollary \ref{cor*int_vs_deriv} with $F' = f$ and $x = b$. \qedhere
\end{proof}

\subsubsection{Integration by Parts and Substitution}
Given a Function $h:[a,b] \to \mathbb{R}$, we use the notation $\big[h(x)\big]_a^b := h(b) - h(a)$.

\begin{theorem}{Integration by Parts}{int_parts}
	If $f,g:[a,b] \to \mathbb{R}$ are continuously differentiable, then
	\[
		\int_{a}^{b} f(x) g'(x) \, dx = \big[f(x)g(x)\big]_a^b - \int_{a}^{b} f'(x)g(x)\, dx.
	\]
\end{theorem}

\begin{proof}
	By Proposition \ref{prop*prod_rule}, $(fg)' = f'g + fg'$. Rearranging and integrating, thanks to Corollary $\ref{cor*int_primitive}$, we get
	\[
		\int_{a}^{b} fg'\, dx = \int_{a}^{b} (fg)'\, dx - \int_{a}^{b}f g'\, dx = \big[fg\big]_a^b - \int_{a}^{b} fg'\, dx. \qedhere
	\]
\end{proof}

As a convention, for any $h:[a,b] \to \mathbb{R}$,
\begin{equation}
	\label{eq:convention_int}
	\int_{b}^{a} h(x)\, dx = -\int_{a}^{b} h(x)\, dx.
\end{equation}

\begin{theorem}{Integration by Substitution, 1st Form}{int_sub1}
	Let $I,J \subseteq \mathbb{R}$ be intervals, $f:I \to J$ be continuously differentiable, and $g:J \to \mathbb{R}$ be continuous. For any $[a,b] \subseteq I$,
	\[
		\int_{a}^{b} g(f(x))f'(x) \, dx = \int_{f(a)}^{f(b)} g(y) \, dy.
	\]
\end{theorem}

\begin{proof}
	Fix $y_0 \in J$ and set $G(y) = \int_{y_0}^{y} g(t)\, dt$. Since $G' = g$, by the chain rule (Theorem \ref{theo*chain_rule}) we get $(G \circ f)' = G'(f)f' = g(f)f'$. Integrating this identity and using Corollary \ref{cor*int_primitive} yields
	\begin{align*}
		\int_{a}^{b} g(f(x))f'(x)\, dx &= \int_{a}^{b} (G\circ f)'(x)\, dx = G(f(b)) - G(f(a))\\
		&= \int_{y_0}^{f(b)} g(t)\, dt - \int_{y_0}^{f(a)} g(t)\, dt = \int_{f(a)}^{f(b)} g(t)\, dt.\qedhere
	\end{align*}
\end{proof}

Before stating the nect result, we note the following: If $h:[a,b]\to \mathbb{R}$ is continuously differentiable with $h' \neq 0$, then $h'$ has constant sign on $[a,b]$, so $h$ is strictly monotone and invertible ; $h^{-1}$ is continuous by Theorem \ref{theo*inv_func_theo} and differentiable on $(h(a), h(b))$ by Theorem \ref{theo*derivative_inv}. Furthermore, since$(h^{-1})' = \frac{1}{h'\circ h^{-1}}$, also $(h^{-1})'$ is continuous.

\begin{theorem}{Integration by Substitution, 2nd Form}{int_sub2}
	Let $I,J \subseteq \mathbb{R}$ be intervals, $f:I \to J$ be $C^1$, and $g:J \to \mathbb{R}$ be continuous. Let $[a,b] \subseteq I$ and assume $f'(x) \neq 0$ on $[a,b]$. If $f^{-1}:[f(a), f(b)] \to \mathbb{R}$ denotes the inverse of $f|_{[a,b]}$, then
	\[
		\int_{a}^{b} g(f(x))\, dx = \int_{f(a)}^{f(b)} g(y)(f^{-1})'(y) \, dy.
	\]
\end{theorem}

\begin{proof}
	In order to apply Theorem \ref{theo*int_sub1}, we first observe that
	\[
		\int_{a}^{b} g(f(x))\, dx = \int_{a}^{b} \frac{g(f(x))}{f'(x)}f'(x)\, dx = \int_{a}^{b}\frac{g(f(x))}{f' \circ f^{-1}(f(x))}f'(x)\, dx.	
	\]
	So we can apply Theorem \ref{theo*int_sub1} with $\frac{g}{f' \circ f^{-1}}$ in place of $g$ to get
	\[
		\int_{a}^{b} g(f(x)) \, dx = \int_{f(a)}^{f(b)} \frac{g(y)}{f'(f^{-1}(y))}\, dy.
	\]
	Since $\frac{1}{f'\circ f^{-1}} = (f^{-1})'$, the result follows. \qedhere
\end{proof}

\subsubsection{Improper Integrals}
A function $f:I \to \mathbb{R}$ is \textbf{locally integrable} if $f|_{[a,b]}$ is integrable for every compact interval $[a,b]\subseteq I$.

\begin{definition}{Improper Integrals}{improper_int}
	Let $I \subseteq \mathbb{R}$ be a non-empty interval and $f:I \to \mathbb{R}$ be locally integrable. Set $c = \inf I \in \mathbb{R}\cup \{-\infty\}$ and $d = \sup I \in \mathbb{R}\cup \{\infty\}$, and fix $x_0 \in I$. We define the \textbf{improper integral} of $f$ on $I$ by
	\[
		\int_{c}^{d} f(x)\, dx := \lim_{a \to c^+}\int_{a}^{x_0}f(x) \, dx + \lim_{b \to d^-} \int_{x_0}^{b} f(x)\, dx,
	\]
	whenever both limits exist and the sum is well-defined (we do not allow the indeterminate form $\infty - \infty$). Here the first limit is taken over $a \in I$ with $c < a < x_0$ (interpreting $a \to -\infty$ if $c = -\infty$) and the second over $b \in I$ with $x_0 < b < d$ (interpreting $b \to +\infty$ if $d = \infty$). If the value is finite we say the integral \textit{converges}; if it is $\pm \infty$ we say it \textit{diverges to} $\pm \infty$; otherwise, it \textit{does not converge}. When defined, the value is independent of the choice of $x_0$. 
\end{definition}

