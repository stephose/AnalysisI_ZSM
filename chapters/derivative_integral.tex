%
% (c) 2025 Autor, ETH Zürich
%
% !TEX root = main.tex
% !TEX encoding = UTF-8
%

\section{The Derivative and the Riemann Integral}

\subsection{The Fundamental Theorem of Calculus}
Throughout this section we fix a compact interval $I \subseteq\mathbb{R}$ that is non-emtpy and contains more than one point. For brevity, we write \textit{integrable} for \textit{Riemann integrable}.

\subsubsection{The Fundamental Theorem}
\begin{definition}{Primitive Function}{primitive_func}
	Let $I \subseteq \mathbb{R}$ be an interval and $f:I \to \mathbb{R}$ a function. Any differentiable function $F : I \to \mathbb{R}$ such that $F' = f$ is called a \textbf{primitive} (or \textbf{antiderivative}) of $f$.
\end{definition}

\begin{remark}
	A primitive may not always exist.
\end{remark}

The next result is known as the \textbf{Fundamental Theorem of (Integral and Differential) Calculus}, going back to Leibniz, Newton and Barrow.

\begin{theorem}{Fundamental Theorem of Calculus}{fund_thm_calc}
	Let $f:[a,b] \to \mathbb{R}$ be continuous. Then:
	\begin{enumerate}
		\item[(i)] For every $C \in \mathbb{R}$, the function $F:[a,b] \to \mathbb{R}$ defined by
		\begin{equation}
			\label{eq:primitive}
			F(x) = \int_{a}^{x} f(t) \, dt + C
		\end{equation}
		is a primitive of $f$.
		\item[(ii)] Every primitive $F:[a,b] \to \mathbb{R}$ of $f$ has the form \ref{eq:primitive} for some constant $C$.
	\end{enumerate}
\end{theorem}

\begin{proof}
	By Theorem \ref{theo*cont_func_int}, $f$ is integrable.
	
	Let $F$ be defined as in Equation \ref{eq:primitive}. To prove (i), we fix $x_0 \in [a,b]$ and we want to show that $F'(x_0) = f(x_0)$. To this aim, fix $\varepsilon > 0$. By continuity, there exists $\delta > 0$ such that
	\begin{equation}
		\label{eq:fund_cont}
		z \in [a,b], \quad |z - x_0| < \delta \quad \Rightarrow \quad |f(z) - f(x_0)| < \varepsilon.
	\end{equation}
	Now, given $x \in (x_0, x_0 + \delta) \cap [a,b]$, it follows from Remark \ref{rmk:int_seperate} that
	\begin{align*}
		\left|\frac{F(x) - F(x_0)}{x - x_0} - f(x_0)\right| &= \left|\frac{1}{x - x_0} \bigg(\int_{a}^{x} f(t) \, dt - \int_{a}^{x_0} f(t) \, dt\bigg) - f(x_0)\right|\\
		&= \left|\frac{1}{x- x_0} \int_{x_0}^{x} f(t) \, dt - f(x_0)\right|.
	\end{align*}
	Also,
	\[
		f(x_0) = f(x_0) \frac{1}{x - x_0} \int_{x_0}^{x} \, dt = \frac{1}{x - x_0} \int_{x_0}^{x} f(x_0) \, dt.
 	\]
 	Combining these two equations and using Theorem \ref{theo*triangle_ineq_int}, we get
 	\begin{align*}
 		\left|\frac{F(x) - F(x_0)}{x - x_0} - f(x_0)\right| &= \left|\frac{1}{x - x_0} \int_{x_0}^{x} f(t) \, dt - \frac{1}{x - x_0} \int_{x_0}^{x} f(x_0)\, dt\right|\\
 		&= \left|\frac{1}{x - x_0} \int_{x_0}^{x} (f(t) - f(x_0)) \, dt\right| \\
 		&\leq \frac{1}{x - x_0} \int_{x_0}^{x} |f(t) - f(x_0)| \, dt.
 	\end{align*}
 	Note now that, in the last integral, $t \in [x_0, x] \subseteq [x_0, x_0 + \delta) \cap [a,b]$. Hence, it follows form Equation \ref{eq:fund_cont} that $|f(t) - f(x_0)| < \varepsilon$, therefore
 	\[
 		\left|\frac{F(x) - F(x_0)}{x - x_0} - f(x_0)\right| < \frac{1}{x - x_0} \int_{x_0}^{x} \varepsilon \, dt = \varepsilon. 
 	\]
 	Similarly, if $x \in (x_0 - \delta, x_0) \cap [a,b]$, then
 	\[
 		\left|\frac{F(x) - F(x_0)}{x - x_0} - f(x_0)\right| = \left|\frac{1}{x_0 - x} \int_{x}^{x_0} (f(t) - f(x_0)) \, dt\right| \leq \frac{1}{x_0 - x} \int_{x}^{x_0} |f(t) - f(x_0)| \, dt < \varepsilon.
 	\]
 	In summary, we proved that
 	\[
 		x \in [a,b], \quad |x - x_0| < \delta \quad \Rightarrow \quad \left|\frac{F(x) - F(x_0)}{x - x_0} - f(x_0)\right| < \varepsilon,
 	\]
 	therefore
 	\[
 		F'(x_0) = \lim_{x \to x_0} \frac{F(x) - F(x_0)}{x - x_0} = f(x_0),
 	\]
 	as desired.
 	
 	We now prove (ii). Let $F$ be a primitive of $f$. Then, since $\left(\int_{a}^{x} f(t) \, dt\right)' = f(x)$ (by (i)), we have that
 	\[
 		\left(F(x) - \int_{a}^{x} f(t)\, dt\right)' = F'(x) - f(x) = f(x) - f(x) = 0 \qquad  \forall x \in (a,b).
 	\]
 	By Corollary \ref{cor*const_deriv}, this implies that $F(x) - \int_{a}^{x} f(t)\, dt$ is constant on $[a,b]$, concluding the proof of (ii). \qedhere
\end{proof}