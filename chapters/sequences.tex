%
% (c) 2025 Autor, ETH Zürich
%
% !TEX root = main.tex
% !TEX encoding = UTF-8
%

\section{Sequences of Real Numbers}

\subsection{Convergence of Sequences}

\begin{definition}{Sequences}{sequences}
	A \textbf{sequence} is a function $a:\mathbb{N} \to \mathbb{R}$. The image $a(n)$ of $n \in \mathbb{N}$ is also written as $a_n$ and is called the $n$-th element of a. Instead of $a:\mathbb{N} \to \mathbb{R}$ one often writes $(a_n)_{n \in \mathbb{N}}, (a_n)_{n =0}^{\infty}, (a_n)_{n \geq 0}$.
\end{definition}

\begin{definition}{(Eventually) Constant Sequences}{const sequences}
	A sequence $(x_n)_{n=0}^{\infty}$ is \textbf{constant} if $x_n = x_m \forall n,m \in \mathbb{N}$. It is \textbf{eventually constant} if there exists $N \in \mathbb{N}$ such that $x_n = x_m \forall n,m \geq N$.
\end{definition}

\begin{definition}{Convergence of Sequences}{conv sequences}
	Let $(x_n)_{n=0}^{\infty}$ be a sequence in $\mathbb{R}$. We say that $(x_n)_{n=0}^{\infty}$ \textbf{converges} (or is \textbf{convergent}) if $\exists A \in \mathbb{R}$ such that 
	\[
		\forall \varepsilon > 0\; \exists N \in \mathbb{N} : |x_n - A| < \varepsilon\quad \forall n \geq N.
	\]
	In this case we write
	\begin{equation}
		\lim_{n \to \infty} x_n = A
	\end{equation}
	and call $A$ the \textbf{limit} of $(x_n)_{n=0}^{\infty}$.
\end{definition}

\begin{lemma}{Uniqueness of the Limit}{unique limit}
	A convergent sequence $(x_n)_{n=0}^{\infty}$ has exactly one limit.
\end{lemma}
\begin{proof}
	Let $A,B \in \mathbb{R}$ be limits of $(x_n)_{n=0}^{\infty}$. Fix $\varepsilon > 0$. Then there exists $N_A, N_B \in \mathbb{N}$ such that $|x_n -A| < \varepsilon$ for all $n \geq N_A$ and $|x_n - B| < \varepsilon$ for all $n \geq N_B$. We define $N := \operatorname{max}\{N_A, N_B\}$. Then it holds that
	\[
		|A - B| \leq |A - x_N| + |x_N - B| < \varepsilon + \varepsilon = 2\varepsilon.
	\]
	Since $\varepsilon > 0$ is arbitrary, it follows that $A = B$. \qedhere
\end{proof}

\subsection{Convergent Subsequences and Acculmulation Points}
\begin{definition}{Subsequences}{subseq}
	Let $(x_n)_{n=0}^{\infty}$ be a sequence. A \textbf{subsequence} is of the form $(x_{n_k})_{k=0}^{\infty}$, where $(n_k)_{k=0}^{\infty}$ is a strictly increasing sequence of non-negative integers, i.e., $n_{k+1} > n_k \; \forall k \in \mathbb{N}$.
\end{definition}

\begin{remark}
	\label{rmk:subseq}
	Since $n_{k+1} > n_k$ for all $k \in \mathbb{N}$ it follows by induction that $n_k \geq k$ for all $k \in \mathbb{N}$.
\end{remark}
\begin{proof}
	For $k=0$ we have that $n_0 \geq 0$, because $(n_k)_{k=0}^{\infty}$ is a sequence of non-negative integers. So the condition is fulfilled.
	For the inductive step we want to show that the condition holds for $k+1$ under the assumption that the condition is true for $k$. Because $(n_k)_{k=0}^{\infty}$ is also a strictly increasing sequence, we have that $n_{k+1} > n_k \geq k$. Additionally since $n_k \in \mathbb{N}$, we have that $n_{k+1} \geq n_k + 1$. 
	So it follows that $n_{k+1} \geq n_k + 1 \geq k + 1$, which proofs the condition for $k+1$.\qedhere
\end{proof}

\begin{lemma}{Subsequences of Convergent Sequences are Convergent}{subseq conv}
	Let $(x_n)_{n=0}^{\infty}$ be a sequence converging to $A \in \mathbb{R}$. Then every subsequence $(x_{n_k})_{k=0}^{\infty}$ also converges to $A$.
\end{lemma}
\begin{proof}
	Let $(x_n)_{n=0}^{\infty}$ be a sequence converging to $A \in \mathbb{R}$.
	Fix $\varepsilon > 0$.
	Since $(x_n)_{n=0}^{\infty}$ converges to $A$, there exists $N \in \mathbb{N}$ such that $|x_n - A| < \varepsilon \; \forall n \geq N$. As by Remark \ref{rmk:subseq} we know that $n_k \geq k$ for all $k \in \mathbb{N}$.
	Therefore for all $k \geq N$ it holds that $|x_{n_k} - A| < \varepsilon$. \qedhere
\end{proof}

\begin{definition}{Accumulation Points of Sequences}{acc pt seq}
	Let $(x_n)_{n=0}^{\infty}$ be a sequence in $\mathbb{R}$. A point $A \in \mathbb{R}$ is an \textbf{accumulation point} of $(x_n)_{n=0}^{\infty}$ if
	\[
		\forall \varepsilon > 0 \; \forall N \in \mathbb{N} \; \exists n \geq N : |x_n - A| < \varepsilon.
	\]
\end{definition}

\begin{proposition}{Subsequences and Accumulation Points}{subseq_acc_pt}
	Let $(x_n)_{n=0}^{\infty}$ be a sequence in $\mathbb{R}$. A point $A$ is and accumulation point of $(x_n)_{n=0}^{\infty}$ if and only if there exists a convergent subsequence of $(x_n)_{n=0}^{\infty}$ with limit $A$.
\end{proposition}

\begin{proof}
	First assume that $A \in \mathbb{R}$ is an accumulation point of $(x_n)_{n=0}^{\infty}$.
	We construct $(n_k)_{k\geq 0}$ recursively:
	\begin{itemize}
		\item first, apply the definition of accumulation point with $N = 1$ and $\varepsilon = 1 = 2^0$ to find $n_0 \geq 1$ with $|x_{n_0} - A| \leq 2^0$,
		\item second, apply the definition the definition of accumulation point with $N = n_0 + 1$ and $\varepsilon = 2^{-1}$ to find $n_1 \geq n_0 + 1$ with $|x_{n_1} - A| \leq 2^{-1}$,
		\item more in general given $n_{k-1}$, we apply the definition of accumulation point with $N = n_{k-1} + 1$ and $\varepsilon = 2^{-k}$ to find $n_k \geq n_{k-1} + 1$ with $|x_{n_k} - A| \leq 2^{-k}$.
	\end{itemize}
	Now given $\varepsilon > 0$ choose $N$ such that $2^{-N} < \varepsilon$. Then for all $k \geq N$ we have that
	\[
		|x_{n_k} - A| \leq 2^{-k} \leq 2^{-N} < \varepsilon,
	\]
	so $\lim_{k \to \infty} x_{n_k} = A$.
	
	Conversely, assume that there exists a subsequence $(x_{n_k})_{k=0}^{\infty}$ converging to $A$. Fix $\varepsilon > 0$ and $N \in \mathbb{N}$. 
	Since $\lim_{k \to \infty} x_{n_k} = A$, there exists $N_0$ such that $|x_{n_k} - A| < \varepsilon$ for all $k \geq N_0$.
	Hence if we choose $k = \max\{N_0, N\}$, because $n_k \geq n$ (recall Remark \ref{rmk:subseq}) we have that $n_k \geq N$ and $|x_{n_k} - A| < \varepsilon$.
	Thus $A$ is an accumulation point. \qedhere
\end{proof}

\begin{corollary}{Infinitely Many Terms Near an Accumulation Point}{inf terms acc pt}
	If $A \in \mathbb{R}$ is an accumulation point of $(x_n)_{n=0}^{\infty}$, then for every $\varepsilon > 0$ there are infinitely many n with $x_n \in (A - \varepsilon, A + \varepsilon)$.
\end{corollary}
\begin{proof}
	By Proposition \ref{prop*subseq_acc_pt}, there exists a subsequence $(x_{n_k})_{k=0}^{\infty}$ with $\lim_{k \to \infty} = A$. Hence for every $\varepsilon > 0$ there exists $K$ such that $x_{n_k} \in (A - \varepsilon, A + \varepsilon)$ for all $k \geq K$, providing infinitely many elements of the sequence inside the interval $(A - \varepsilon, A + \varepsilon)$. \qedhere
\end{proof}

\begin{corollary}{Accumulation Points of Convergent Sequences}
	A convergent sequence has exactly one accumulation point, namely its limit.
\end{corollary}

\subsection{Addition, Multiplication and Inequalities}
\begin{proposition}{Limits and Operations}{lim_op}
	Let $(x_n)_{n=0}^{\infty}$ and $(y_n)_{n=0}^\infty$ be sequences converging to $A, B \in \mathbb{R}$ respectively. Then:
	\begin{enumerate}
		\item The sequence $(x_n + y_n)_{n=0}^{\infty}$ converges to $A + B$.
		\item The sequence $(x_ny_n)_{n=0}^{\infty}$ converges to $AB$.
		\item Given $\alpha \in \mathbb{R}$, the sequence $(\alpha x_n)_{n=0}^{\infty}$ converges to $\alpha A$.
		\item Suppose $x_n \neq 0$ for all $n \in \mathbb{N}$ and $A \neq 0$. Then the sequence $(x_n^{-1})_{n=0}^{\infty}$ converges to $A^{-1}$. 
	\end{enumerate}
\end{proposition}

\begin{proposition}{Limits and Inequalities}{lim_ineq}
	Let $(x_n)_{n=0}^{\infty}$ and $(y_n)_{n=0}^\infty$ be sequences converging to $A, B \in \mathbb{R}$ respectively.
	\begin{enumerate}
		\item If $A < B$, then there exists $N \in \mathbb{N}$ such that $x_n < y_n$ for all $n \geq \mathbb{N}$.
		\item If there exists $N \in \mathbb{N}$ such that $x_n \leq y_n$ for all $n \geq N$, then $A \leq B$.
	\end{enumerate}
\end{proposition}

\begin{remark}
	In Propostion \ref{prop*lim_ineq} even if we assume that $x_n < y_n$ for all $n \in \mathbb{N}$, we cannot conclude that $A < B$. for example take
	\[
		x_n = \frac{1}{n}, \quad y_n = \frac{1}{n}.
	\]
	Then we have that $x_n < y_n$ for all $n \in \mathbb{N}$ but $A = B = 0$.
\end{remark}

\begin{lemma}{Sandwich Lemma}{sandwich_seq}
	Let $(x_n)_{n=0}^{\infty}$, $(y_n)_{n=0}^{\infty}$, $(z_n)_{n=0}^{\infty}$ be sequences such that for some $N \in \mathbb{N}$, we have that
	\[
		x_n \leq y_n \leq z_n \quad \forall n \geq N.
	\]
	Suppose that both $(x_n)_{n=0}^{\infty}$ and $(z_n)_{n=0}^{\infty}$ converge to the same limit. Then $(y_n)_{n=0}^{\infty}$ also converges, and we have that
	\[
		\lim_{n \to \infty} x_n = \lim_{n \to \infty} y_n = \lim_{n \to \infty} z_n.
	\]
\end{lemma}

\begin{proof}
	Let $(x_n)_{n=0}^{\infty}$, $(y_n)_{n=0}^{\infty}$, $(z_n)_{n=0}^{\infty}$ be sequences such that for some $N_0 \in \mathbb{N}$, we have that
	\[
		x_n \leq y_n \leq z_n \quad \forall n \geq N_0.
	\]
	Additionally suppose that $(x_n)_{n=0}^{\infty}$ and $(z_n)_{n=0}^{\infty}$ converge to $A \in \mathbb{R}$. Fix $\varepsilon > 0$. Since $(x_n)_{n=0}^{\infty}$, $(z_n)_{n=0}^{\infty}$ converge to $A$ there exists $N_x, N_z \in \mathbb{N}$ such that
	\begin{align*}
		A - \varepsilon &< x_n < A + \varepsilon \quad \forall n \geq N_x\\
		A - \varepsilon &< z_n < A + \varepsilon \quad \forall n_z \geq N_z.
	\end{align*}
	So we choose $N := \operatorname{max}\{N_0, N_x, N_z\}$. Then we have that
	\[
		A - \varepsilon < x_n \leq y_n \leq z_n < A + \varepsilon \quad \forall n \geq N,
	\] 
	which shows that $\lim_{n \to \infty} y_n = A$.\qedhere
\end{proof}

\begin{definition}{Bounded Sequences}{bound_seq}
	A sequence $(x_n)_{n=0}^{\infty}$ is called \textbf{bounded} if there exists a real number $M \geq 0$ such that
	\[
		|x_n| \leq M \quad \forall n \in \mathbb{N}.
	\]
\end{definition}

\begin{lemma}{Convergent Sequences are Bounded}{conv_seq_bound}
	Every convergent sequence is bounded.
\end{lemma}
\begin{proof}
	Let $(x_n)_{n=0}^{\infty}$ be a sequence converging to $A \in \mathbb{R}$. Let $\varepsilon = 1$. Then, by convergence of $(x_n)_{n=0}^{\infty}$, there exists $N$ such that $|x_n - A| < 1$ for all $n \geq N$. So we have that
	\[
		|x_n| = |x_n - A + A| \leq |x_n - A| + |A| < 1 + |A|. 
	\]
	We choose
	\[
		M = \operatorname{max}(|x_0|, |x_1|, \hdots , |x_{N-1}|, 1 + |A|).
	\]
	Then $(x_n)_{n=0}^{\infty} \leq M$ for all $n \in \mathbb{N}$ as desired. \qedhere 
\end{proof}


\begin{definition}{Monotone Sequences}{mono_seq}
	A sequence $(x_n)_{n=0}^{\infty}$ is called:
	\begin{itemize}
		\item \textbf{(monotonically) increasing} if $m > n\; \Rightarrow \; x_m \geq x_n$,
		\item \textbf{strictly (monotonically) increasing} if $m > n \; \Rightarrow \; x_m > x_n$,
		\item \textbf{(monotonically) decreasing} if $m > n \; \Rightarrow \; x_m \leq x_n$,
		\item \textbf{strictly (monotonically) decreasing} if $m > n \; \Rightarrow \; x_m < x_n$.
	\end{itemize}
	If a sequence is decreasing or increasing we call it monotone. If a sequence is strictly increasing or strictly decreasing then we call it strictly monotone.
\end{definition}

\begin{remark}
	An equivalent formulation of monotone sequences can be given using only successive terms:
	\begin{itemize}
		\item $(x_n)_{n=0}^{\infty}$ is increasing if $x_{n + 1} \geq x_n$ for all $n$,
		\item $(x_n)_{n=0}^{\infty}$ is strictly increasing if $x_{n+1} > x_n$ for all $n$,
		\item $(x_n)_{n=0}^{\infty}$ is decreasing if $x_{n+1} \leq x_n$ for all $n$,
		\item $(x_n)_{n=0}^{\infty}$ is strictly decreasing if $x_{n+1} < x_n$ for all $n$.
	\end{itemize}
\end{remark}

\begin{theorem}{Convergence of Monotone Sequences}{conv_mono_seq}
	A monotone sequence $(x_n)_{n=0}^{\infty}$ converges if and only if it is bounded. More precisely, let $X = \{x_n\; |\; n \in \mathbb{N}\}$ denote the set of points in the sequence.
	\begin{itemize}
		\item If $(x_n)_{n=0}^{\infty}$ is increasing, then $\lim_{n \to \infty} x_n = \operatorname{sup}(X)$,
		\item if $(x_n)_{n=0}^{\infty}$ decreasing, then $\lim_{n \to \infty} x_n = \operatorname{inf}(X)$.
	\end{itemize}
\end{theorem}
\begin{proof}
	If $(x_n)_{n=0}^{\infty}$ converges Lemma \ref{lem*conv_seq_bound} says that its bounded.
	
	Conversely, let $(x_n)_{n=0}^{\infty}$ be a bounded monotone sequence. Wlog assume that $(x_n)_{n=0}^{\infty}$ is increasing (otherwise consider $(- x_n)_{n=0}^{\infty}$). Since $(x_n)_{n=0}^{\infty}$ is bounded from above, the set $X = \{x_n\; |\; n \in \mathbb{N}\}$ has a supremum, that we'll call $A = \operatorname{sup}(X)$.
	
	By definiton of $A$:
	\begin{enumerate}
		\item[(i)] $x_n \leq A \quad \forall n \in \mathbb{N}$,
		\item[(ii)] $\forall \varepsilon > 0$ there exists $N \in \mathbb{N}$ such that $x_N > A - \varepsilon$.
	\end{enumerate}
	Then, for all $n \geq N$ using (ii) and monotonicity, we have that $x_n \geq x_N > A - \varepsilon$. Then using (i), we conclude that
	\begin{align*}
		A - \varepsilon < x_n < A + \varepsilon \quad \forall n \geq N.
	\end{align*}
	\qedhere
\end{proof}

\subsection{Superior and Inferior Limits}

Let $(x_n)_{n=0}^{\infty}$ be a bounded sequence. To study its behavior for large $n$ its is useful to look at its tails
\[
	X_{\geq n} = \{x_k \;|\; k \geq n \} \subseteq \mathbb{R}.
\]
The concept of limits can be restated using the tails of a sequence, i.e., the sequence $(x_n)_{n=0}^{\infty}$ converges to $A \in \mathbb{R}$ if and only if, for every $\varepsilon > 0$ there exists $N \in \mathbb{N}$ such that $X_{N} \subseteq (A - \varepsilon, A + \varepsilon)$.

However, since not every sequence has a limit we now introduce a related notion (the \textbf{superior} and \textbf{inferior limits}), which always exist for bounded sequences.

For each $n \in \mathbb{N}$, define
\[
	s_n = \sup(X_{\geq n}) = \sup_{k\geq n}x_k, \qquad i_n = \inf(X_{\geq n}) = \inf_{k \geq n} x_k.
\]
Since $X_{\geq m} \subset X_{\geq n}$, whenever $m > n$, we have that
\[
	i_n \leq i_m \leq s_m \leq s_n \qquad \forall m > n.
\]
Thus, $(s_n)_{n=0}^{\infty}$ is a monotonically decreasing sequence, while $(i_n)_{n=0}^{\infty}$ is a monotonically increasing sequence.
Moreover, since $(x_n)_{n=0}^{\infty}$ is bounded both $(s_n)_{n=0}^{\infty}$ and $(i_n)_{n=0}^{\infty}$ are bounded as well.
Hence by Theorem \ref{theo*conv_mono_seq}, both sequences converge. Their limits will be called the \textit{superior} and the \textit{inferior limit} of $(x_n)_{n=0}^{\infty}$ respectively.

Note that, since $x_n \in X_{\geq n}$, we have that
\begin{equation}
	\label{eq:ineq_sup_inf}
	i_n \leq x_n \leq s_n \qquad \forall n \in \mathbb{N}.
\end{equation}

\begin{definition}{Superior and Inferior Limits}{sup_inf}
	Let $(x_n)_{n=0}^{\infty}$ be a bounded sequence in $\mathbb{R}$. The numbers
	\[
		\limsup_{n \to \infty} x_n = \lim_{n \to \infty} \big(\sup_{k\geq n}x_k\big), \qquad
		\liminf_{n \to \infty} x_n = \lim_{n \to \infty} \big(\inf_{k \geq n}x_k\big)
	\]
	are called the \textbf{superior} and \textbf{inferior limit} of $(x_n)_{n=0}^{\infty}$ respectively.
	From Equation \ref{eq:ineq_sup_inf} and Proposition \ref{prop*lim_ineq}, we have
	\[
		\liminf_{n \to \infty} x_n \leq \limsup_{n \to \infty} x_n.
	\]
\end{definition}

\begin{lemma}{Convergence and Superior/Inferior Limits}{conv_limsup_liminf}
	A bounded sequence $(x_n)_{n=0}^{\infty}$ in $\mathbb{R}$ converges if and only if
	\[
		\limsup_{n \to \infty} x_n = \liminf_{n \to \infty} x_n.
	\]
\end{lemma}

\begin{proof}
	For every $n \in \mathbb{N}$, define
	\[
		i_n = \inf_{k \geq n} x_k, \qquad s_n = \sup_{k\geq n} x_k,
	\]
	and set
	\[
		I = \lim_{n \to \infty} i_n = \liminf_{n \to \infty} x_n, \qquad S = \lim_{n \to \infty} s_n = \limsup_{n \to \infty} x_n.
	\]
	
	First suppose that $I = S$. Since $i_n \leq x_n \leq s_n$ (see Equation \ref{eq:ineq_sup_inf}), the Sandwich Lemma \ref{lem*sandwich_seq} implies that the sequence $(x_n)_{n=0}^{\infty}$ converges, and its limit equals $I = S$.
	
	Conversely, assume that $(x_n)_{n=0}^{\infty}$ converges to $A \in \mathbb{R}$. Given $\varepsilon > 0$, there exists $N \in \mathbb{N}$ such that
	\[
		A - \varepsilon < x_n < A + \varepsilon \qquad \forall n \geq N.
	\]
	Then for all $n \geq N$, the same inequalities holds for $i_n$ and $s_n$, i.e.,
	\[
		A - \varepsilon \leq i_n \leq s_n \leq A + \varepsilon.
	\]
	Taking limits and using Proposition \ref{prop*lim_ineq}, we obtain
	\[
		A - \varepsilon \leq I \leq S \leq A + \varepsilon.
	\]
	Since $\varepsilon > 0$ is arbitrary, it follows that $A = I = S$, which proves the result.\qedhere
\end{proof}

