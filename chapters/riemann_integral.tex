%
% (c) 2025 Autor, ETH Zürich
%
% !TEX root = main.tex
% !TEX encoding = UTF-8
%

\section{The Riemann Integral}

\subsection{Step Functions and their Integral}
\subsubsection{Decompositions and Step Functions}

\begin{definition}{Partitions}{partitions}
	Two sets $A, B$ are called \textbf{disjoint} if $A \cap B = \emptyset$. For a collection $\mathcal{A}$ of sets, we say that the sets in $\mathcal{A}$ are pairwise disjoint, if for all $A_1, A_2 \in \mathcal{A}$ with $A_1 \neq A_2$ it holds that $A_1 \cap A_2 = \emptyset$.
	
	Let $X$ be a set. A Partition of $X$ is a family $\mathcal{P}$ of non-empty pairwise disjoint subsets of $X$ such that
	\[
		X = \bigcup_{P \in \mathcal{P}} P.
	\]
\end{definition}

\begin{definition}{Decomposition of an Interval}{decomp_interval}
	A \textbf{decomposition} of $[a, b]$ is a finite sequence of points
	\[
		a = x_0 < x_1 < \; \hdots \; < x_{n-1} < x_n = b,
	\]
	with $n \in \mathbb{N}$. The points $x_0, \hdots , x_n$ are called the \textbf{division points} of the decomposition.
\end{definition}

Formally, a decomposition of $[a,b]$ is a finite subset of $[a,b]$ containing $a$ and $b$, together with the ordering of its elements. Each decomposition induces a natural partition of $[a,b]$, i.e.,
\[
	[a,b] = \{x_0\} \cup (x_0, x_1) \cup \{x_1\} \cup \; \hdots \; \cup (x_{n-1}, x_n) \cup \{x_n\},
\]
which we will use implicitly from now on.

A decomposition
\[
	a = y_0  < y_1 < \; \hdots \; < y_m = b
\]
is called a \textbf{refinement} of the decomposition
\[
	a = x_0 < x_1 < \; \hdots \; < x_n = b
\]
if
\[
	\{x_0, x_1, \hdots, x_n\} \subseteq \{y_0, y_1, \hdots , y_m\}.
\]
The notion of refinement defines a partial order on the set of all decompositions of $[a,b]$. Note that any two decompositions of $[a,b]$ admit a common refinement given by the union of all division points.

\begin{definition}{Step Functions}{step_func}
	A function $f:[a,b] \to \mathbb{R}$ is called a \textbf{step function} if there exists a decomposition
	\[
		a = x_0 < x_1 < \; \hdots \; < x_n = b
	\]
	such that, for each $k = 1, \hdots , n$ the restriction of $f$ to the open interval $(x_{k-1}, x_k)$ is constant. In this case, we say that $f$ is a step function \textit{with respect to} the decomposition $a= x_0  < x_1 < \; \hdots \; < x_n = b$.
\end{definition}

\begin{proposition}{Linearity of the Space of Step Functions}{lin_step_func}
	Let $f, g:[a, b] \to \mathbb{R}$ be step functions, and $\alpha, \beta \in \mathbb{R}$. Then $\alpha f + \beta g$ is also a step function.
\end{proposition}

\begin{proof}
	Let $f$ be a step function with respect to the decomposition $a = x_0 < x_1 <\; \hdots \; < x_n = b$, and let $g$ be a step function with respect to the decomposition $a = y_0 < y_1 <\; \hdots \; < y_m = b$. The union of all division points $\{x_0, x_1, \hdots , x_n\} \cup \{y_0, y_1, \hdots , y_m\}$ defines a new decomposition
	\[
		a = z_0 < z_1 < \; \hdots \; < z_N = b
	\]
	that is a common refinement of the two. Since both $f$ and $g$ are constant on each open interval $(z_{k - 1}, z_k)$, so is the function $\alpha f + \beta g$. Thus, $\alpha f + \beta g$ is also a step function with respect to this decomposition. \qedhere
\end{proof}

\begin{remark}
	As in the proof of Proposition \ref{prop*lin_step_func}, one can show that the product of two step functions is again a step function.  Moreover, step functions are bounded, since they take only finitely many values.
\end{remark}

\subsubsection{The Integral of a Step Function}

\begin{definition}{Integral of a Step Function}{integral_step_func}
	Let $f:[a,b] \to \mathbb{R}$ be a step function with respect to a decomposition
	\[
		a = x_0 < x_1 <\; \hdots \; < x_n = b.
	\]
	We define the \textbf{integral} of $f$ on $[a, b]$ as the real number
	\begin{equation}
		\label{eq:int_step_func}
		\int_{a}^{b} f(x)\, dx = \sum_{k=1}^{n} c_k (x_{k} - x_{k-1}),
	\end{equation}
	where $c_k$ denotes the constant value of $f$ on the interval $(x_{k - 1}, x_k)$.
\end{definition}

\begin{proposition}{Linearity of the Integral of Step Functions}{lin_int_step_func}
	Let $f,g:[a,b] \to \mathbb{R}$ be step functions, and let $\alpha , \beta \in \mathbb{R}$. Then
	\[
		\int_{a}^{b} (\alpha f + \beta g)(x) \, dx = \alpha \int_{a}^{b} f(x) \, dx + \beta \int_{a }^{b} g(x) \, dx.
	\]
\end{proposition}

\begin{proof}
	As in the proof of Proposition \ref{prop*lin_step_func}, we can find a decomposition $a = x_0 < x_1 <\; \hdots \; < x_n = b$ such that both $f$ and $g$ (and hence $\alpha f + \beta g$) are constant on the interval $(x_{k-1}, x_k)$. If $f$ takes the values $c_k$ and $g$ the values $d_k$ on $(x_{k-1}, x_k)$, then $\alpha f + \beta g$ takes the value $\alpha c_k + \beta d_k$. Thus,
	\begin{align*}
		\int_{a}^{b} (\alpha f + \beta g)(x) \, dx &= \sum_{k=1}^{n} (\alpha c_k + \beta d_k) (x_k - x_{k-1})\\
		&= \alpha \sum_{k=1}^{n} c_k (x_k - x_{k-1}) + \beta \sum_{k=1}^{n} d_k (x_k - x_{k-1})\\
		&= \alpha \int_{a}^{b} f(x) \, dx + \beta \int_{a}^{b} g(x) \, dx,
	\end{align*}
	as claimed. \qedhere
\end{proof}

\begin{proposition}{Monotonicity of the Integral of Step Functions}{mono_int_step_func}
	Let $f,g:[a,b] \to \mathbb{R}$ be step functions such that $f \leq g$. Then
	\[
		\int_{a}^{b} f(x) \, dx \leq \int_{a}^{b} g(x) \, dx.
	\]
\end{proposition}

\begin{proof}
	As in the proof of Proposition \ref{prop*lin_step_func}, we can find a decomposition $a = x_0 < x_1 <\; \hdots \; < x_n = b$ such that both $f$ and $g$ are constant on each interval $(x_{k-1}, x_k)$. Writing $c_k$ and $d_k$ for their respective values, the assumption $f \leq g$ implies that $c_k \leq d_k$ for all $k = 1, \hdots , n$. Hence
	\[
		\int_{a}^{b} f(x) \, dx = \sum_{k=1}^{n} c_k (x_k - x_{k-1}) \leq \sum_{k=1}^{n} d_k (x_k - x_{k-1}) = \int_{a}^{b} g(x) \, dx. \qedhere
	\]
\end{proof}

Applying Proposition \ref{prop*mono_int_step_func} with $g \equiv 0$, we deduce the following corollary.

\begin{corollary}{Positivity of the Integral of Step Functions}{pos_int_step_func}
	If $f:[a,b] \to \mathbb{R}$ is a step functions such that $f(x) \geq 0$ for all $x \in [a,b]$, then
	\[
		\int_{a}^{b} f(x) \, dx \geq 0.
	\]
\end{corollary}

\subsection{Definition and First Properties of the Riemann Integral}
As in the last section, we consider functions on a compact interval $[a,b] \subseteq \mathbb{R}$. To alleviate notation, we write $\mathcal{SF}$ for the set of step functions on $[a,b]$. Also, we often write $\int_{a}^{b} f\, dx$ in place of $\int_{a}^{b} f(x) \, dx$.

\subsubsection{Integrability of Real-Valued Functions}
Before defining lower and upper sums, we recall a simple but useful property of the supremum and infimum of two related sets. We use this fact several times in what follows.

\begin{definition}{Relation Between Supremum and Infimum}{relation_sup_inf}
	Let $A, B \subseteq \mathbb{R}$ be non-empty sets such that $s \leq t$ for all $s \in A$ and $t \in B$. Then
	\begin{equation}
		\label{eq:relation_sup_inf}
		\sup A \leq \inf B.
	\end{equation}
	Moreover,
	\begin{equation}
		\label{eq:relation_sup_inf_2}
		\sup A = \inf B \quad \Leftrightarrow \quad \forall \varepsilon > 0 \; \exists s \in A \, \exists t \in B \text{ such that }t - s < \varepsilon.
	\end{equation}
\end{definition}


\begin{definition}{Lower and Upper Sums}{lower_upper_sums}
	Let $f:[a, b] \to \mathbb{R}$ be a function. Define the sets of \textbf{lower sums} $\mathcal{L}(f) \subseteq \mathbb{R}$ and \textbf{upper sums} $\mathcal{U}(f) \subseteq \mathbb{R}$ by 
	\[
		\mathcal{L}(f) = \biggl\{\int_{a}^{b} \ell \, dx \; \bigg|\; \ell \in \mathcal{SF} \text{ and } \ell \leq f\biggr\}, \qquad \mathcal{U}(f) = \biggl\{\int_{a}^{b} u \, dx \;\bigg|\; u \in \mathcal{SF} \text{ and } f \leq u\biggr\}.
	\]
\end{definition}
If $f$ is bounded, then these sets are non-empty. Indeed, if $|f| \leq M$, then the constant step functions
\[
	\ell(x) = -M \quad \forall x \in [a,b], \qquad u(x) = M \quad \forall x\in [a,b],
\]
satisfy $\ell \in \mathcal{L}(f)$ and $u \in \mathcal{U}(f)$.

For $\ell, u \in \mathcal{SF}$ with $\ell \leq f \leq u$, Proposition \ref{prop*mono_int_step_func} gives
\[
	\int_{a}^{b} \ell \, dx \leq \int_{a}^{b } u \, dx.
\]
This implies that $s \leq t$ for all $s \in \mathcal{L}(f)$ and $t \in \mathcal{U}(f)$, so Equation \eqref{eq:relation_sup_inf} yields
\[
	\sup \mathcal{L}(f) \leq \inf \mathcal{U}(f).
\]

\begin{definition}{Riemann Integral}{riemann_integral}
	A bounded function $f:[a,b] \to \mathbb{R}$ is \textbf{Riemann integrable} if $\sup \mathcal{L}(f) = \inf \mathcal{U}(f)$. In this case, this common value is called the \textbf{Riemann integral} of $f$, and we write
	\[
		\int_{a}^{b} f \, dx = \sup \mathcal{L}(f) = \inf \mathcal{U}(f).
	\]
	We call $a$ the \textbf{lower (integration) limit} and $b$ the \textbf{upper (integration) limit}, and the function $f$ the \textbf{integrand} of the integral $\int_{a}^{b} f\, dx$. If $f \geq 0$ is Riemann integrable, we interpret the number $\int_{a}^{b} f\, dx$ as the \textbf{area} of the set
	\[
		\{(x, y) \in \mathbb{R}^2 \;|\; a \leq x \leq b, \quad 0 \leq y \leq f(x)\}.
	\]
\end{definition}

\begin{proposition}{Riemann Integrability Condition}{int_cond}
	Let $f:[a,b] \to \mathbb{R}$ be bounded. Then $f$ is Riemann integrable if and only if for every $\varepsilon > 0$ there exists step functions $\ell, u \in \mathcal{SF}$ such that
	\[
		\ell \leq f \leq u \qquad \text{and} \qquad \int_{a}^{b} (u - \ell) < \varepsilon.
	\]
	In this case,
	\[
		\left|\int_{a}^{b} f \, dx - \int_{a}^{b} \ell \, dx\right| < \varepsilon, \qquad \left|\int_{a}^{b} u \, dx - \int_{a}^{b} f \, dx\right| < \varepsilon.	
	\]
\end{proposition}

\begin{proof}
	By Equation \eqref{eq:relation_sup_inf_2} applied with $A = \sup \mathcal{L}(f)$ and $B = \inf \mathcal{U}(f)$ we obtain
	\begin{align*}
		f \text{ is Riemann integrable} \quad &\Leftrightarrow \quad \sup \mathcal{L}(f) = \mathcal{U}(f)\\
		&\Leftrightarrow \quad \forall \varepsilon > 0 \; \exists s \in \mathcal{L}(f) \, \exists t \in \mathcal{U}(f):\; t - s < \varepsilon\\
		&\Leftrightarrow \quad \forall \varepsilon > 0\; \exists \ell, u \in \mathcal{SF}: \; \ell \leq f \leq u \text{ and } \int_{a}^{b} u \, dx - \int_{a}^{b} \ell \, dx < \varepsilon\\
		&\Leftrightarrow \quad \forall \varepsilon > 0\; \exists \ell, u \in \mathcal{SF}: \; \ell \leq f \leq u \text{ and } \int_{a}^{b} (u - \ell) \, dx < \varepsilon,
	\end{align*}
	where we used Proposition \ref{prop*lin_int_step_func} to deduce that
	\[
		\int_{a}^{b}u \, dx - \int_{a}^{b} \ell \, dx = \int_{a}^{b}(u-\ell)\, dx
	\]
	Finally, the concluding inequalities follow from
	\[
		\int_{a}^{b} \ell \, dx \leq \int_{a}^{b} f\, dx \leq \int_{a}^{b} u \, dx \qquad \text{and} \qquad \int_{a}^{b}(u - \ell )\, dx < \varepsilon. \qedhere
	\]
\end{proof}

\subsubsection*{Example}
Not all functions are Riemann integrable. Indeed, consider the function $f:[0,1] \to \mathbb{R}$ defined by
\[
	f(x) = \begin{cases}
		1, \qquad &x \in \mathbb{Q},\\
		0, \qquad &x \notin \mathbb{Q}.
	\end{cases}
\]
We claim that $f$ is not Riemann integrable.

Let $u \in \mathcal{SF}$ with $f \leq u$, and let $0 = x_0 < \; \hdots \; < x_1 = 1$ be a decomposition such that $u$ is constant $c_k$ on $(x_{k-1}, x_{k})$. Since $\mathbb{Q}$ is dense in $\mathbb{R}$, there exists $x \in (x_{k-1}, x_k) \cap \mathbb{Q}$, hence $1 = f(x) \leq u(x) = c_k$, so $c_k \geq 1$. Therefore,
\[
	\int_{0}^{1} u(x)\, dx = \sum_{k=1}^{n} c_k (x_k - x_{k-1}) \geq \sum_{k=1}^{n} (x_k - x_{k-1}) = x_n - x_0 = 1.
\]
Thus $\inf \mathcal{U}(f) = 1$, and taking $u \equiv 1$ gives $\inf \mathcal{U}(f)$. A similar argument with lower sums shows that $\sup \mathcal{L}(f) = 0$. Hence, $f$ is not Riemann integrable.

\begin{theorem}{Linearity of the Riemann Integral}{lin_int}
	If $f,g:[a,b] \to \mathbb{R}$ are integrable and $\alpha , \beta \in \mathbb{R}$, then $\alpha f + \beta g$ is integrable, and
	\[
		\int_{a}^{b} (\alpha f + \beta g) \, dx = \alpha \int_{a}^{b} f \, dx + \beta \int_{a}^{b} g\, dx
	\]
\end{theorem}

\begin{proof}
	Given $\varepsilon > 0$, Proposition \ref{prop*int_cond} yields step functions $\ell_1, \ell_2, u_1, u_2$ with
	\[
		\ell_1 \leq f \leq u_1, \quad \ell_2 \leq g \leq u_2,  \quad \int_{a}^{b} (u_1 - \ell_1) \, dx < \varepsilon, \quad \int_{a}^{b}(u_2 - \ell_2) \, dx < \varepsilon,
	\]
	and
	\[
		\left|\int_{a}^{b}f \, dx - \int_{a}^{b} \ell_1 \, dx\right| < \varepsilon, \qquad \left|\int_{a}^{b} g \, dx - \int_{a}^{b} \ell_2 \, dx \right| < \varepsilon.
	\]
	
	Assume first $\alpha, \beta \geq 0$. Then
	\[
		\alpha \ell_1 + \beta \ell_2 \leq \alpha f + \beta g \leq \alpha u_1 + \beta u_2,
	\]
	and
	\[
		\int_{a}^{b}[(\alpha u_1 + \beta u_2) - (\alpha \ell_1 + \beta \ell_2)] \, dx = \alpha \int_{a}^{b} (u_1 - \ell_1) \, dx + \beta \int_{a}^{b} (u_2 - \ell_2) \, dx < (\alpha + \beta) \varepsilon.
	\]
	Since $\varepsilon > 0$ is arbitrary, this proves that $\alpha f + \beta g$ is integrable. Moreover, by the triangle inequality and Proposition \ref{prop*lin_int_step_func}, we have that
	\begin{align*}
		\bigg|\int_{a}^{b} (\alpha f + \beta g) \, dx - \alpha \int_{a}^{b} f\, dx - \beta \int_{a}^{b} g \, dx\bigg| &\leq \bigg|\int_{a}^{b} (\alpha f + \beta g)\, dx - \int_{a}^{b} (\alpha \ell_1 + \beta \ell_2) \, dx\bigg|\\
		&+ \bigg|\underbrace{\int_{a}^{b} (\alpha \ell_1 + \beta \ell_2) \, dx - \alpha \int_{a}^{b} \ell_1\, dx - \beta \int_{a}^{b} \ell_2 \, dx}_{=0}\bigg|\\
		&+ \alpha \bigg|\int_{a}^{b} \ell_1 \, dx - \int_{a}^{b} f\, dx\bigg| + \beta \bigg|\int_{a}^{b} \ell_2 \, dx - \int_{a}^{b} g\, dx\bigg|\\
		&\leq (\alpha + \beta) \varepsilon + \alpha \varepsilon + \beta \varepsilon = 2 (\alpha + \beta) \varepsilon.
	\end{align*}
	Since $\varepsilon > 0$ is arbitrary, the linearity follows.
	
	The case when one of $\alpha, \beta$ is negative is analogous, but one needs to reverse the corresponding inequalities. For instance, if $\alpha \geq 0$ and $\beta < 0$, then
	\[
		\alpha \ell_1 + \beta u_2 \leq \alpha f + \beta g \leq \alpha u_1 + \beta \ell_2,
	\]
	and
	\[
		\int_{a}^{b}[(\alpha u_1 + \beta \ell_2) - (\alpha \ell_1 + \beta u_2)]\, dx = \alpha \int_{a}^{b} (u_1 - \ell_1) \, dx + |\beta| \int_{a}^{b} (u_2 - \ell_2) \, dx < (\alpha + |\beta|)\varepsilon.
	\]
	This implies again that $\alpha f + \beta g$ is integrabl, and the linearity identity holds similarly. \qedhere
\end{proof}

\begin{proposition}{Monotonicity of the Riemann Integral}{mono_int}
	Let $f,g:[a,b] \to \mathbb{R}$ be integrable. If $f \leq g$, then
	\[
		\int_{a}^{b} f \, dx \leq \int_{a}^{b} g \, dx.
	\]
\end{proposition}
	
\begin{proof}
	Since $f \leq g$, for any step function $\ell$ with $\ell \leq f$ we have $\ell \leq g$. This implies that $\mathcal{L}(f) \subseteq \mathcal{L}(g)$, therefore
	\[
		\int_{a}^{b} f\, dx = \sup \mathcal{L}(f) \leq \sup \mathcal{L}(g) = \int_{a}^{b} g\, dx. \qedhere
	\]
\end{proof}

\begin{definition}{Positive and Negative Parts}{pos_neg_parts}
	Given a function $f:D \to \mathbb{R}$, we define its \textbf{positive part} $f^+:D \to \mathbb{R}$ and \textbf{negative part} $f^-:D \to \mathbb{R}$ by
	\[
		f^+(x) = \max\{0, f(x)\}, \qquad f^-(x) = -\min\{0, f(x)\}.
	\]
	These satisfy
	\[
		f = f^+ - f^-, \quad |f| = f^+ + f^-, \quad f^+ = \frac{|f| + f}{2}, \quad f^- = \frac{|f| - f}{2}.
	\]
	Moreover, for any functions $f, g:D \to \mathbb{R}$,
	\[
		f \leq g \quad \Rightarrow \quad f^+ \leq g^+, \qquad f \leq g \quad \Rightarrow \quad f^- \geq g^-.
	\]
\end{definition}

\begin{remark}
	\label{rmk:prop_pos_part}
	For any real numbers $z_1, z_2 \in \mathbb{R}$, one has
	\begin{equation}
		\label{eq:ineq_pos_part}
		(z_1 - z_2)^+ \geq z_1^+ - z_2^+.
	\end{equation}
	Indeed, since $z^+ \geq z$ and $z^+ \geq 0$ for all $z \in \mathbb{R}$, by applying these inequalities with $z = z_1 - z_2$ and $z = z_2$ we obtain
	\[
		z_1 = (z_1 - z_2) + z_2 \leq (z_1 - z_2)^+ + z_2^+ \quad \text{and} \quad 0 \leq (z_1 - z_2)^+ + z_2^+.
	\]
	Hence $(z_1 - z_2)^+ + z_2^+$ is greater or equal to both $z_1$ and 0, and therefore
	\[
		z_1^+ = \max\{z_1, 0\} \leq (z_1 - z_2)^+ + z_2^+,
	\]
	which yields Equation \eqref{eq:ineq_pos_part} after rearranging.
\end{remark}

\begin{theorem}{Triangle Inequality for the Riemann Integral}{triangle_ineq_int}
	Let $f,g:[a,b] \to \mathbb{R}$ be integrable. Then $f^+$, $f^-$, and $|f|$ are integrable, and
	\[
		\left|\int_{a}^{b} f \, dx\right| \leq \int_{a}^{b} |f|\, dx.
	\]
\end{theorem}

\begin{proof}
	Fix $\varepsilon > 0$. Since $f$ is integrable, there exists step functions $\ell \leq f \leq u$ with $\int_{a}^{b}(u - \ell)\, dx < \varepsilon$. Then $\ell^+$ and $u^+$ are step functions with $\ell^+ \leq f^+ \leq u^+$.
	
	Since $u - \ell \geq 0$, we have $(u - \ell) = (u - \ell)^+$. Moreover, applying Equation \eqref{eq:ineq_pos_part} with $z_1 = u(x)$ and $z_2 = \ell(x)$, we obtain
	\[
		(u(x) - \ell(x))^+ \geq u(x)^+ - \ell(x)^+ \qquad \forall x \in [a,b].
	\]
	Hence
	\[
		\int_{a}^{b}(u^+ - \ell^+) \, dx \leq \int_{a}^{b} (u - \ell)^+\, dx = \int_{a}^{b} (u - \ell) \, dx < \varepsilon,
	\]
	so $f^+$ is integrable. By Theorem \ref{theo*lin_int}, also $f^- = f^+ - f$ and $|f| = 2f^+ - f$ are integrable. Finally,
	\[
		\left|\int_{a}^{b} f\, dx\right| = \left|\int_{a}^{b}f^+ \, dx - \int_{a}^{b} f^- \, dx\right| \leq \int_{a}^{b} f^+ \, dx + \int_{a}^{b} f^- \, dx = \int_{a}^{b} |f| \, dx. \qedhere
	\]
\end{proof}

\begin{remark}
	\label{rmk:int_seperate}
	Let $a < b < c$. A function $f:[a,c] \to \mathbb{R}$ is integrable if and only if $f|_{[a,b]}$ and $f|_{[b,c]}$ are integrable, and
	\[
		\int_{a}^{c} f \, dx = \int_{a}^{b} f|_{[a,b]} \, dx + \int_{b}^{c} f|_{[b,c]} \, dx.
	\]
\end{remark}

\subsection{Integrability Theorems}

\subsubsection{Integrability of Monotone Functions}
As before we work on a compact interval $[a,b] \subseteq \mathbb{R}$. Note that very monotone function $f:[a,b] \to \mathbb{R}$ is bounded; for instance, if $f$ is increasing then $f(a)$ is a lower bound and $f(b)$ is an upper bound.

\begin{theorem}{Monotone Functions are Integrable}{mono_func_int}
	Every monotone function $f:[a,b] \to \mathbb{R}$ is Riemann integrable.
\end{theorem}

\begin{proof}
	W.l.o.g, $f$ is increasing (otherwise replace $f$ by $-f$ and use Theorem \ref{theo*lin_int}). We want to apply Proposition \ref{prop*int_cond}. Given $\varepsilon > 0$, we need to construct step functions $\ell, u \in \mathcal{SF}$ such that $\ell \leq f \leq u$ and $\int_{a}^{b}(u - \ell) <, dx < \varepsilon$.
	
	Fix $n \in \mathbb{N}$ (to be chosen later) and the uniform partition
	\[
		a = x_0 < x_1 < \; \hdots \; < x_n = b, \qquad x_k = a + \frac{k}{n}(b - a).
	\]
	Define the step functions $ell, u:[a,b] \to \mathbb{R}$ as
	\begin{align*}
		\ell(x) = f(x_{k-1}) \quad \text{and} \quad u(x) = f(x_k) \qquad &\text{for } x \in (x_{k-1}, x_k), \quad k = 1, \hdots, n,\\
		\ell(x) = u(x) = f(x) \qquad &\text{for } x \in \{x_0, \hdots , x_n\}.
	\end{align*}
	Note that, since $f$ is increasing, $\ell \leq f \leq u$. Moreover, for each $k$ we have $u - \ell = f(x_k) - f(x_{k-1})$ on $(x_{k-1}, x_k)$. Recalling that $x_k - x_{k-1} = \frac{b - a}{n}$, this yields
	\begin{align*}
		\int_{a}^{b} (u - \ell) \, dx &= \sum_{k=1}^{n} (f(x_k) - f(x_{k-1})) (x_k - x_{k-1}) = \frac{b - a}{n} \sum_{k=1}^{n} (f(x_k) - f(x_{k-1})) \\
		&= \frac{b - a}{n} (f(x_n) - f(x_0)) = \frac{b - a}{n} (f(b) - f(a)).
	\end{align*}
	Choosing $n \in \mathbb{N}$ so large that $\frac{b - a}{n}(f(b) - f(a)) < \varepsilon$, Proposition \ref{prop*int_cond} implies that $f$ is Riemann integrable. \qedhere
\end{proof}

\begin{definition}{Piecewise Monotone Functions}{piecewise_mono_func}
	A function $f:[a,b] \to \mathbb{R}$ is \textbf{piecewise monotone} if there exists a decomposition
	\[
		a = x_0 < x_1 < \; \hdots \; < x_n = b
	\]
	such that $f|_{(x_{k-1}, x_k)}$ is monotone for every $k = 1, \hdots , n$.
\end{definition}

\begin{corollary}{Piecewise Monotone Functions are Integrable}{piecewise_mono_func_int}
	Every bounded piecewise monotone function $f:[a,b] \to \mathbb{R}$ is Riemann integrable.
\end{corollary}

\subsubsection{Integrability of Continuous Functions}
Using boundedness and uniform continuity on compact intervals (Theorems \ref{theo*boundedness} and \ref{theo*unif_cont_compact_int}), we can prove that continuous functions are integrable.

\begin{theorem}{Continuous Functions are Integrable}{cont_func_int}
	Every continuous function $f:[a,b] \to \mathbb{R}$ is Riemann integrable.
\end{theorem}

\begin{proof}
	Let $f:[a,b] \to \mathbb{R}$ be continuous and fix $\varepsilon > 0$. By uniform continuity \ref{theo*unif_cont_compact_int}, there exists $\delta >0$ such that
	\begin{equation}
		\label{eq:unif_cont}
		|x - y| < \delta \quad \Rightarrow \quad |f(x) - f(y)|< \varepsilon \qquad \forall x,y \in [a,b].
	\end{equation}
	Choose a partition $a = x_0 < x_1 < \; \hdots \; < x_n = b$ with $x_k - x_{k-1} < \delta$. For each $k$ set 
	\[
		c_k = \min\{f(x) \;|\; x_{k-1} \leq x \leq x_k\}, \qquad d_k = \max\{f(x) \,|\; x_{k-1} \leq x \leq x_k\},
	\]
	which exists by Theorem \ref{theo*extreme_val_theo}, and let $y_k, z_k \in [x_{k-1}, x_k]$ satisfy $f(y_k) = c_k$ and $f(z_k) = d_k$. Then since, $|y_k - z_k| \leq x_k - x_{k-1} < \delta$, Equation \eqref{eq:unif_cont} yields $d_k - c_k < \varepsilon$.
	
	Define now the step functions $\ell, u:[a,b] \to \mathbb{R}$ as
	\begin{align*}
		\ell(x) = c_k, \quad \text{and} \quad u(x) = d_k \qquad &\text{for } x \in (x_{k-1}, x_k), \quad k = 1,\hdots , n,\\
		\ell(x) = u(x) = f(x) \qquad &\text{for } x \in \{x_0, \hdots , x_n\}.
	\end{align*}
	Then $\ell \leq f \leq u$ and $u - \ell = d_k - c_k$ on $(x_{k-1} , x_k)$, hence
	\[
		\int_{a}^{b} (u - \ell) \, dx = \sum_{k=1}^{n} (d_k - c_k)(x_k - x_{k-1}) < \varepsilon \sum_{k=1}^{n} (x_k - x_{k-1}) = \varepsilon (b - a).
	\]
	Since $\varepsilon > 0$ is arbitrary, $f$ is integrable. \qedhere
\end{proof}

\begin{definition}{Piecewise Continuous Functions}{piecewise_cont_func}
	A function $f:[a,b] \to \mathbb{R}$ is \textbf{piecewise continuous} if there exists a decomposition
	\[
		a = x_0 < x_1 < \; \hdots \; < x_n = b
	\]
	such that $f|_{(x_{k-1}, x_k)}$ is continuous for all $k$ and both one sided limits $\lim_{x \to x_{k-1}^+} f(x)$ and $\lim_{x \to x_k^-} f(x)$ exist. Equivalently, each $f|_{(x_{k-1}, x_k)}$ extends to a continuous function on $[x_{k-1}, x_k]$.
\end{definition}

\begin{corollary}{Piecewise Continuous Functions are Integrable}{piecewise_cont_func_int}
	Every piecewise continuous function $f:[a,b] \to \mathbb{R}$ is Riemann integrable.
\end{corollary}

\subsubsection{Integration and Sequences of Functions}
Let $(f_n)_{n=0}^{\infty}$ with $f_n:[a,b] \to \mathbb{R}$ be a sequence of integrable functions. Assume that $f_n$ converges pointwise or uniformly to $f:[a,b] \to \mathbb{R}$. Is $f$ integrable? And if so, does
\[
	\lim_{n \to \infty} \int_{a}^{b} f_n \, dx = \int_{a}^{b} f\, dx
\]
hold?

In general, the pointwise limit of integrable functions need not to be integrable. Also, as the following example shows, even when the pointwise limit of $f$ is integrable, one may have that $\lim_{n \to \infty} \int f_n \neq \int f$.

\subsubsection*{Example}
Let $D = [0,1]$ and define $f_n:D \to \mathbb{R}$ by
\[
	f_n(x) = \begin{cases}
		n^2x \qquad &\text{if } 0 \leq x \leq \frac{1}{2n},\\
		n^2(\frac{1}{n} - x) \qquad &\text{if } \frac{1}{2n} \leq x \leq \frac{1}{n},\\
		0 \qquad &\text{if } \frac{1}{n} \leq x \leq 1.
	\end{cases}
\]
Each $f_n$ is continuous (hence integrable). Also, its graph is a triangle of base $\frac{1}{n}$ and height $\frac{n}{2}$, so 
\[
	\int_{0}^{1} f_n(x) \, dx = \frac{1}{2}\cdot \frac{1}{n} \cdot \frac{n}{2} = \frac{1}{4}.
\]
Moreover, $f_n(0) = 0$ for all $n$, and for every $x > 0$ we have $f_n(x) = 0$ for all $n > 1/x$, hence $f_n(x) \to 0$. Thus, $f_n$ converges pointwise to the constant function $f = 0$, but
\[
	\int_{0}^{1}f_n(x) \, dx = \frac{1}{4} \neq 0 = \int_{0}^{1} f(x) \, dx.
\]

On the other hand, as the next result shows, uniform convergence is sufficient for both integrability of the limit and interchange of limit and integral.

\begin{theorem}{Uniform Convergence and Riemann Integrals Commute}{unif_conv_int_commute}
	Let $(f_n)_{n=0}^{\infty}$, with $f_n:[a,b] \to \mathbb{R}$, be a sequence of integrable functions converging uniformly to $f:[a,b] \to \mathbb{R}$. Then $f$ is integrable and
	\begin{equation}
		\label{eq:unif_conv_int}
		\int_{a}^{b}f \, dx = \lim_{n \to \infty} \int_{a}^{b} f_n \, dx.
	\end{equation}
\end{theorem}

\begin{proof}
	Fix $\varepsilon > 0$. By uniform convergence, there exists $N \in \mathbb{N}$ such that $|f_n - f| < \varepsilon$ on $[a,b]$ for all $n \geq N$.
	
	Since $f_N$ is integrable, there exists step functions $\ell, u \in \mathcal{SF}$ with $\ell \leq f_N \leq u$ and $\int_{a}^{b} (u - \ell) \, dx < \varepsilon$. Set $\hat{\ell} = \ell - \varepsilon$ and $\hat{u} = u + \varepsilon$. Then $\hat{\ell}, \hat{u} \in \mathcal{SF}$. Also, since $|f_N - f| < \varepsilon$,
	\[
	\hat{\ell} = \ell - \varepsilon \leq f_N - \varepsilon \leq f \leq f_N + \varepsilon \leq u + \varepsilon = \hat{u}
	\]
	and (because $\hat{u} - \hat{\ell} = u - \ell + 2\varepsilon$)
	\[
	\int_{a}^{b} (\hat{u} - \hat{\ell}) \, dx = \int_{a}^{b} (u - \ell) \, dx + 2\varepsilon(b-a) < \varepsilon + 2\varepsilon(b-a).
	\]
	As $\varepsilon > 0$ is arbitrary, Proposition \ref{prop*int_cond} yields that $f$ is integrable.
	
	Moreover, using monotonicity (Proposition \ref{prop*mono_int}) and the triangle inequality for the Riemann integral (Theorem \ref{theo*triangle_ineq_int}),
	\[
		\left|\int_{a}^{b}f \, dx - \int_{a}^{b}f_n \, dx\right| = \left|\int_{a}^{b} (f - f_n) \, dx\right| \leq \int_{a}^{b} |f - f_n|\, dx \leq \varepsilon (b-a) \qquad \forall n \geq N,
	\]
	proving Equation \eqref{eq:unif_conv_int}. \qedhere
\end{proof}