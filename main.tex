%
% main.tex -- BA von Maurin Doswald und Stephan Oseghale
%
% (c) 2025 Autor, ETH Zürich
%
% !TEX root = main.tex
% !TEX encoding = UTF-8
%

\documentclass[10pt, a4paper, twoside]{article}

\usepackage[utf8]{inputenc}
\usepackage[T1]{fontenc}
\usepackage[english]{babel}
\usepackage{amsmath, amssymb, bm, amsthm}
\usepackage{graphicx}
\usepackage{geometry}
\usepackage{setspace}
\usepackage{hyperref}
\usepackage{csquotes}
\usepackage{tocloft}
\usepackage{titlesec}
\usepackage{pdfpages}
\usepackage{enumitem}
\usepackage{float}
\usepackage[acronym, toc]{glossaries}
\usepackage[backend=bibtex, style=numeric]{biblatex}
\usepackage{tikz, tcolorbox}
\usepackage{multirow}
\usepackage[colorinlistoftodos, textwidth=20mm]{todonotes}
\setlength{\marginparwidth}{2cm}
\usepackage{xurl}
\selectlanguage{english}
\tcbuselibrary{theorems}

\newcounter{dnscounter}[section]

\newtcbtheorem[number within=section, use counter=dnscounter]{definition}{Definition}{colback=violet!5, colframe=violet!30, coltitle=black, fonttitle=\large}{def}
\newtcbtheorem[use counter=dnscounter, number within=section]{theorem}{Theorem}{colback=yellow!5!orange!5, colframe=yellow!25!orange!25, coltitle=black, fonttitle=\large, fontupper=\itshape, label separator=*}{theo}
\newtcbtheorem[use counter=dnscounter, number within=section]{lemma}{Lemma}{colback=yellow!5!orange!5, colframe=yellow!25!orange!25, coltitle=black, fontupper=\itshape, fonttitle=\large, label separator=*}{lem}
\newtcbtheorem[use counter=dnscounter, number within=section]{corollary}{Corollary}{colback=olive!5, colframe=olive!30, coltitle=black, fonttitle=\large, fontupper=\itshape,label separator=*}{cor}
\newtcbtheorem[use counter=dnscounter, number within=section]{proposition}{Proposition}{colback=olive!5, colframe=olive!30, coltitle=black, fonttitle=\large, fontupper=\itshape, label separator=*}{prop}


\newtheorem{remark}[dnscounter]{Remark}




\usetikzlibrary{shapes.geometric, arrows}





\geometry{left=3cm, right=2.5cm, top=2.5cm, bottom=2.5cm}

\relpenalty=10000
\binoppenalty=10000


\usepackage{fancyhdr}
\pagestyle{fancy}
\fancyhf{}
\fancyhead[LE]{\thepage}
\fancyhead[LO]{\nouppercase{\rightmark}}
\fancyhead[RE]{\nouppercase{\leftmark}}
\fancyhead[RO]{\thepage}


\onehalfspacing
%\singlespacing
%\setlength{\itemsep}{1pt} 

\fancypagestyle{plain}{              
	\fancyhf{}    
	\fancyhead[LE]{\thepage}
	\fancyhead[LO]{\nouppercase{\rightmark}}
	\fancyhead[RE]{\nouppercase{\leftmark}}
	\fancyhead[RO]{\thepage}                      
	%\fancyhead[L]{\nouppercase{\leftmark}}
	%\fancyhead[R]{\thepage}
	 \renewcommand{\headrulewidth}{0.4pt} 
}

\fancypagestyle{plain2}{              
	\fancyhf{}    
	\fancyhead[LE]{\thepage}
	\fancyhead[LO]{}
	\fancyhead[RE]{}
	\fancyhead[RO]{\thepage}                      
	%\fancyhead[L]{\nouppercase{\leftmark}}
	%\fancyhead[R]{\thepage}
	\renewcommand{\headrulewidth}{0.4pt} 
}


% Kapitel X wird entfernt bei \chapter
%\titleformat{\chapter}[display]
%{\normalfont\LARGE\bfseries} 
%{}                           
%{0pt}                        
%{\LARGE}                     

\titlespacing*{\chapter}{0pt}{-40pt}{20pt}
\newcommand{\HRule}[1]{\rule{\linewidth}{#1}}



\def\labelitemi{--}
%\addbibresource{references.bib}
\makeglossaries
%\input{chapters/akronyme.tex}

\numberwithin{equation}{section}



\begin{document}
	\begin{titlepage}
	\begin{center}
		\vspace*{1cm}
		\HRule{0.5pt}
		\Huge
		\textbf{Analysis I}\\
		\textbf{Theorems \& Lemmas}
		
		\vspace{0.5cm}
		\LARGE
		%Thesis Subtitle
		\HRule{2pt}
		\vspace{1.5cm}
		
		\Large
		\textbf{Stephan Oseghale}
		
		\vfill
		
		\large
		
		
		\vspace{0.8cm}
		
		
		\normalsize
		Mathematics Bachelor\\
		ETH Zürich\\
		\today
		
	\end{center}
\end{titlepage}	
	
	
	\pagenumbering{roman}	
%	\begin{abstract}
%		\input{chapters/abstract.tex}
%	\end{abstract}
%	\newpage

	\tableofcontents
	
%	\newpage
%	\listoffigures                       
%	\listoftables   
	
	\newpage    

%	\clearpage
%	\thispagestyle{plain2}
%	\vspace*{\fill}
%	\begin{footnotesize}
%		\begin{quote}
%			As regards electric instruments for producing sound, the enmity with which the few musicians who know them is manifest. They judge them superficially, consider them ugly, of small practical value, unnecessary... [Meanwhile, the inventors] undiscerningly want the new electric instruments to imitate the instruments now in use as faithfully as possible and to serve the music we already have. What is needed is an understanding of the possibilities of the new instruments. We must clearly evaluate the increase they bring to our own capacity for expression... The new instruments will produce an unforeseen music, as unlooked for as the instruments themselves... 
%		\end{quote}
%		\hfill\textit{- Chavez, 1936}\\
%	\end{footnotesize}
%	\vspace*{\fill} 
%	\clearpage
%	\thispagestyle{plain2}
%	\cleardoublepage
	
	\pagenumbering{arabic}
	
	%
% (c) 2025 Autor, ETH Zürich
%
% !TEX root = main.tex
% !TEX encoding = UTF-8
%

\section{Functions}

\begin{definition}{Functions/Maps/Transformations}{func}
	A \textbf{function} f from a set $X$ to a set $Y$ is an assignment of an element of $Y$ to each element of $X$.
	The element $y \in Y$ to which $x \in X$ is assigned to is denoted $f(x)$.
	We write $f:X \to Y$ and sometimes also speak of a \textbf{map}, \textbf{mapping} or a \textbf{transformation}.
	The set $X$ is the \textbf{domain} and the set $Y$ is the \textbf{codomain}.
	We refer to the set $X$ as \textbf{domain} or \textbf{domain of definition}, and the set $Y$ as \textbf{domain of values} or \textbf{codomain}.
	The set
	\[
		\{(x, f(x)) \;|\; x\in X\} \subseteq X \times Y
	\] 
	is called the \textbf{graph} of f.
	In the context of a function $f:X \to Y$, an element of the domain of definition is also called \textbf{argument}, and an element $y = f(x) \in Y$ assumed by the function, is also called \textbf{value} of the function.
	If $f:X \to Y$ is a function, one also writes
	\begin{align*}
		f:X &\to Y\\
		x &\mapsto f(x),
	\end{align*}
	where $f(X)$ could be a concrete formula.
	We pronounce ´$\mapsto$´ as ´is mapped to'.
	Two functions $f_1:X_1 \to Y_1$ and $f_2:X_2 \to Y_2$ are said to be equal if $X_1 = X_2$, $Y_1 = Y_2$ and $f_1(x) = f_2(x) \quad \forall x \in X_1$.
\end{definition}

\begin{definition}{Injective, Surjective and Bijective Functions}{inj_func}
	Let $f: X \to Y$ be a function. We call $f$:
	\begin{enumerate}
		\item \textbf{injective} (or an \textbf{injection}) if
		\[
			\forall x_1, x_2 \in X \,:\, x_1 \neq x_2 \Rightarrow f(x_1) \neq f(x_2);
		\]
		\item \textbf{surjective} (or a \textbf{sujection}) if
		\[
			\forall y \in Y\; \exists x \in X \,:\, f(x) = y;
		\]
		\item \textbf{bijective} (or a \textbf{bijection}) if $f$ is both injective and surjective.
	\end{enumerate}
\end{definition}
Thus, a function $f:X \to Y$ is \textit{not} injective if there exists distinct $x_1 \neq x_2 \in X$ such that $f(x_1) = f(x_2)$, and \textit{not} surjective if there exists $y \in Y$ such that $f(x) \neq y$ for all $x \in X$.

\begin{definition}{Image and Preimage of a Function}{img_func}
	For $f:X \to Y$ and $A \subseteq X$, define the \textbf{image} of $A$ under the function $f$ as
	\[
		f(A) := \{y \in Y \;|\; \exists x \in X : f(x) = y\}.
	\]
	For $B \subseteq Y$, define the \textbf{preimage} of $B$ under the function $f$ as
	\[
		f^{-1}(B) := \{x \in X \;|\; f(x) \in B\}.
	\]
\end{definition}

\begin{remark}
	Saying that $f:X \to Y$ is surjective is equivalent to $f(X) = Y$. Equivalently, $f$ is surjective if $f^{-1}(\{y\}) \neq \emptyset$ for all $y \in Y$. 
\end{remark}

\begin{definition}{Composition of Functions}{comp_func}
	Let $f:X \to Y$ and $g:Y \to Z$. The \textbf{composition} is $g \circ f: X \to Z$, defined by $(g \circ f)(x) = g(f(x))$ for all $x \in X$.
	
	\textbf{Associativity:} If $f:W \to X, g:X \to Y$ and $h:Y \to Z$, then
	\[
		h \circ (g \circ f) = (h \circ g) \circ f.
	\]
	Indeed, for all $w \in W$, we have
	\[
		h \circ (g \circ f)(w) = h((g \circ f)(w)) = h(g(f(w))) = (h \circ g)(f(w)) = ((h \circ g) \circ f)(w).
	\]
	Therefore, we may omit parentheses and write $h \circ g \circ f:W \to Z$.
\end{definition}

\begin{definition}{Identity and Inverse Function}{id_inv_func}
	Given a set $X$, the \textbf{identity function} $\operatorname{id}_{X}:X \to X$ is defined by
	\[
		\operatorname{id}_{X}(x) = x \qquad \forall x \in X.
	\]
	If $f:X \to Y$ is bijective, then there exists a unique function $g:Y \to X$ such that, for each $y \in Y$, the value $g(y)$ is the unique element $x \in X$ satisfying $f(x) = y$. With this definition,
	\[
		g \circ f = \operatorname{id}_{X} \qquad \text{and} \qquad f \circ g = \operatorname{id}_{Y}.
	\]
	The function $g$ is called the \textbf{inverse function} (or \textbf{inverse mapping}) of $f$, and is denoted by $f^{-1}$.
\end{definition}

\begin{remark}
	A function $f: X \to Y$ is bijective if and only if there exists a function $g: Y \to X$ such that $g \circ f = id_{X}$ and $f \circ g = id_{Y}$.
\end{remark}
	%
% (c) 2025 Autor, ETH Zürich
%
% !TEX root = main.tex
% !TEX encoding = UTF-8
%

\section{The Real Numbers}

\subsection{Groups, Rings, Fields}

\begin{definition}{Groups}{groups}
	A \textbf{group} is a non-empty set $G$ together with a rule (called an \textit{operation}) denoted by $\star : G \times G \to G$ that combines any two elements of G into another element of G. This operation must satisfy three conditions:
	\begin{itemize}
		\item \textbf{Associativity:} No matter how you place parentheses, the result is the same for all $a, b, c \in G$,
		\[
			(a \star b) \star c = a \star (b \star c).
		\]
		\item \textbf{Neutral element:} There is a special element $e \in G$ such that combining it with any $a \in G$ leaves $a$ unchanged, i.e.,
		\[
			\forall a \in G : a \star e = e \star a = a.
		\]
		\item \textbf{Inverse element:} Every $a \in G$ has a 'partner' $a^{-1} \in G$ that 'cancels it out', giving the neutral element, i.e.,
		\[
			a \star a^{-1} = a^{-1} \star a = e.
		\]
	\end{itemize}
	Note that, in general, one does not require that $a \star b = b \star a$. If the order of the operation does not matter, i.e., $a \star b = b \star a$ for all $a, b \in G$, the group is called \textbf{commutative} or \textbf{abelian}.
\end{definition}

\begin{lemma}{Basic Properties of Groups}{properties_groups}
	Let $(G, \star)$ be a group. Then:
	\begin{enumerate}
		\item The neutral element is unique.
		\item The inverse of an element is unique.
		\item The inverse of the inverse of an element is the element itself, namely $(a^{-1})^{-1} = a$ for all $a \in G$.
	\end{enumerate}
\end{lemma}

\begin{proof}
	1. Assume that, in addition to $e \in G$, we have a second element $e'$ with the property that $e' \star a = a \star e' = a$ for all elements $a \in G$. Then, we can choose $a = e$ to obtain
	\[
		e \star e' = e.
	\]
	Similarly, since $e$ is a neutral element, we have
	\[
		e \star e' = e'.
	\]
	Combining the two identities, we get
	\[
		e = e \star e' = e'.
	\]
	This proves that $e' = e$, so we speak of \textit{the} neutral element of a group.
	
	2. Assume that for an element $a \in G$, there exists two elements $b, c \in G$ that are both the inverse of $a$, namely
	\[
		a \star b = b \star a = e, \qquad a \star c = c \star a = e.
	\]
	Then, using associativity, we observe that
	\[
		b = b \star e = b \star (a \star c) = (b \star a) \star c = e \star c = c.
	\]
	This proves that the inverse of an element $a$ is unique, so we can speak of \textit{the} inverse element, and the notation $a^{-1}$ makes sense.
	
	3. Since $a \star a^{-1} = e$, we deduce that $a$ is the inverse element of $a^{-1}$, thus
	\begin{equation}
		\label{eq:inverse_element}
		(a^{-1})^{-1} = a. \qedhere
	\end{equation}
\end{proof}

Groups capture the idea of combining elements with a single operation. But to describe the arithmetic of numbers more faithfully, we also need a second operation (as we do with addition and multiplication). This leads us to the notion of \textit{rings} and \textit{fields}.

\begin{definition}{Rings and Fields}{rings_fields}
	A \textbf{ring} is a non-empty set $R$ in which we can both 'add' and 'multiply' elements with two operations '$+$' and '$\cdot$'. Also, these two operations are compatible with each other. More precisely:
	\begin{itemize}
		\item $(R, +)$ is a \textbf{commutative group}, with neutral element denoted 0.
		\item Multiplication $\cdot$ is \textbf{associative}, has a \textbf{neutral element} (usually written as 1), and \textbf{distributes over addition}, i.e.,
		\[
			a \cdot (b + c) = a \cdot b + a \cdot c, \qquad (b + c) \cdot a = b \cdot a + c \cdot a \qquad \forall a,b,c \in R.
		\]
	\end{itemize}
	If multiplication is also commutative, we call $(R, +, \cdot)$ a \textbf{commutative ring}.
	Note that, unlike addition, we do not require that every element has an inverse for multiplication.
	A \textbf{field} is a special kind of commutative ring, i.e. every non-zero element has an inverse for multiplication.
	In other words, if $(R, +, \cdot)$ is a commutative ring, then $(R, +, \cdot)$ is a field if $R \setminus \{0\}$ forms a commutative group under multiplication. Traditionally, we use the letter $F$ to denote a field. We also write $F^{\ast} = F \setminus \{0\}$ for the set of all invertible elements of $F$.
\end{definition}

\begin{lemma}{Basic Properties of Fields}{properties_fields}
	Let $(F, +, \cdot)$ be a field and let $a,b \in F$. Then:
	\begin{enumerate}
		\item $0 \cdot a = a \cdot 0 = 0$.
		\item $a \cdot (-b) = -(a \cdot b) = (-a) \cdot b$. In particular $(-1) \cdot a = -a$.
		\item $(-a)\cdot (-b) = a \cdot b$. In particular, $(-a)^{-1} = -(a^{-1})$ whenever $a \neq 0$.
	\end{enumerate}
\end{lemma}

\begin{proof}
	1. Since 0 is the neutral element for the addition, we have $0 + 0 = 0$. Hence, using distributivity, we get
	\[
		0 \cdot a = (0 + 0) \cdot a = (0 \cdot a) + (0 \cdot a).	
	\]
	Adding $-0 \cdot a$ (i.e., the inverse of $0 \cdot a$), we deduce that $0 \cdot a = 0$. The case of $a \cdot 0$ is analogous.
	
	2. By the distributive law,
	\[
		a \cdot b + a \dot (-b) = a \cdot (b + (-b)) = a \cdot 0 = 0.
	\]
	So $a \cdot (-b)$ is the additive inverse of $a \cdot b$, i.e., $-(a \cdot b) = a \cdot (-b)$. Taking $b = 1$ gives $-a = (-1)\cdot a$.
	The validity of $(-a)\cdot b = -(a \cdot b)$ follows by exchanging $a$ and $b$ in the argument above.
	
	3. By 2. we know that $-(a \cdot b) = a \cdot (-b)$. Hence, recalling Equation \ref{eq:inverse_element},
	\[
		a \cdot b = -(a \cdot (-b)).
	\]
	On the other hand, applying 2. with $(-b)$ instead of $b$, we also have
	\[
		-(a \cdot (-b)) = (-a) \cdot (-b).
	\]
	Combining the two identities above, we conclude that $(-a)\cdot (-b) = a \cdot b$. Finally, taking $b = a^{-1}$ yields $(-a) \cdot (-(a^{-1})) = a \cdot a^{-1} = 1$, which gives the second assertion. \qedhere
\end{proof}

\subsection{Order Relation}

\begin{definition}{Cartesian Product}{cart_prod}
	Let $X$ and $Y$ be two sets. The \textbf{cartesian product} $X \times Y$ is the set of ordered pairs of elements of $X$ and $Y$, i.e.,
	\[
		X \times Y := \{(x, y) \;|\; x \in X, y \in Y\}.
	\]
\end{definition}

\begin{definition}{Subsets}{subsets}
	Let $P$ and $Q$ be sets. Then
	\begin{itemize}
		\item $P$ is a \textbf{subset} of $Q$, written $P \subset Q$ (or $P \subseteq Q$), if every element of $P$ also belongs to $Q$.
		\item $P$ is a \textbf{proper subset} of $Q$, written $P \subsetneq Q$, if $P$ is a subset of $Q$ but $P \neq Q$.
		\item We write $P \nsubseteq Q$ if $P$ is not a subset of $Q$.
	\end{itemize}
\end{definition}

\begin{definition}{Relations}{relations}
	Let $X$ be a set. A \textbf{relation} on $X$ is a subset $\mathcal{R} \subseteq X \times X$, that is, a collection of ordered pairs of elements of $X$. If $(x, y) \in \mathcal{R}$ we write $x\mathcal{R}y$. Common symbols for relations include $<, \leq , \sim, \equiv, \cong$.
	If $\sim$ is a relation on $X$, we write $x \nsim y$ if $x \sim y$ does not hold. A realtion $\sim$ may have the following properties:
	\begin{enumerate}
		\item \textbf{Reflexive:} if $x \sim x \qquad \forall x \in X$.
		\item \textbf{Transitive:} if $x \sim y$ and $y \sim z$, then $x \sim z$.
		\item \textbf{Symmetric:} if $x \sim y$, then $y \sim x$.
		\item \textbf{Antisymmetric:} if $x \sim y$ and $y \sim x$, then $x = y$.
	\end{enumerate}
	A relation is an \textbf{equivalence relation} if it is reflexive, transitive and symmetric. It is an \textbf{order relation} if it is reflexive, transitive and antisymmetric.
\end{definition}

\subsection{Ordered Fields}

\begin{definition}{Ordered Field}{ordered_field}
	Let $F$ be a field, and let $\leq$ be an order relation on $F$. We call $(F, \leq)$, or simply $F$, an \textbf{ordered field} if the following hold:
	\begin{enumerate}
		\item \textbf{Linearity of order:} for all $x, y \in F$, at least one of $x \leq y$ or $y \leq x$ holds.
		\item \textbf{Compatibility with addition:} for all $x,y,z \in F$,
		\[
			x \leq y \Rightarrow x + z \leq y + z.
		\]
		\item \textbf{Compatibility with multiplication:} for all $x, y \in F$,
		\[
			0 \leq x \;\wedge \; 0 \leq y \Rightarrow 0 \leq x \cdot y.
		\]
	\end{enumerate}
\end{definition}

\begin{lemma}{Ordered Field: Basic Consequences}{consequences_ordered_field}
	Let $(F, \leq)$ be and ordered field, and let $x, y, z, w \in F$. Then:
	\begin{enumerate}
		\item[(a)] (Trichotomy) Either $x < y$, or $x = y$, or $x > y$.
		\item[(b)] If $x < y$ and $y \leq z$, then $x < z$. (Analogously, $x \leq y and y < z imply x < z$.)
		\item[(c)] (Addition of inequalities) If $x \leq y$ and $z \leq w$, then $x + z \leq y + w$. (Analogously, $x < z$ and $z \leq w$ imply $x + z < y + w$.)
		\item[(d)] $x \leq y$ if and only if $0 \leq y - x$.
		\item[(e)] $x \leq 0$ if and only if $0 \leq -x$.
		\item[(f)] $x^2 \geq 0$, and $x^2 > 0$ if $x \neq 0$.
		\item[(g)] 0 < 1.
		\item[(h)] If $0 \leq x$ and $y \leq z$, then $x y \leq x z$.
		\item[(i)] If $x \leq 0$ and $y \leq z$, then $x y \geq x z$.
		\item[(j)] If $0 < x \leq y$, then $0 < y^{-1} \leq x^{-1}$.
		\item[(k)] If $0 \leq x \leq y$ and $0 \leq z \leq w$, then $0 \leq x z \leq y w$.
		\item[(l)] If $x + y \leq x + z$, then $y \leq z$.
		\item[(m)] If $x y \leq x z$ and $x > 0$, then $y \leq z$.
	\end{enumerate}
\end{lemma}

\begin{lemma}{Integers and Rationals Inside an Ordered Field}{int_rat_ordered_field}
	Let $(F, \leq)$ be and ordered field, and denote by 0 and 1 the neutral elements for addition and multiplication, respectively. Then:
	\begin{enumerate}
		\item[(i)] The elements $\hdots, -2, -1, 0, 1, 2, \hdots $ defined by
		\[
			2 = 1 + 1, \quad 3 = 2 + 1, \hdots , \quad -n = (-1)\cdot n
		\]
		are all distinct and satisfy
		\[
			\hdots < -2 < -1 < 0 < 1 < 2 < 3 < \hdots.
		\]
		We denote this set of elements by $\mathbb{Z}$, and we call them 'integers'
		\item[(ii)] Every fraction $pq^{-1}$ with $p, q \in \mathbb{Z}, \, q \neq 0$, lies in $F$ and the set of all such elements is denoted by $\mathbb{Q}$. Also,
		\[
			\mathbb{Z} \subsetneq \mathbb{Q} \subseteq F.
		\]
	\end{enumerate}
\end{lemma}

\begin{proof}
	(i) By Lemma \ref{lem*consequences_ordered_field}(g), we have that 0 < 1. Then Lemma \ref{lem*consequences_ordered_field}(c) yields $0 < 1 < 2 < 3 < \hdots$, and taking negatives gives $\hdots < -2 < -1 < 0$. Hence all these elements are distinct.
	
	(ii) For $q \neq 0$, q is invertible in $F$; define $\frac{p}{q} = pq^{-1}$. The set of such fractions is a field contained in $F$, which we denote by $\mathbb{Q}$.
	
	To show that $\mathbb{Q}$ strictly contains $\mathbb{Z}$, consider $\frac{1}{2}$ (the inverse of 2). Since 2 > 1, it follows from Lemma \ref{lem*consequences_ordered_field}(j) that $0 < \frac{1}{2} < 1$, so $\frac{1}{2} \notin \mathbb{Z}$.\qedhere
\end{proof}

\begin{definition}{Absolute Value and Sign}{abs_sgn}
	Let $(F, \leq)$ be and ordered field.
	\begin{itemize}
		\item[$\bullet$] The \textbf{absolute value} (or \textbf{modulus}) is the function $|\cdot|: F \to F$ defined by
		\[
			|x| = \begin{cases}
				x, \quad &x \geq 0,\\
				-x, \quad &x < 0.
			\end{cases}
		\]
		\item[$\bullet$] The \textbf{sign} is the function $\operatorname{sgn}:F \to \{-1, 0, 1\}$ defined by
		\[
			\operatorname{sgn}(x) = \begin{cases}
				-1, \quad &x < 0,\\
				0, \quad &x = 0,\\
				1, \quad &x > 0.
			\end{cases}
		\]
	\end{itemize}
\end{definition}

\begin{lemma}{Absolute Value and Sign: Basic Properties}{prop_abs_sign}
	Let $(F, \leq)$ be an ordered field and let $x,y \in F$. Then:
	\begin{enumerate}
		\item[(a)] $x = \operatorname{sgn}(x) |x|, \quad |-x| = |x|, \quad \operatorname{sgn}(-x) = - \operatorname{sgn}(x)$.
		\item[(b)] $|x| \geq 0$, and $|x| = 0$ if and only if $x = 0$ (by Trichotomy Lemma \ref{lem*consequences_ordred_field}).
		\item[(c)] (Multiplicativity) $\operatorname{sgn}(xy) = \operatorname{sgn}(x) \operatorname{sgn}(y)$ and $|x y| = |x||y|$.
		\item[(d)] If $x \neq 0$, then $|x^{-1}| = |x|^{-1}$.
		\item[(e)] $|x| \leq y$ iff $-y \leq x \leq y$.
		\item[(f)] $|x| < y$ iff $-y < x < y$.
		\item[(g)] (Triangle inequality) $|x + y| \leq |x| + |y|$.
		\item[(h)] (Inverse triangle inequality) $||x| - |y|| \leq |x - y|$.
	\end{enumerate}
\end{lemma}

\begin{proof}
	(g) Thanks to (e) we have $-|x| \leq x \leq |x|$ and $-|y| \leq y \leq |y|$. Adding these two inequalities we get
	\[
		- (|x| + |y|) \leq x + y \leq |x| + |y|.
	\]
	Applying (e) again yields the result.
	
	(h) From (g) we have $|x| \leq |x - y| + |y|$, therefore
	\[
		|x| - |y| \leq |x - y|.
	\]
	Exchanging the roles of $x$ and $y$, we also have $|y| - |x| leq |y - x| = |x - y|$. Combining these two inequalities yields
	\[
		-|x - y| \leq |x| - |y| \leq |x -y|,
	\]
	and the result follows by applying (e) again. \qedhere
\end{proof}

\subsection{Completeness Axiom}

\begin{definition}{Completeness Axiom}{compl_axiom}
	Let $(K, \leq)$ be an ordered field. We say that $(K, \leq)$ is \textbf{complete} (or a \textbf{completely ordered field}) if the following statement holds:
	\begin{itemize}
		\item[] Let $X, Y$ be non-empty subsets of $K$ such that $x \leq y$ for all $x \in X$ and $y \in Y$. Then there exists $c \in K$ lying between $X $ and $Y$, in the sense that $x \leq c \leq y$ for all $x \in X$ and $y \in Y$.
	\end{itemize}
	The statement above is called the \textbf{completeness axiom}.
\end{definition}

\begin{definition}{Real Numbers}{real_numbers}
	We call \textbf{the field of real numbers}, any completely ordered field and denote it by $\mathbb{R}$.
\end{definition}

\subsection{Intervals}

\begin{definition}{Intervals}{interval}
	Let $a, b \in \mathbb{R}$. We define:
	\begin{itemize}
		\item[$\bullet$] The \textbf{closed interval}
		\[
			[a, b] := \{x \in R\;|\; a \leq x \leq b\};
		\]
		\item[$\bullet$] The \textbf{open interval}
		\[
			(a, b) := \{x \in R \;|\; a < x < b\};
		\]
		\item[$\bullet$] The \textbf{half-open intervals}
		\[
			[a, b) := \{x \in R \;|\; a \leq x < b\} \quad \text{and} \quad (a, b] := \{x \in R \;|\; a < x \leq b\};
		\]
		\item[$\bullet$] The \textbf{unbounded closed intervals}
		\[
			[a, \infty) := \{x \in R \;|\; a \leq x\} \quad \text{and} \quad (-\infty, b] := \{x \in R \;|\; x \leq b\};
		\]
		\item[$\bullet$] The \textbf{unbounded open intervals}
		\[
			(a, \infty) := \{x \in R \;|\; a < x\} \quad \text{and} \quad (-\infty, b) := \{x \in R \;|\; x < b\};
		\]
	\end{itemize}
\end{definition}

\begin{definition}{Set Operations}{set_op}
	Let $P, Q$ be sets. The \textbf{intersection} $P \cap Q$, the \textbf{union} $P \cup Q$, the \textbf{relative complement} $P \setminus Q$ and the \textbf{symmetric difference} $P \bigtriangleup Q$ are defined by
	\begin{align*}
		P \cap Q &= \{x \;|\; x \in P \text{and} x \in Q\},\\
		P \cup Q &= \{x \;|\; x \in P \text{or} x \in Q\},\\
		P \setminus Q &= \{x \;|\; x \in P \text{and} x \notin Q\},\\
		P \bigtriangleup Q &= (P \setminus Q) \cup (Q  \setminus P) = (P \cup Q) \setminus (P \cap Q).
	\end{align*}
\end{definition}

\begin{definition}{Union and Intersection of several Sets}{union_int_sets}
	Let $\mathcal{A}$ be a family of sets (i.e., a set whose elements are sets). We define the \textbf{union} and \textbf{intersection} of the sets in $\mathcal{A}$ as
	\[
		\bigcup_{A \in \mathcal{A}} A = \{x \;|\; \exists A \in A : x \in A\}, \quad \bigcap_{A \in \mathcal{A}} A = \{x \;|\; \forall A \in \mathcal{A} : x \in A\}.
	\]
	If $\mathcal{A} = \{A_1, A_2, \hdots \}$, we also write
	\[
		\bigcup_{i = 1}^{\infty} A_i = \{x \;|\; \exists i \geq 1 : x \in A_i\}, \quad \bigcap_{i = 1}^{\infty} A_i = \{x \;|\; \forall i \geq 1 : x \in A_i\}.
	\]
\end{definition}

\begin{definition}{Neighborhoods}{neigh}
	Let $x \in \mathbb{R}$. A \textbf{neighborhood} of $x$ is a set containing an interval $I$ such that $x \in I$. Given $\delta > 0$, the open interval $(x - \delta, x + \delta)$ is called the $\delta$\textbf{-neighborhood} of $x$.
\end{definition}

\begin{definition}{Open and Closed Sets}{open_closed_sets}
	A subset $U \subseteq \mathbb{R}$ is called \textbf{open} in $\mathbb{R}$ if for every $x \in U$ there exists open interval $I$ such that $x \in I$ and $I \subseteq U$.
	A subset $F \subseteq \mathbb{R}$ is called \textbf{closed} in $\mathbb{R}$ if its complement $\mathbb{R} \setminus F$ is open.
\end{definition}

\begin{remark}
	The sets $\emptyset$ and $\mathbb{R}$ are both open in $\mathbb{R}$. Hence, they are also closed since $\emptyset^{c} = \mathbb{R}$ and $\mathbb{R}^{c} = \emptyset$. We note that $\mathbb{Q} \subseteq \mathbb{R}$ and $[a, b) \subseteq \mathbb{R}$ are neither open nor closed.
\end{remark}

\begin{remark}
	Let $\mathcal{U}$ be a family of open sets, and $\mathcal{F}$ be a family of closed subsets of $\mathbb{R}$. Then the union and intersection
	\[
		\bigcup_{U \in \mathcal{U}} U, \quad \bigcap_{F \in \mathcal{F}} F
	\]
	Are open and closed, respectively.
\end{remark}

\subsection{Complex Numbers}
Starting from the field of real numbers $\mathbb{R}$, we define the set of \textbf{complex numbers} as
\[
	\mathbb{C} = \mathbb{R}^2 = \{(x, y) \;|\; x, y \in \mathbb{R}\}.
\]
We denote the elements $z = (x, y) \in \mathbb{C}$ in the form $z = x + iy$, where $i$ is the \textbf{imaginary unit}.
Here $x \in \mathbb{R}$ is the \textbf{real part} of $z$, written as $x = \operatorname{Re}(z)$, and $y \in \mathbb{R}$ is the \textbf{imaginary part}, written as $y = \operatorname{Im}(z)$. Elements with $\operatorname{Im}(z) = 0$ are called \textbf{real}, while those with $\operatorname{Re}(z) = 0$ are \textbf{purely imaginary}. Via the injective map $\mathbb{R} \owns x \mapsto x + i\cdot 0 \in \mathbb{C}$, we identify $\mathbb{R}$ with the subset of real numbers inside $\mathbb{C}$.

As you may expect from previous knowledge, we want to satisfy $i^2 = -1$. To achieve this, we define addition and multiplication on $\mathbb{C}$ so that it becomes a field. Additionally, we want these operations to coincide with the usual addition and multiplication when considering real numbers.

Since $i^2 = -1$, using commutativity and distributivity we get
\[
	(x_1 + iy_1) (x_2 + iy_2) = x_1x_2 + i x_1y_2 + iy_2x_1 + i^2y_1y_2 = (x_1x_2 - y_1y_2) + i (x_1y_2 + y_1x_2).
\]
This motivates the following definition

\begin{definition}{Addition and Multiplication on $\mathbb{C}$}{add_mult_cmplx}
	On $\mathbb{C} = \mathbb{R} \times \mathbb{R}$ we define \textbf{addition} and \textbf{multiplication} as follows:
	\begin{align*}
		(x_1, y_1) + (x_2, y_2) &= (x_1 + x_2,\, y_1 + y_2),\\
		(x_1, y_1) \cdot (x_2, y_2) &= (x_1x_2 - y_1y_2,\, x_1y_2 + x_2y_1).
	\end{align*}
\end{definition}

\begin{proposition}{$\mathbb{C}$ is a Field}{cmplx_field}
	With the operation of Definition \ref{def*add_mult_cmplx}, together with the zero element $(0, 0)$ and the unit element $(1, 0)$, the set $\mathbb{C}$ is a field.
\end{proposition}

\begin{definition}{Complex Conjugation}{cmplx_conj}
	For $z = x + iy \in \mathbb{C}$ we define its \textbf{conjugate} as $\bar{z} = x - iy$. The mapping $\mathbb{C} \owns z \mapsto \bar{z} \in \mathbb{C}$ is called \textbf{complex conjugation}.
\end{definition}

\begin{lemma}{Properties of Complex Conjugation}{prop_cmplx_conj}
	For all $z, w \in \mathbb{C}$:
	\begin{enumerate}
		\item[(i)] $z \bar{z} = x^2 + y^2 \in \mathbb{R}_{\geq 0}$. In particular, $z\bar{z} = 0$ if and only if $z = 0$.
		\item[(ii)] $\overline{z + w} = \bar{z} + \bar{w}$.
		\item[(iii)] $\overline{z w} = \bar{z} \bar{w}$.
	\end{enumerate}
\end{lemma}

\begin{proof}
	Property (i) follows from the fact that, for $z = x + iy$, $(x + iy)(x - iy) = x^2 + y^2$. Also, $x^2 + y^2 = 0$ if and only if $x + iy = 0$.
	Properties (ii) and (iii) follow from a direct computation, writing $z = x_+ + iy_1$ and $w = x_2 + iy_2$, which yields
	\begin{align*}
		\overline{z + w} = \overline{(x_1 + x_2) + i(y_1 + y_2)} = (x_1 + x_2) - i (y_1 + y_2) &= (x_1 - iy_2) + (x_2 - i y_2) = \bar{z} + \bar{w},\\
		\overline{z \cdot w} = \overline{(x_1x_2 - y_1y_2) + i (x_1y_2 + x_2y_1)} = (x_1 x_2 - y_1y_2) &- i (x_1y_2 + y_1x_2)\\
		 &= (x_1 - i y_1) \cdot (x_2 - i y_2) = \bar{z} \cdot \bar{w}. \qedhere
	\end{align*}
\end{proof}

\begin{definition}{Absolute Value}{abs_val}
	The \textbf{absolute value} (or \textbf{norm}) on $\mathbb{C}$ is the map $|\cdot|:\mathbb{C} \to \mathbb{R}$ given by
	\[
		|z| = \sqrt{z \bar{z}} = \sqrt{x^2 + y^2}, \qquad z = x + iy \in \mathbb{C}.
	\]
\end{definition}

\begin{lemma}{Cauchy-Schwart Inequality}{cauchy_schwarz_ineq}
	If $z = x_1 + iy_1$, and $w = x_2 + iy_2$, then
	\begin{equation}
		\label{eq:cauchy_schwarz_ineq}
		x_1x_2 + y_1y_2 \leq |z||w|.
	\end{equation}
\end{lemma}

\begin{proof}
	We observe that
	\begin{align*}
		|z|^2|w|^2 - (x_1x_2 + y_1y_2)^2 &= (x_1^2 + y_1^2)(x_2^2 + y_2^2) - (x_1x_2 + y_1y_2)^2\\
		&= x_1^2x_2^2 +y_1^2y_2^2 + y_1^2x_2^2 + x_+^2y_2^2 - (x_1^2x_2^2 + y_1^2y_2^2 + 2x_1x_2y_1y_2)\\
		&= y_1^2x_2^2 + x_+^2y_2^2 - 2x_1x_2y_1y_2\\
		&= (y_1 x_2 - x_1y_2)^2 \geq 0.
	\end{align*}
	This proves that $(x_1x_2 + y_1y_2)^2 \leq |z|^2|w|^2$, so it follows that
	\[
		|x_1x_2 +y_1y_2| \leq |z||w|.
	\]
	Since $x \leq |x|$ for all $x \in \mathbb{R}$, we obtain Equation \ref{eq:cauchy_schwarz_ineq}.\qedhere
\end{proof}

\begin{proposition}{Trianlge Inequality}{triangle_ineq}
	For all $z, w \in \mathbb{C}$, one has
	\[
		|z + w| \leq |z| + |w|.
	\]
\end{proposition}

\begin{proof}
	For $z = x_1 + iy_1$ and $w = x_2 + iy_2$, using Lemma \ref{lem*cauchy_schwarz_ineq}, we have
	\begin{align*}
		|z + w|^2 &= (x_1 + x_2)^2 + (y_1 + y_2)^2\\
		&= |z|^2 + |w|^2 + 2(x_1x_2 + y_1y_2)\\
		&\leq |z|^2 + |w|^2 + 2 |z||w| = (|z| + |w|)^2.
	\end{align*}
	Taking square roots proves the result. \qedhere
\end{proof}

\begin{definition}{Cicular Disks}{circ_disk}
	For $z \in \mathbb{C}$ and $r > 0$, we define the \textbf{open disk} with radius $r >0$ around $z$ as
	\[
		B(z, r) := \{w \in \mathbb{C}\;|\; |z - w| < r\},
	\]
	and the \textbf{closed disk} with radius $r >0$ around $z$ as
	\[
		\overline{B(z, r)} := \{w \in \mathbb{C}\;|\; |z - w| \leq r\}.
	\]
\end{definition}

In other words, the open disk $B(z, r)$ is the set of points at distance strictly less than $r$ form $z$. We note that this definition is compatible with the one of neighborhoods in $\mathbb{R}$: if $x \in \mathbb{R}$ and $r > 0$, then 
\[
	B(x, r) \cap \mathbb{R} = (x - r, x + r).
\]

\begin{definition}{Open and Closed Sets}{open_closed_sets_cmplx}
	A set $U \subseteq \mathbb{C}$ is \textbf{open} if for every $z \in U$ there exists $r > 0$ such that $B(z, r) \subseteq U$. A set $C \subseteq \mathbb{C}$ is \textbf{closed} if its complement $\mathbb{C} \setminus C$ is open.
\end{definition}

\subsection{Consequences of Completeness}


	%
% (c) 2025 Autor, ETH Zürich
%
% !TEX root = main.tex
% !TEX encoding = UTF-8
%

\section{Sequences of Real Numbers}

\subsection{Convergence of Sequences}

\begin{definition}{Sequences}{sequences}
	A \textbf{sequence} is a function $a:\mathbb{N} \to \mathbb{R}$. The image $a(n)$ of $n \in \mathbb{N}$ is also written as $a_n$ and is called the $n$-th element of a. Instead of $a:\mathbb{N} \to \mathbb{R}$ one often writes $(a_n)_{n \in \mathbb{N}}, (a_n)_{n =0}^{\infty}, (a_n)_{n \geq 0}$.
\end{definition}

\begin{definition}{(Eventually) Constant Sequences}{const sequences}
	A sequence $(x_n)_{n=0}^{\infty}$ is \textbf{constant} if $x_n = x_m \forall n,m \in \mathbb{N}$. It is \textbf{eventually constant} if there exists $N \in \mathbb{N}$ such that $x_n = x_m \forall n,m \geq N$.
\end{definition}

\begin{definition}{Convergence of Sequences}{conv sequences}
	Let $(x_n)_{n=0}^{\infty}$ be a sequence in $\mathbb{R}$. We say that $(x_n)_{n=0}^{\infty}$ \textbf{converges} (or is \textbf{convergent}) if $\exists A \in \mathbb{R}$ such that 
	\[
		\forall \varepsilon > 0\; \exists N \in \mathbb{N} : |x_n - A| < \varepsilon\quad \forall n \geq N.
	\]
	In this case we write
	\begin{equation}
		\lim_{n \to \infty} x_n = A
	\end{equation}
	and call $A$ the \textbf{limit} of $(x_n)_{n=0}^{\infty}$.
\end{definition}

\begin{lemma}{Uniqueness of the Limit}{unique limit}
	A convergent sequence $(x_n)_{n=0}^{\infty}$ has exactly one limit.
\end{lemma}
\begin{proof}
	Let $A,B \in \mathbb{R}$ be limits of $(x_n)_{n=0}^{\infty}$. Fix $\varepsilon > 0$. Then there exists $N_A, N_B \in \mathbb{N}$ such that $|x_n -A| < \varepsilon$ for all $n \geq N_A$ and $|x_n - B| < \varepsilon$ for all $n \geq N_B$. We define $N := \operatorname{max}\{N_A, N_B\}$. Then it holds that
	\[
		|A - B| \leq |A - x_N| + |x_N - B| < \varepsilon + \varepsilon = 2\varepsilon.
	\]
	Since $\varepsilon > 0$ is arbitrary, it follows that $A = B$. \qedhere
\end{proof}

\subsection{Convergent Subsequences and Acculmulation Points}
\begin{definition}{Subsequences}{subseq}
	Let $(x_n)_{n=0}^{\infty}$ be a sequence. A \textbf{subsequence} is of the form $(x_{n_k})_{k=0}^{\infty}$, where $(n_k)_{k=0}^{\infty}$ is a strictly increasing sequence of non-negative integers, i.e., $n_{k+1} > n_k \; \forall k \in \mathbb{N}$.
\end{definition}

\begin{remark}
	\label{rmk:subseq}
	Since $n_{k+1} > n_k$ for all $k \in \mathbb{N}$ it follows by induction that $n_k \geq k$ for all $k \in \mathbb{N}$.
\end{remark}
\begin{proof}
	For $k=0$ we have that $n_0 \geq 0$, because $(n_k)_{k=0}^{\infty}$ is a sequence of non-negative integers. So the condition is fulfilled.
	For the inductive step we want to show that the condition holds for $k+1$ under the assumption that the condition is true for $k$. Because $(n_k)_{k=0}^{\infty}$ is also a strictly increasing sequence, we have that $n_{k+1} > n_k \geq k$. Additionally since $n_k \in \mathbb{N}$, we have that $n_{k+1} \geq n_k + 1$. 
	So it follows that $n_{k+1} \geq n_k + 1 \geq k + 1$, which proofs the condition for $k+1$.\qedhere
\end{proof}

\begin{lemma}{Subsequences of Convergent Sequences are Convergent}{subseq conv}
	Let $(x_n)_{n=0}^{\infty}$ be a sequence converging to $A \in \mathbb{R}$. Then every subsequence $(x_{n_k})_{k=0}^{\infty}$ also converges to $A$.
\end{lemma}
\begin{proof}
	Let $(x_n)_{n=0}^{\infty}$ be a sequence converging to $A \in \mathbb{R}$.
	Fix $\varepsilon > 0$.
	Since $(x_n)_{n=0}^{\infty}$ converges to $A$, there exists $N \in \mathbb{N}$ such that $|x_n - A| < \varepsilon \; \forall n \geq N$. As by Remark \ref{rmk:subseq} we know that $n_k \geq k$ for all $k \in \mathbb{N}$.
	Therefore for all $k \geq N$ it holds that $|x_{n_k} - A| < \varepsilon$. \qedhere
\end{proof}

\begin{definition}{Accumulation Points of Sequences}{acc pt seq}
	Let $(x_n)_{n=0}^{\infty}$ be a sequence in $\mathbb{R}$. A point $A \in \mathbb{R}$ is an \textbf{accumulation point} of $(x_n)_{n=0}^{\infty}$ if
	\[
		\forall \varepsilon > 0 \; \forall N \in \mathbb{N} \; \exists n \geq N : |x_n - A| < \varepsilon.
	\]
\end{definition}

\begin{proposition}{Subsequences and Accumulation Points}{subseq_acc_pt}
	Let $(x_n)_{n=0}^{\infty}$ be a sequence in $\mathbb{R}$. A point $A$ is and accumulation point of $(x_n)_{n=0}^{\infty}$ if and only if there exists a convergent subsequence of $(x_n)_{n=0}^{\infty}$ with limit $A$.
\end{proposition}

\begin{corollary}{Infinitely Many Terms Near an Accumulation Point}{inf terms acc pt}
	If $A \in \mathbb{R}$ is an accumulation point of $(x_n)_{n=0}^{\infty}$, then for every $\varepsilon > 0$ there are infinitely many n with $x_n \in (A - \varepsilon, A + \varepsilon)$.
\end{corollary}
\begin{proof}
	By Proposition \ref{prop*subseq_acc_pt}, there exists a subsequence $(x_{n_k})_{k=0}^{\infty}$ with $\lim_{k \to \infty} = A$. Hence for every $\varepsilon > 0$ there exists $K$ such that $x_{n_k} \in (A - \varepsilon, A + \varepsilon)$ for all $k \geq K$, providing infinitely many elements of the sequence inside the interval $(A - \varepsilon, A + \varepsilon)$. \qedhere
\end{proof}

\begin{corollary}{Accumulation Points of Convergent Sequences}
	A convergent sequence has exactly one accumulation point, namely its limit.
\end{corollary}

\subsection{Addition, Multiplication and Inequalities}
\begin{proposition}{Limits and Operations}{lim_op}
	Let $(x_n)_{n=0}^{\infty}$ and $(y_n)_{n=0}^\infty$ be sequences converging to $A, B \in \mathbb{R}$ respectively. Then:
	\begin{enumerate}
		\item The sequence $(x_n + y_n)_{n=0}^{\infty}$ converges to $A + B$.
		\item The sequence $(x_ny_n)_{n=0}^{\infty}$ converges to $AB$.
		\item Given $\alpha \in \mathbb{R}$, the sequence $(\alpha x_n)_{n=0}^{\infty}$ converges to $\alpha A$.
		\item Suppose $x_n \neq 0$ for all $n \in \mathbb{N}$ and $A \neq 0$. Then the sequence $(x_n^{-1})_{n=0}^{\infty}$ converges to $A^{-1}$. 
	\end{enumerate}
\end{proposition}

\begin{proposition}{Limits and Inequalities}{lim_ineq}
	Let $(x_n)_{n=0}^{\infty}$ and $(y_n)_{n=0}^\infty$ be sequences converging to $A, B \in \mathbb{R}$ respectively.
	\begin{enumerate}
		\item If $A < B$, then there exists $N \in \mathbb{N}$ such that $x_n < y_n$ for all $n \geq \mathbb{N}$.
		\item If there exists $N \in \mathbb{N}$ such that $x_n \leq y_n$ for all $n \geq N$, then $A \leq B$.
	\end{enumerate}
\end{proposition}

\begin{remark}
	In Propostion \ref{prop*lim_ineq} even if we assume that $x_n < y_n$ for all $n \in \mathbb{N}$, we cannot conclude that $A < B$. for example take
	\[
		x_n = \frac{1}{n}, \quad y_n = \frac{1}{n}.
	\]
	Then we have that $x_n < y_n$ for all $n \in \mathbb{N}$ but $A = B = 0$.
\end{remark}

\begin{lemma}{Sandwich Lemma}{sandwich_seq}
	Let $(x_n)_{n=0}^{\infty}$, $(y_n)_{n=0}^{\infty}$, $(z_n)_{n=0}^{\infty}$ be sequences such that for some $N \in \mathbb{N}$, we have that
	\[
		x_n \leq y_n \leq z_n \quad \forall n \geq N.
	\]
	Suppose that both $(x_n)_{n=0}^{\infty}$ and $(z_n)_{n=0}^{\infty}$ converge to the same limit. Then $(y_n)_{n=0}^{\infty}$ also converges, and we have that
	\[
		\lim_{n \to \infty} x_n = \lim_{n \to \infty} y_n = \lim_{n \to \infty} z_n.
	\]
\end{lemma}

\begin{proof}
	Let $(x_n)_{n=0}^{\infty}$, $(y_n)_{n=0}^{\infty}$, $(z_n)_{n=0}^{\infty}$ be sequences such that for some $N_0 \in \mathbb{N}$, we have that
	\[
		x_n \leq y_n \leq z_n \quad \forall n \geq N_0.
	\]
	Additionally suppose that $(x_n)_{n=0}^{\infty}$ and $(z_n)_{n=0}^{\infty}$ converge to $A \in \mathbb{R}$. Fix $\varepsilon > 0$. Since $(x_n)_{n=0}^{\infty}$, $(z_n)_{n=0}^{\infty}$ converge to $A$ there exists $N_x, N_z \in \mathbb{N}$ such that
	\begin{align*}
		A - \varepsilon &< x_n < A + \varepsilon \quad \forall n \geq N_x\\
		A - \varepsilon &< z_n < A + \varepsilon \quad \forall n_z \geq N_z.
	\end{align*}
	So we choose $N := \operatorname{max}\{N_0, N_x, N_z\}$. Then we have that
	\[
		A - \varepsilon < x_n \leq y_n \leq z_n < A + \varepsilon \quad \forall n \geq N,
	\] 
	which shows that $\lim_{n \to \infty} y_n = A$.\qedhere
\end{proof}

	%
% (c) 2025 Autor, ETH Zürich
%
% !TEX root = main.tex
% !TEX encoding = UTF-8
%

\section{Functions of one Real Variable}
In this chapter we study real-valued functions defined on subsets of $\mathbb{R}$, typically intervals. The central concept is \textit{continuity}.

\subsection{Real valued functions}

\subsubsection{Boundedness and Monotonitcity}
For a non-empty set $D \subseteq \mathbb{R}$, the set of \textbf{real-valued} functions on $D$ is
\[
	\mathcal{F}(D) = \{f\;|\;f:D \to \mathbb{R}\}.
\]
For $f_1, f_2 \in \mathcal{F}(D)$, $\alpha \in \mathbb{R}$, and $x \in D$ we define
\[
	(f_1 + f_2)(x) = f_1(x) + f_2(x), \qquad (\alpha f_1)(x) = \alpha f_1(x), \qquad (f_1f_2)(x) = f_1(x)f_2(x).
\]
Given $\alpha \in \mathbb{R}$, we write $f \equiv \alpha$ for the constant function $x \mapsto \alpha$ on $D$.

\begin{remark}
	With the operations above, $\mathcal{F}(D)$ is a commutative ring (the additive identity is $f \equiv 0$ and the multiplicative identity is $f \equiv 1$).
\end{remark}

A point $x \in D$ is a \textbf{zero} of $f \in \mathcal{F}(D)$ if $f(x) = 0$. The \textbf{zero set} of $f$ is $\{x \in D\;|\; f(x) = 0\}$.
We order $\mathcal{F}(D)$ pointwise: for $f_1, f_2 \in \mathcal{F}(D)$,
\begin{align*}
	f_1 \leq f_2 \quad &\Leftrightarrow \quad f_1(x) \leq f_2(x) \quad \forall x \in D,\\
	f_1 < f_2 \quad &\Leftrightarrow \quad f_1(x) < f_2(x) \quad \forall x \in D.
\end{align*}
We say that $f \in \mathcal{F}(D)$ is \textbf{non-negative} if $f \geq 0$, and \textbf{positive} if $f > 0$.

\begin{definition}{Bounded Functions}{bounded_func}
	Let $D \neq \emptyset$ and $f:D \to \mathbb{R}$. We say that $f$ is \textbf{bounded from above} if there exists $M > 0$ such that
	\[
		f(x) \leq M \qquad \forall x \in D.
	\]
	We say that $f$ is \textbf{bounded from below} if there exists $M > 0$ such that
	\[
		f(x) \geq -M \qquad \forall x \in D.
	\]
	We say that $f$ is \textbf{bounded} if it is both bounded from above and from below. Equivalently, $f$ if bounded if there exists $M > 0$ such that
	\[
		|f(x)| \leq M \qquad \forall x \in D.
	\]
\end{definition}

\begin{definition}{Monotone Functions}{monotone_func}
	Let $D \subseteq \mathbb{R}$ and $f:D \to \mathbb{R}$. The function $f$ is:
	\begin{enumerate}
		\item \textbf{increasing} if $x < y \quad \Rightarrow \quad f(x) \leq f(y) \quad \forall x,y \in D$;
		\item \textbf{strictly increasing} if $x < y \quad \Rightarrow \quad f(x) < f(y) \quad \forall x,y \in D$;
		\item \textbf{decreasing} if $x < y \quad \Rightarrow \quad f(x) \geq f(y) \quad \forall x,y \in D$;
		\item \textbf{strictly decreasing} if $x < y \quad \Rightarrow \quad f(x) > f(y) \quad \forall x,y \in D$.
	\end{enumerate}
	We call $f$ \textbf{monotone} if it is increasing or decreasing, and \textbf{strictly monotone} if it is strictly increasing or strictly decreasing.
\end{definition}

\subsubsection{Continuity}

\begin{definition}{Continuous Functions}{cont_func}
	Let $D \subseteq \mathbb{R}$ and $f:D \to \mathbb{R}$. We say that $f$ is \textbf{continuous at} $x_0 \in D$ if for all $\varepsilon > 0$ there exists $\delta > 0$ such that
	\[
		\forall x \in D, \quad |x - x_0| < \delta \quad \Rightarrow \quad |f(x) - f(x_0)| < \varepsilon.
	\]
	We say that $f$ is \textbf{continuous on} $D$ if it is continuous at every point of $D$.
\end{definition}

\begin{remark}
	It suffices to verify the implication above for small $\varepsilon$. Precisely, assume there exists $\varepsilon_0 > 0$ such that for every $\varepsilon \in (0, \varepsilon_0]$ there is a $\delta > 0$ such that
	\[
		\forall x \in D, \quad |x - x_0| < \delta \quad \Rightarrow \quad |f(x) - f(x_0)| < \varepsilon.
	\]
	Then $f$ is continuous at $x_0$.
	
	Indeed, for $\varepsilon_0 > \varepsilon$ we can choose the number $\delta > 0$ corresponding to $\varepsilon$ to get
	\[
		\forall x \in D, \quad |x - x_0| < \delta \quad  \Rightarrow \quad |f(x) - f(x_0)| < \varepsilon < \varepsilon_0.
	\]
	In other words, if $\delta$ works for $\varepsilon$, then it works for all $\varepsilon_0 > \varepsilon$.
\end{remark}

\begin{definition}{Restriction}{restriction}
	Let $D \subseteq \mathbb{R}$ and $f:D \to \mathbb{R}$. For any $D' \subseteq D$ the \textbf{restriction} of $f$ to $D'$ is the function $f|_{D'}:D' \to \mathbb{R}$ defined by
	\[
		f|_{D'}(x) = f(x) \qquad \forall x \in D'.
	\]
	We regard $f|_{D'}$ and $f$ as different functions unless $D' = D$.
\end{definition}

\begin{proposition}{Combination of Continuous Functions}{comb_cont_func}
	Let $D \subseteq \mathbb{R}$, and let $f_1, f_2:D \to \mathbb{R}$ be continuous at $x_0 \in D$. Then $f_1 + f_2$, $f_1f_2$, and $\alpha f_1$ (for any $\alpha \in \mathbb{R}$) are continuous at $x_0$.
\end{proposition}

\begin{proof}
	We first prove the result for the sum. Let $\varepsilon > 0$. Since $f_1$ and $f_2$ are continuous at $x_0$, there exists $\delta_1, \delta_2 > 0$ such that for all $x \in D$,
	\[
		|x - x_0| < \delta_1 \; \Rightarrow \; |f_1(x) - f_1(x_0)| < \frac{\varepsilon}{2}, \quad |x - x_0| < \delta_2 \; \Rightarrow \; |f_2(x) - f_2(x_0)| < \frac{\varepsilon}{2}.
	\]
	So, choosing $\delta = \min{\delta_1, \delta_2}$, for $|x - x_0| < \delta$ we get
	\[
		|(f_1 + f_2)(x) - (f_1 + f_2)(x_0)| \leq |f_1(x) - f_1(x_0)| + |f_2(x) - f_2(x_0)| < \varepsilon,
	\]
	which shows that $f_1 + f_2$ is continuous at $x_0$.
	
	For the product, note that
	\begin{align*}
		|f_1(x)f_2(x) - f_1(x_0)f_2(x_0)| &= |f_1(x)f_2(x) - f_1(x_0)f_2(x) + f_1(x_0)f_2(x) - f_1(x_0)f_2(x_0)|\\
		&\leq |f_1(x)f_2(x) - f_1(x_0)f_2(x)| + |f_1(x_0)f_2(x) - f_1(x_0)f_2(x_0)|\\
		&= |f_2(x)||f_1(x) - f_1(x_0)| + |f_1(x_0)||f_2(x) - f_2(x_0)|.
	\end{align*}
	Now, first choose $\delta_0 > 0$ such that $|x - x_0| < \delta_0$ implies $|f_2(x) - f_2(x_0)| < 1$, so that
	\[
		|x - x_0| < \delta_0 \quad \Rightarrow \quad |f_2(x)| < 1 + |f_2(x_0)|.
	\]
	Then choose $\delta_1, \delta_2 > 0$ such that
	\begin{align*}
		|x - x_0| < \delta_1 \quad \Rightarrow \quad |f_1(x) - f_1(x_0)| < \frac{\varepsilon}{2 (1 + |f_2(x_0)|)},\\
		|x - x_0| < \delta_2 \quad \Rightarrow \quad |f_2(x) - f_2(x_0)| < \frac{\varepsilon}{2 (1 + |f_1(x_0)|)}.
	\end{align*}
	So choosing $\delta = \min{\delta_0, \delta_1, \delta_2}$, for $|x - x_0| < \delta$ we get
	\begin{align*}
		|f_1(x)f_2(x) - f_1(x_0)f_2(x_0)| &< |f_2(X)|\, \frac{\varepsilon}{2 (1 + |f_2(x_0)|)} + |f_1(x_0)|\, \frac{\varepsilon}{2 (1 + |f_1(x_0)|)}\\
		&< (1 + |f_2(x_0)|)\, \frac{\varepsilon}{2 (1 + |f_2(x_0)|)} + |f_1(x_0)|\, \frac{\varepsilon}{2 (1 + |f_1(x_0)|)}\\
		&< \frac{\varepsilon}{2} + \frac{\varepsilon}{2} = \varepsilon,
	\end{align*}
	thus $f_1f_2$ is continuous at $x_0$.
	
	Finally, the statement about $\alpha f_1$ follows by choosing $f_2 \equiv \alpha$ (a constant function) and using the product case proved above: since $f_1$ and $f_2$ are continuous at $x_0$, their product $f_1f_2 = \alpha f_1$ is continuous at $x_0$. \qedhere
\end{proof}

\begin{definition}{Sum and Product Notation}{sum_prod_notation}
	Let $n \in \mathbb{N}$ and $a_0, a_1, \hdots , a_n \in \mathbb{R}$. We use the notation
	\[
		\sum_{j=0}^{n} a_j = a_0 + a_1 + \hdots + a_n, \qquad \prod_{j=0}^{n} a_0 \cdot a_1 \cdot \hdots \cdot a_n.
	\]
	Here $a_j$ is a \textbf{summand} in the sum and a \textbf{factor} in the product; $j$ is the \textbf{index} (or \textbf{running variable}).
	If $J$ is a finite set and numbers $(a_j)_{j\in J}$ are given, we write
	\[
		\sum_{j \in J} a_j, \qquad \prod_{j\in J} a_j.
	\]
	By convention, for the empty index set $\emptyset$,
	\[
		\sum_{j\in \emptyset} a_j = 0, \qquad \prod_{j\in \emptyset} a_j = 1.
	\]
\end{definition}

\begin{proposition}{Composition of Continuous Functions}{comp_cont_func}
	Let $D_1, D_2 \subseteq \mathbb{R}, x_0 \in D_1$ and $f:D_1 \to D_2$ be continuous at $x_0$. If $g:D_2 \to \mathbb{R}$ is continuous at $f(x_0)$, then $g \circ f:D_1 \to \mathbb{R}$ is continuous at $x_0$. In particular, the composition of continuous functions is continuous.
\end{proposition}

\begin{proof}
	Let $\varepsilon > 0$. By continuity of $g$ at $f(x_0)$, there exists $\eta > 0$ such that
	\[
		\forall y \in D_2, \quad |y - f(x_0)| < \eta \quad \Rightarrow \quad |g(y) - g(f(x_0))| < \varepsilon.
	\]
	By continuity of $f$ at $x_0$, there exists $\delta > 0$ such that
	\[
		\forall x \in D_1, \quad |x - x_0| < \delta \quad \Rightarrow \quad |f(x) - f(x_0)| < \eta.
	\]
	Combining the implications gives, for any $x \in D_1$,
	\[
		|x - x_0| < \delta \quad \Rightarrow \quad |f(x) - f(x_0)| < \eta \quad \Rightarrow \quad |g(f(x)) - g(f(x_0))| < \varepsilon. \qedhere
	\]
\end{proof}

\begin{remark}
	\label{rmk:modulus_cont}
	Applying Proposition \ref{prop*comp_cont_func} with $g(y) = |y|$, we see that if $f:D \to \mathbb{R}$ is continuous, then $x \mapsto |f(x)|$ is continuous.
\end{remark}

\subsubsection{Sequential Continuity}

\begin{definition}{Notation for Limits of Sequences}{notation_lim_seq}
	Let $(x_n)_{n=0}^{\infty} \subseteq \mathbb{R}$ and $\overline{x} \in \mathbb{R}$. We write
	\[
		x_n \to \bar{x} \qquad \text{or} \qquad x_n \underset{n \to \infty}{\longrightarrow} \bar{x}
	\] 
	to mean
	\[
		\lim_{n \to \infty} x_n = \bar{x}.
	\]
\end{definition}

\begin{theorem}{Continuity = Sequential Continuity}{cont_seq_cont}
	Let $D \subseteq \mathbb{R}$, $f:D \to \mathbb{R}$, and $\bar{x} \in D$. Then $f$ is continuous at $\bar{x}$ if and only if for every sequence $(x_n)_{n=0}^{\infty} \subseteq D$ with $x_n \to \bar{x}$ we have $f(x_n) \to f(\bar{x})$.  
\end{theorem}

\begin{proof}
	'$\Rightarrow$': First Assume that $f$ is continuous at $\bar{x}$. Then, given $\varepsilon > 0$, there exists $\delta > 0$ such that
	\[
		\forall x \in D, \quad |x - \bar{x}| < \delta \quad \Rightarrow \quad |f(x) - f(\bar{x})| < \varepsilon.
	\]
	Also, since $x_n \to \bar{x}$, there exists $N \in \mathbb{N}$ such that
	\[
		n \geq N \quad \Rightarrow \quad |x_n - \bar{x}| < \delta.
	\]
	Thus,
	\[
		n \geq N \quad \Rightarrow \quad |f(x_n) - f(\bar{x})| < \varepsilon,
	\]
	which implies that the sequence $(f(x_n))_{n=0}^{\infty}$ converges to $f(\bar{x})$.
	
	'$\Leftarrow$': To prove the converse, assume that $f$ is not continuous at $\bar{x}$. This means that there exists $\varepsilon > 0$ such that, for every $\delta > 0$, there is $x \in D$ with
	\[
		|x - \bar{x}| < \delta \quad \text{and} \quad |f(x) - f(\bar{x})| \geq \varepsilon.
	\]
	Now, for every $n \in \mathbb{N}$, we apply this property with $\delta = 2^{-n}$ to find a point $x_n \in D$ such that
	\[
		|x_n - \bar{x}| < 2^{-n} \quad \text{and} \quad |f(x_n) - f(\bar{x})| \geq \varepsilon
	\]
	Then the sequence constructed in this way satisfies $x_n \to \bar{x}$ but $f(x_n) \not\to f(\bar{x})$. \qedhere
\end{proof}

\begin{remark}
	\label{rmk:neg_seq_cont}
	The proof above shows that if $f:D \to \mathbb{R}$ is not continuous at $\bar{x}$, then there exists $\varepsilon > 0$ and a sequence $(x_n)_{n=0}^{\infty} \subseteq D$ with $x_n \to \bar{x}$ such that $|f(x_n) - f(\bar{x})| \geq \varepsilon$ for all $n \in \mathbb{N}$. This is useful to show that a function $f$ is not continuous at $\bar{x}$.   
\end{remark}

\subsection{Continuous Functions}

\subsubsection{Intermediate Value Theorem}
In this section we prove a fundamental theorem that formalizes the idea that the graph of a continuous function on an interval is a continuous curve, and thus cannot make any jumps.

\begin{theorem}{Intermediate Value Theorem}{int_val_theo}
	Let $f:[a, b] \to \mathbb{R}$ be a continuous function with $f(a) \leq f(b)$. Then, for every real number $c$ with $f(a) \leq c \leq f(b)$, there exists $\bar{x} \in [a, b]$ such that $f(\bar{x}) = c$.
\end{theorem}

\begin{proof}
	Fix $c \in [f(a), f(b)]$. Then define
	\[
		X = \{x \in [a, b] \;|\; f(x) \leq c\}.
	\]
	Since $a \in X$ and $X \subseteq [a, b]$, the set is non-empty and bounded from above. By Theorem \ref{theo*sup_exist}, its supremum
	\[
		\bar{x} = sup(X) \in [a, b]
	\]
	exists.
	We now use the continuity of $f$ at $x_0$ to show that $f(\bar{x}) = c$.
	
	Since $\bar{x}$ is the supremum of $X$, for each $n \geq 0$, we can find a point $x_n \in [\bar{x} - 2^{-n}, \bar{x}]$.
	Then $|x_n - \bar{x}| \leq 2^{-n}$, hence $x_n \longrightarrow \bar{x}$.
	Also, by the definition of $X$, we have $f(x_n) \leq c$. Thus, by Theorem \ref{theo*cont_seq_cont} (continuity of $f$ along sequences),
	\[
		\lim_{n \to \infty} f(x_n) = f(\bar{x}).
	\]
	And Proposition \ref{prop*lim_ineq} yields $\lim_{n \to \infty} f(x_n) \leq c$. Therefore, $f(\bar{x}) \leq c$.
	
	Suppose, by contradiction, $f(\bar{x}) < c$ and set $\varepsilon := c - f(\bar{x}) > 0$. By continuity at $\bar{x}$, there exists $\delta > 0$ such that for all $x \in [a, b]$
	\[
		|x - \bar{x}| < \delta \quad \Rightarrow \quad |f(x) - f(\bar{x})| < \varepsilon,
	\]
	hence $f(x) < f(\bar{x}) + \varepsilon = c$. 
	Therefore, by the definition of $X$,
	\[
		(\bar{x} - \delta, \bar{x} + \delta) \cap [a, b] \subseteq X.
	\]
	Moreover, since $f(\bar{x}) < c \leq f(b)$, we cannot have $\bar{x} = b$; hence $\bar{x} < b$.
	Because $\bar{x} < b$, the interval $(\bar{x} , \bar{x} + \delta) \cap [a, b] \subseteq X$ is non-empty.
	Pick 
	\[
		y \in (\bar{x}, \bar{x} + \delta) \cap [a, b] \subseteq X.
	\]
	Then $y \in X$ and $y > \bar{x}$, which contradicts the defining property of the supremum: $\bar{x}$ is an upper bound of $X$, and $X$ cannot contain elements larger than $\bar{x}$. This contradiction shows that $f(\bar{x}) \geq c$. Together with $f(\bar{x}) \leq c$ proved above, we conclude that $f(\bar{x}) = c$, as desired. \qedhere
\end{proof}

\begin{theorem}{Inverse Function Theorem}{inv_func_theo}
	Let $I$ be an interval and $f: I \to \mathbb{R}$ a continuous strictly monotone function. Then $f(I)$ is an interval, and the mapping $f:I \to f(I)$ has a continuous strictly monotone inverse function $f^{-1}:f(I) \to I$.
\end{theorem}

\begin{proof}
	We may assume that $I$ is non-empty and not a single point.
	Also, w.l.o.g, suppose $f$ is strictly increasing (otherwise replace $f$ with $-f$).
	
	Let $J = f(I)$. Since $f$ is strictly monotone it is injective. Also, since by definition $J = f(I)$, it is surjective, hence bijective. 
	Therefore there exists a unique inverse $g = f^{-1}:J \to I$.
	
	Because $f$ is strictly increasing, we have
	\begin{equation}
		\label{eq:f_strict_incr}
		x_1 < x_2 \quad \Leftrightarrow \quad f(x_1) < f(x_2) \qquad \forall x_1, x_2 \in I.
	\end{equation}
	(Note: here we have equivalence in the statements because $f$ is both injective and strictly increasing)
	Defining $y_1 = f(x_1)$ and $y_2 = f(x_2)$, this is equivalent to
	\[
		y_1 < y_2 \quad \Leftrightarrow \quad g(y_1) < g(y_2) \qquad \forall y_1, y_2 \in J
	\]
	Thus, $g$ is strictly increasing.
	
	To show that $J$ is an interval, $y_1, y_2 \in J$, and assume w.l.o.g that $y_ 1 < y_2$. Since, $J = f(I)$, Equation \eqref{eq:f_strict_incr} implies that $y_1 = f(x_1), y_2 = f(x_2)$ for some $x_1, x_2 \in I$ with $x_1 < x_2$.
	Now by the Intermediate Value Theorem \ref{theo*int_val_theo} applied to $f:[x_1, x_2] \to \mathbb{R}$, we have that all values $c \in [y_1, y_2]$ are in the image of $f:[x_1, x_2] \to \mathbb{R}$, i.e.,
	\[
		[y_1, y_2] \subseteq f([x_1, x_2]) \subseteq J.
	\]
	Since, $y_1, y_2$ were two arbitrary points in $J$, this proves that $J$ is an interval.
	
	It remains to show that $g = f^{-1}$ is continuous. 
	Fix $\bar{y} \in J$ and suppose, by contradiction, that $g$ is not continuous at $\bar{y}$. Then by Remark \ref{rmk:neg_seq_cont}, there exists $\varepsilon > 0$ and a sequence $(y_n)_{n=0}^{\infty} \subseteq J$ such that
	\begin{equation}
		\label{eq:g_not_cont}
		y_n \longrightarrow \bar{y} \qquad \text{but} \qquad |g(y_n) - g(\bar{y})| \geq \varepsilon \quad \forall n \in \mathbb{N}.
	\end{equation}
	Set $x_n = g(y_n) \in I$ and $\bar{x} = g(\bar{y}) \in I$.
	Then for every $n \in \mathbb{N}$, either $x_n \leq \bar{x} - \varepsilon$ or $x_n \geq \bar{x} + \varepsilon$.
	In particular, at least one of these cases must occur infinitely often. W.l.o.g, assume $x_n \leq \bar{x} - \varepsilon$ for infinitely many $n$, and extract a subsequence $(x_{n_k})_{k=0}^{\infty}$ with $x_{n_k} \leq \bar{x} - \varepsilon$ for all $k$. 
	Since, $I$ is an interval, $\bar{x} - \varepsilon \in I$, and by strict monotonicity of $f$ we obtain
	\[
		y_{n_k} = f(x_{n_k}) \leq f(\bar{x} - \varepsilon) < f(\bar{x}) = \bar{y}.	
	\]
	Then Proposition \ref{prop*lim_ineq} gives (recall $y_n \longrightarrow \bar{y}$, see \eqref{eq:g_not_cont})
	\[
		\bar{y} = \lim_{k \to \infty} y_{n_k} \leq f(\bar{x} - \varepsilon) < f(\bar{x}) = \bar{y},
	\]
	a contradiction. Hence, $g$ is continuous. \qedhere
\end{proof}

\subsection{Continuous Functions on Compact Intervals}
In this section we show that continuous functions on \textbf{bounded closed} intervals, called \textbf{compact intervals}, enjoy special properties.

\subsubsection{Boundedness and Extrema}

\begin{lemma}{Compactness}{compact}
	Let $[a, b]$ be a compact interval, and let $(x_n)_{n=0}^{\infty}$ be a sequence contained in $[a, b]$.
	Then there exists a subsequence $(x_{n_k})_{k=0}^{\infty}$ such that
	\[
		\lim_{k \to \infty} x_{n_k} = \bar{x} \qquad \text{for some } \bar{x} \in [a, b].
	\]
\end{lemma}

\begin{proof}
	Since $(x_n)_{n=0}^{\infty}$ is bounded (as it lies in $[a, b]$), Corollary \ref{cor*bound_seq_conv_subseq} ensures the existence of a convergent subsequence $(x_{n_k})_{k=0}^{\infty}$. Let $\bar{x}$ denote its limit. Because $a \leq x_{n_k} \leq b$ for all k, Proposition \ref{prop*lim_ineq} yields $a \leq \bar{x} \leq b$. \qedhere
\end{proof}

\begin{theorem}{Boundedness}{boundedness}
	Let $[a, b]$ be compact interval, and let $f:[a, b] \to \mathbb{R}$ be continuous. Then $f$ is bounded.
\end{theorem}

\begin{proof}
	Assume by contradiction that $f$ is unbounded. Then, for every $n \in \mathbb{N}$, there exists $x_n \in [a, b]$ such that $|f(x_n)| \geq n$. By Lemma \ref{lem*compact}, there is a subsequence $(x_{n_k})_{k=0}^{\infty}$ converging to some $\bar{x} \in [a, b]$.
	
	Since $f$ is continuous, so is $|f|$ (recall Remark \ref{rmk:modulus_cont}), therefore $|f(x_{n_k})| \longrightarrow |f(\bar{x})| \in \mathbb{R}$. This contradicts $|f(x_{n_k})| \geq n_k \longrightarrow \infty$, so $f$ must be bounded. \qedhere
\end{proof}

\begin{exercise}
	Find examples of:
	\begin{enumerate}
		\item a continuous but unbounded function on a bounded \textit{open} interval.
		\[
			f:(0, 1) \to \mathbb{R},\; x \mapsto \frac{1}{x}.
		\]
		\item a continuous but unbounded function on an \textit{unbounded closed} interval.
		\[
			f:[0, \infty) \to \mathbb{R},\; x \mapsto x.
		\]
		\item an unbounded function on a compact interval but discontinuous at only one point.
		\[
			f:[0, 1] \to \mathbb{R},\; x \mapsto \begin{cases}
				\frac{1}{x}, \quad &\text{for } x \neq 0\\
				a \in \mathbb{R}, \quad &\text{for } x = 0.
			\end{cases}
		\]
	\end{enumerate}
\end{exercise}

\begin{definition}{Extreme Values}{extreme_val}
	Let $D \subseteq \mathbb{R}$ and $f: D \to \mathbb{R}$.
	\begin{itemize}
		\item[$\bullet$] We say that $f$ takes its \textbf{maximum value} at $x_0 \in D$ if $f(x) \leq f(x_0)$ for all $x \in D$.
		Then $f(x_0)$ is the \textbf{maximum} of $f$.
		\item[$\bullet$] We say that $f$ takes its \textbf{minimum value} at $x_0 \in D$ if $f(x) \geq f(x_0)$ for all $x \in D$.
		Then $f(x_0)$ is the \textbf{minimum} of $f$.
	\end{itemize}
	Maxima and minima ar called \textbf{extreme values} or \textbf{extrema}.
\end{definition}

\begin{theorem}{Extreme Value Theorem}{extreme_val_theo}
	Let $[a, b]$ be a compact interval, and let $f:[a, b] \to \mathbb{R}$ be continuous. Then $f$ attains both its minimum and its maximum.
\end{theorem}

\begin{proof}
	Theorem \ref{theo*boundedness} guarantees that $f$ is bounded, or equivalently, that $f([a, b]) \subseteq \mathbb{R}$ is a bounded subset of $\mathbb{R}$. Thus, Theorem \ref{theo*sup_exist} implies that
	\[
		S := \sup f([a, b])
	\]
	exists.
	By definition of the supremum, for each $n \in \mathbb{N}$ there exists $y_n \in f([a, b])$ such that $S - 2^{-n} \leq y_n \leq S$. Hence, $y_n \to S$. Also, since $y_n \in f([a, b])$, there exists $x_n \in [a, b]$ such that $f(x_n) = y_n$.
	
	Now, by Lemma \ref{lem*compact}, we can find a subsequence $(x_{n_k})_{k=0}^{\infty}$ such that $x_{n_k} \to \bar{x} \in [a, b]$. By continuity of $f$, we have that
	\[
		f(\bar{x}) = \lim_{k \to \infty} f(x_{n_k}) = \lim_{k \to \infty} y_{n_k} = S,
	\]
	so $f$ attains its maximum at $\bar{x}$.
	
	Applying the same reasoning to $-f$ shows that $f$ also attains its minimum. \qedhere
\end{proof}

\subsubsection{Uniform Continuity}

\begin{definition}{Uniform Continuity}{unif_cont}
	Let $D \subseteq \mathbb{R}$. A function $f:D \to \mathbb{R}$ is \textbf{uniformly continuous} if, for every $\varepsilon > 0$, there exists $\delta > 0$ such that
	\[
		|x - y| < \delta \quad \Rightarrow \quad |f(x) - f(y)| < \varepsilon \qquad \forall x,y \in D.
	\]
\end{definition}

\begin{remark}
	The difference between the usual definition of continuity and the one of uniform continuity lies in how the choice of $\delta$ depends on the points considered.
	
	For a function that is continuous at each $x_0 \in D$, the $\delta$ in the definition may depend on both $\varepsilon$ \textbf{and} $x_0$: for every $\varepsilon > 0$ and each $x_0$, we can find a $\delta = \delta(\varepsilon, x_0)$ that works near $x_0$.
	
	Uniform continuity is stronger: there exists a single $\delta = \delta(\varepsilon)$ that works \textbf{simultaneously} for all $x, y \in D$. In other words, the control on the variation of $f$ does not deteriorate as we move along the domain. This property is automatically satisfied on compact intervals for continuous functions, as we will prove below.
\end{remark}

\begin{theorem}{Uniform Continuity on Compact Intervals}{unif_cont_compact_int}
	Let $[a, b]$ be a compact interval, and $f:[a, b] \to \mathbb{R}$ continuous on $[a, b]$. Then $f$ is uniformly continuous.
\end{theorem}

\begin{proof}
	Assume, by contradiction, that $f$ is not uniformly continuous on $[a, b]$. Then there exists $\varepsilon > 0$ such that for every $\delta > 0$ one can find $x, y \in [a, b]$ with
	\[
		|x - y| < \delta \quad \text{and} \quad |f(x) - f(y)| \geq \varepsilon.
	\]
	Taking $\delta = 2^{-n}$ for each $n \in \mathbb{N}$, we obtain sequences $(x_n)_{n=0}^{\infty}$ and $(y_n)_{n=0}^{\infty}$ in $[a, b]$ with
	\begin{equation}
		\label{eq:contra_unif_cont}
		|x_n - y_n| < 2^{-n} \quad \text{and} \quad |f(x_n) - f(y_n)| \geq \varepsilon.
	\end{equation}
	By Lemma \ref{lem*compact}, the sequence $(x_n)_{n=0}^{\infty}$ has a subsequence $(x_{n_k})_{k=0}^{\infty}$ converging to some $\bar{x} \in [a, b]$. Then
	\[
		|y_{n_k} - \bar{x}| \leq |y_{n_k} - x_{n_k}| + |x_{n_k} - \bar{x}| < 2^{-n_k} + |x_{n_k} - \bar{x}| \underset{k \to \infty}{\longrightarrow} 0,
	\]
	so $y_{n_k} \to \bar{x}$ as well. Thus, by continuity of $f$ and Theorem \ref{theo*cont_seq_cont}, we have that
	\[
		\lim_{k \to \infty} f(x_{n_k}) = \lim_{k \to \infty} f(y_{n_k}) = f(\bar{x}),
	\]
	therefore,
	\[
		|f(x_{n_k}) - f(y_{n_k})| \leq |f(x_{n_k}) - f(\bar{x})| + |f(\bar{x}) - f(y_{n_k})| \underset{k \to \infty}{\longrightarrow} 0,
	\]
	which contradicts Equation \eqref{eq:contra_unif_cont}. Hence, $f$ is uniformly continuous on $[a, b]$. \qedhere
\end{proof}

\begin{definition}{Lipschitz Continuity}{lip_cont}
	Let $D \subseteq \mathbb{R}$, and $f: D \to \mathbb{R}$. We say that $f$ is \textbf{Lipschitz continuous} if there exists $L \geq 0$ such that
	\[
		|f(x) - f(y)| \leq L|x - y| \qquad \forall x, y \in D.
	\]
\end{definition}

\begin{lemma}{Lipschitz Continuity $\Rightarrow$ Uniform Continuity}{lip_cont_unif_cont}
	Let $D \subseteq \mathbb{R}$, and $f:D \to \mathbb{R}$ be a Lipschitz continuous function. Then $f$ is uniformly continuous.
\end{lemma}

\begin{proof}
	Let $D \subseteq \mathbb{R}$ and assume that $f:D \to \mathbb{R}$ is a Lipschitz continuous function. Then there exists $L \geq 0$ such that
	\[
		|f(x) - f(y)| \leq L|x - y| \qquad \forall x, y \in D.
	\]
	Now, fix $\varepsilon > 0$. We assume that $L \neq 0$ (otherwise the result follows immediately) and choose $\delta = \frac{\varepsilon}{L}$.
	Because of the Lipschitz continuity of $f$, we have that for all $x, y \in D$ it holds that
	\begin{align*}
		|x - y| < \delta = \frac{\varepsilon}{L} \quad \Leftrightarrow \quad L|x - y| < \varepsilon\\
		\Rightarrow \qquad |f(x) - f(y)| \leq L|x - y| < \varepsilon,
	\end{align*}
	which shows that $f$ is also uniformly continuous.\qedhere
\end{proof}


\subsection{Example: Exponential and Logarithmic Functions}

\subsubsection{Definition of the Exponential Function}

\begin{lemma}{Bernoulli's Inequality}{bernoulli_ineq}
	For all $a \in \mathbb{R}$ with $a \geq -1$ and all $n \in \mathbb{N}$ with $n \geq 1$, it holds that
	\[
		(1 + a)^{n} \geq 1 + na.
	\]
\end{lemma}

\begin{proof}
	We proceed by induction. For $n = 1$ we have $(1 + a)^1 = 1 + a = 1 + 1 \cdot a$.
	
	Now assume that the inequality holds for some $n \geq 1$. Since $1 + a \geq 0$ by assumption, we find
	\[
		(1 + a)^{n + 1} = (1 + a)^n (1 + a) \geq (1 + n a) (1 + a) = 1 + na + a + na^2 \geq 1 + (n + 1)a,
	\]
	which establishes the induction step and completes the proof. \qedhere
\end{proof}

\begin{proposition}{Existence of the Exponential}{exp_exist}
	Let $x \in \mathbb{R}$. The sequence $(a_n)_{n=1}^{\infty}$ defined by
	\[
		a_n = \left(1 + \frac{x}{n}\right)^n
	\]
	is convergent, and its limit is a positive real number.
\end{proposition}

\begin{lemma}{Monotonicity}{exp_mono}
	Given $x \in \mathbb{R}$, let $n_0 \in \mathbb{N}$ satisfy $n_0 \geq 1$ and $n_0 > -x$. Then the sequence $(a_n)_{n=n_0}^{\infty}$ defined in Proposition \ref{prop*exp_exist} is increasing.
\end{lemma}

\begin{definition}{Exponential Function}{exp}
	The \textbf{exponential function} $\exp:\mathbb{R} \to \mathbb{R}_{>0}$ is defined by
	\[
		\exp(x) = \lim_{n \to \infty} \left(1 + \frac{x}{n}\right)^n \qquad \forall x \in \mathbb{R}.
	\]
\end{definition}

\begin{corollary}{Growth of the Exponential}{growth_exp}
	Given $n \in \mathbb{N}$ with $n \geq 1$, the exponential function satisfies
	\[
		\exp(x) \geq \left(1 + \frac{x}{n}\right)^n \qquad \forall x > -n.
	\]
\end{corollary}

\begin{proof}
	By Lemma \ref{lem*exp_mono} and Definition \ref{def*exp}, for $x > -n$ we have
	\[
		a_n \leq a_{n+1} \leq \hdots \leq \exp(x). \qedhere
	\]
\end{proof}

\subsubsection{Properties of the Exponential Function}

\begin{theorem}{Properties of the Exponential Function}{exp_properties}
	The exponential function $\exp:\mathbb{R} \to \mathbb{R}_{>0}$ is bijective, strictly increasing, and continuous. Moreover,
	\begin{align*}
		\exp(0) &= 1,\\
		\exp(-x) &= \exp(x)^{-1},\\
		\exp(x + y) &= \exp(x)\exp(y),
	\end{align*}
	for all $x, y \in \mathbb{R}$.
\end{theorem}

\subsubsection{The Natural Logarithm}
\begin{definition}{Logarithm}{log}
	The unique inverse function
	\[
		\log:\mathbb{R}_{>0} \to \mathbb{R}
	\]
	of the bijective map $exp:\mathbb{R} \to \mathbb{R}_{>0}$ is called the \textbf{logarithm}.
\end{definition}

\begin{corollary}{Properties of the Logarithm}{log_properties}
	The logarithm $\log:\mathbb{R}_{>0} \to \mathbb{R}$ is strictly increasing, continuous, and bijective. Moreover,
	\begin{align*}
		\log(1) &= 0,\\
		\log(a^{-1}) &= -\log(a),\\
		\log(ab) &= \log(a) + \log(b),
	\end{align*}
	for all $a, b \in \mathbb{R}_{>0}$.
\end{corollary}

The logarithm defined here is also called the \textbf{natural logarithm} to distinguish it from logarithms with another \textbf{base} $a > 1$ (for instance $a = 10$ or $a = 2$). For any $a > 1$, we define
\[
	\log_a(x) = \frac{\log(x)}{\log(a)} \qquad \forall x > 0.
\]
Unless stated otherwise, $\log(x)$ always denotes the natural logarithm, i.e., the logarithm to base $e$.

We can now define powers with arbitrary real exponents. For $a > 0$ and $x \in \mathbb{R}$ we set
\[
	a^x = \exp(x\,\log(a)).
\]

\subsection{Limits of Functions}
We consider functions $f:D \to \mathbb{R}$ defined on a subset $D \subseteq \mathbb{R}$, and we wish to define the limit of $f(x)$ as $x \in D$ approaches a point $x_0 \in \mathbb{R}$. Typical examples include $D = \mathbb{R}, D = [0, 1]$ or $D = (0, 1)$, with $x_0 = 0$ in each case.

\subsubsection{Limit in the Vicinity of a Point}
\label{sec:lim_vicinity_pt}
Let $D \subseteq \mathbb{R}$ be non-empty, and let $x_0 \in \mathbb{R}$ be such that
\begin{equation}
	\label{eq:acc_pt}
	D \cap (x_0 - \delta, x_0  + \delta) \neq \emptyset
\end{equation}
for all $\delta > 0$. Whenever this holds, we say that $x_0$ is an \textbf{accumulation point} of $D$. Note that if $x_0 \in D$, then Equation \eqref{eq:acc_pt} is automatically satisfied.

Condition \ref{eq:acc_pt} ensures that there exists a sequence of points in $D$ converging to $x_0$.

\begin{definition}{Limit of a Function}{lim_func}
	Let $f:D \to \mathbb{R}$, and $x_0$ be an accumulation point of $D$.
	A number $L \in \mathbb{R}$ is called the \textbf{limit of} $f(x)$ \textbf{as} $x \to x_0$ if, for every $\varepsilon > 0$, there exists $\delta > 0$ such that
	\[
		|x - x_0| < \delta \quad \Rightarrow \quad |f(x) - L| < \varepsilon \qquad \forall x \in D.
	\]
\end{definition}
In general, the limit of $f(x)$ as $x \to x_0$ may not exist. However, if it exists, it is uniquely determined. Hence we speak of \textit{the} limit and write
\[
	\lim_{x \to x_0} f(x) = L
\]
to indicate the limit exists and is equal to $L$. Informally, this means that the function values $f(x)$ are arbitrarily close to $L$ whenever $x \in D$ is sufficiently close to $x_0$.

The limit of a function satisfies properties analogous to those of Proposition \ref{prop*lim_ineq}. More precisely, if $f, g$ are functions on $D$ such that
\[
	\lim_{x \to x_0} f(x) = L_1 \qquad \text{and} \qquad \lim_{x \to x_0} g(x) = L_2,
\]
then
\[
	\lim_{x \to x_0} (f+g)(x) = L_1 + L_2, \qquad \lim_{x \to x_0} (f\cdot g)(x) = L_1\cdot L_2.
\]
Moreover, $f \leq g$ implies $L_1 \leq L_2$, and the sandwich lemma holds: if $f \leq h \leq g$ and $L_1 = L_2$ then $\lim_{x \to x_0} h(x) = L_1 = L_2$.

\begin{remark}
	Let $f:D \to \mathbb{R}$ be a function. If $x_0 \in D$, then $f$ is continuous at $x_0$ if and only if $\lim_{x \to x_0} f(x) = f(x_0)$.
\end{remark}

Suppose that $x_0 \in D$ is an accumulation point of $D \setminus \{x_0\}$. Let $f:D \to \mathbb{R}$, and consider the restriction $f|_{D \setminus \{x_0\}}$. It may happen that $f$ is discontinuous at $x_0$, but the limit
\begin{equation}
	L = \lim_{x \to x_0} f|_{D \setminus \{x_0\}}(x)
\end{equation}
nevertheless exists. In this case, the point $x_0$ is called a \textbf{removable discontinuity} of $f$, and one also writes
\begin{equation}
	\label{eq:limit_neq_x}
	L = \underset{x \neq x_0}{\lim_{x \to x_0}} f(x).
\end{equation}
If we now define
\begin{equation}
	\label{eq:cont_ext}
	\tilde{f}(x) = \begin{cases}
		f(x), \quad &x \in D \setminus \{x_0\},\\
		L, \quad &x = x_0,
	\end{cases}
\end{equation}
then $\tilde{f}$ is continuous at $x_0$. In other words, we can remove the discontinuity of $f$ by redefining its value at $x_0$ to be $L$.

If instead $x_0 \notin D$ but the limit in Equation \eqref{eq:limit_neq_x} exists, we call the function $\tilde{f}$ defined in Equation \eqref{eq:cont_ext} the \textbf{continuous extension} of $f$ to $D \cup \{x_0\}$.

Arguing as in the proof of Theorem \ref{theo*cont_seq_cont}, we obtain the following result.

\begin{lemma}{Limit and Sequences}{lim_and_seq}
	Let $f:D \to \mathbb{R}$. Then $L = \lim_{x \to \bar{x}} f(x)$ if and only if, for every sequence $(x_n)_{n=0}^{\infty} \subseteq D$ converging to $\bar{x}$, one has $\lim_{n \to \infty} f(x_n) = L$.
\end{lemma}

We now state a result describing the behaviour of limits under composition with a continuous function.

\begin{proposition}{Limit and Composition}{lim_and_comp}
	Let $E \subseteq \mathbb{R}$, and let $f:D \to E$ be such that the limit $L = \lim_{x \to \bar{x}} f(x)$ exists and belongs to $E$. If $g:E \to \mathbb{R}$ is continuous at $L$, then
	\[
		\lim_{x \to \bar{x}} g(f(x)) = g(L).
	\]
\end{proposition}

\begin{proof}
	Let $(x_n)_{n=0}^{\infty} \subseteq D$ be a sequence converging to $\bar{x}$. By Lemma \ref{lem*lim_and_seq}, we have $\lim_{n \to \infty} f(x_n) = L$. Since $g$ is continuous at $L$, Theorem \ref{theo*cont_seq_cont} gives $\lim_{n \to \infty} g(f(x_n)) = g(L)$. Because $(x_n)_{n=0}^{\infty}$ was arbitrary, using Lemma \ref{lem*lim_and_seq} again, we conclude that $\lim_{x \to \bar{x}} g(f(x)) = g(L)$. \qedhere
\end{proof}

We now introduce conventions for improper limits of functions, in analogy with improper limits for sequences.

\begin{definition}{Improper Limits}{improper_lim}
	Let $f:D \to \mathbb{R}$, and let $x_0$ be an accumulation point of $D$.
	We say that $f$ \textbf{diverges to} $+\infty$ \textbf{as} $x \to x_0$, and write
	\[
		\lim_{x \to x_0} f(x) = + \infty,
	\]
	if for every $M > 0$, there exists $\delta > 0$ such that
	\[
		\forall x \in D: \quad |x - x_0| < \delta \quad \Rightarrow \quad f(x) \geq M.
	\]
	Analogously, $f$ \textbf{diverges to} $-\infty$ \textbf{as} $x \to x_0$ and we write $\lim_{x \to x_0} f(x) = -\infty$, if for every $M > 0$, there exists $\delta > 0$ such that
	\[
		\forall x \in D: \quad |x - x_0| < \delta \quad \Rightarrow \quad f(x) \leq -M.
	\]
\end{definition}

\subsubsection{One-Sided Limits}
It is often useful to consider limits taken form one side only and to allow $x_0$ to be $\pm \infty$ as well. To this end, let $x_0 \in \mathbb{R}$ be such that
\begin{equation}
	\label{eq:x_right_acc_pt}
	D \cap (x_0, x_0 + \delta) \neq \emptyset
\end{equation}
for every $\delta > 0$. In this case, we say that $x_0$ is a \textbf{right-hand accumulation point} of $D$. Analogously, if
\begin{equation}
	\label{eq:x_left_acc_pt}
	D \cap (x_0 - \delta, x_0) \neq \emptyset
\end{equation}
for every $\delta > 0$, we say that $x_0$ is a \textbf{left-hand accumulation point} of $D$.

\begin{definition}{One-Sided Limits}{one_side_lim}
	Let $f:D \to \mathbb{R}$, and let $x_0 \in \mathbb{R}$ be a right-hand accumulation point of $D$. A number $L \in \mathbb{R}$ is called the \textbf{right-hand limit} of $f$ at $x_0$ if, for every $\varepsilon > 0$, there exists $\delta > 0$ such that
	\[
		x \in D \cap (x_0, x_0 + \delta) \quad \Rightarrow \quad |f(x) - L| < \varepsilon.
	\]
	In this case we write $L = \lim_{x \to x_0^+} f(x)$.
	We also allow improper one-sided limits. We say that
	\[
		\lim_{x \to x_0^+} f(x) = +\infty
	\]
	if for every $M > 0$, there exists $\delta > 0$ such that
	\[
		x \in D \cap (x_0, x_0 + \delta) \quad \Rightarrow \quad f(x) \geq M.
	\]
	Similarly, $\lim_{x \to x_0^+} f(x) = -\infty$ means that, for every $M > 0$, there exists $\delta > 0$ such that
	\[
		x \in D \cap (x_0, x_0 + \delta) \quad \Rightarrow \quad f(x) \leq -M.
	\]
	The \textbf{left-hand limit} is defined analogously, considering a left-hand accumulation point of $D$ and writing $\lim_{x \to x_0^-} f(x)$.
\end{definition}

Next, we define the notion of limit at infinity.

\begin{definition}{Limits at Infinity}{lim_at_inf}
	Let $f:D \to \mathbb{R}$, and assume that $D \cap (R, \infty) \neq \emptyset$ for every $R > 0$.
	A number $L \in \mathbb{R}$ is called the \textbf{limit of} $f$ \textbf{as} $x \to +\infty$ if, for every $\varepsilon > 0$, there exists $R > 0$ such that
	\[
		x \in D \cap (R, \infty) \quad \Rightarrow \quad |f(x) - L| < \varepsilon.
	\]
	We say that $f$ \textbf{diverges to} $+\infty$ \textbf{as} $x \to +\infty$ if, for every $M > 0$, there exists $R > 0$ such that
	\[
		x \in D \cap (R, \infty) \quad \Rightarrow \quad f(x) \geq M.
	\]
	The corresponding definition for $x \to -\infty$ and diverges to $-\infty$ are analogous.
\end{definition}

Limits at $+\infty$ can be converted into right-hand limits at 0 via inversion. Given $f:D \to \mathbb{R}$ as above, define
\[
	E = \{x > 0 \;|\; x^{-1} \in D\}, \qquad g:E \to \mathbb{R}, \qquad g(x) = f(x^{-1}).
\]
Then 
\[
	\lim_{x \to +\infty} f(x) = \lim_{x \to 0^+} g(x),  
\]
so one limit exists if and only if the other does.

\begin{definition}{One-Sided Continuity and Jumps}{one_side_cont_jump}
	Let $f:D \to \mathbb{R}$ and $x_0 \in D$. If $\lim_{x \to x_0^+} f(x)$ exists and equals $f(x_0)$, then $f$ is \textbf{continuous from the right} at $x_0$. \textbf{Continuity form the left} is defined similarly.
	We call $x_0$ a \textbf{jump point} if both one-sided limits exist but are different, i.e.,
	\[
		L_{-} := \lim_{x \to x_0^-} f(x) \in \mathbb{R}, \qquad L_+ := \lim_{x \to x_0^+} f(x) \in \mathbb{R}, \qquad L_{-} \neq L_+.
	\]
\end{definition}

\subsubsection{Landau Notation}
We introduce two standard notations that compare the asymptotic behaviour of a function to that of another function. (often called \textit{relative} asymptotics).

\begin{definition}{Big-O at a Point}{big_o}
	Let $f,g:D \to \mathbb{R}$, and let $x_0$ be an accumulation point of $D$. We write
	\[
		f(x) = O(g(x)) \quad \text{as } x \to x_0
	\]
	if there exists $M > 0$ and $\delta > 0$ such that
	\[
		x \in D \cap (x_0 - \delta, x_0 + \delta) \quad \Rightarrow \quad |f(x)| \leq M |g(x)|.
	\]
	We then say that $f$ is a \textbf{Big-O} of $g$ as $x \to x_0$.
\end{definition}
If $g(x) \neq 0$ for all $x$ sufficiently close to $x_0$ (with $x \in D$), then
\[
	f(x) = O(g(x)) \quad \text{as } x \to x_0 \qquad \Leftrightarrow \qquad \frac{f(x)}{g(x)} \text{ is bounded near $x_0$}.
\]

\begin{definition}{Big-O at Infinity}{big_o_inf}
	Let $f, g:D \to \mathbb{R}$, and assume $D \cap (R, \infty) \neq \emptyset$ for every $R > 0$. We write
	\[
		f(x) = O(g(x)) \quad \text{as } x \to +\infty
	\]
	if there exists $M > 0$ and $R > 0$ such that
	\[
		x \in D \cap (R, \infty) \quad \Rightarrow \quad |f(x)| \leq M |g(x)|.
	\]
	The definition for $x \to -\infty$ is analogous.
\end{definition}

The big-O notation hides the precise bound by an \textit{implicit constant} $M$, which is often irrelevant for the argument one is interested in.

\subsubsection*{Example}
\begin{itemize}
	\item[$\bullet$] if $f$ and $g$ are bounded and continuous near $x_0$ with $g(x_0) \neq 0$, then $f(x) = O(g(x))$ as $x \to x_0$.
	\item[$\bullet$] As $x \to 0$, one has $x^2 = O(x)$, but $x \neq O(x^2)$ (since $x / x^2$ is unbounded near 0).
	\item[$\bullet$] As $x \to +\infty$, $\frac{3x^3}{x^3 + 3} = O(1)$, but $\frac{3x^3}{x^3 + 3} \neq O(x^{\alpha})$ for $\alpha < 0$.
\end{itemize}

As discussed above, the big-O means that $f$ is bounded by a multiple of $g$. One may also consider a stronger condition, namely that $f$ is asymptotically negligible with respect to $g$. This leads to the following definition.

\begin{definition}{Little-O at a Point}{lil_o}
	Let $f, g:D \to \mathbb{R}$, and let $x_0$ be an accumulation point of $D$. We write
	\[
		f(x) = o(g(x)) \quad \text{as } x\to x_0
	\]
	if, for every $\varepsilon > 0$ there exists $\delta > 0$ such that
	\[
		x \in D \cap (x_0 - \delta, x_0 + \delta) \quad \Rightarrow \quad |f(x)| \leq \varepsilon |g(x)|.
	\]
	We then say that $f$ is a \textbf{little-o} of $g$ as $x \to x_0$.
\end{definition}

If $g(x) \neq 0$ for all $x$ near $x_0$ (with $x \in D$), then
\[
	f(x) = o(g(x)) \quad \text{as } x \to x_0 \qquad \Leftrightarrow \qquad \lim_{x \to x_0} \frac{f(x)}{g(x)} = 0.
\]
Moreover, $f(x) = o(g(x)) \; \Rightarrow \; f(x) = O(g(x))$.

\begin{definition}{Little-o at Infinity}{lil_o_inf}
	Let $f, g:D \to \mathbb{R}$, and assume that $D \cap (R, \infty) \neq \emptyset$ for every $R > 0$. We write
	\[
		f(x) = o(g(x)) \quad \text{as } x \to +\infty
	\]
	if, for every $\varepsilon > 0$ there exists $R > 0$ such that
	\[
		x \in D \cap (R, \infty) \quad \Rightarrow \quad |f(x)| \leq \varepsilon |g(x)|.
	\]
	The definition for $x \to -\infty$ is analogous.
\end{definition}

\subsubsection*{Example}
\begin{itemize}
	\item[$\bullet$] $x = o(x^2)$ as $x \to +\infty$, and $x^2 = o(x)$ as $x \to 0$
	\item[$\bullet$] For any $\alpha < 1$,
	\[
		\frac{3x^3}{2x^2 + x^10} = o(|x|^{\alpha}) \quad \text{as } x\to 0,
	\]
	but not for $\alpha \geq 1$. Indeed,
	\[
		\big|\frac{3x^3}{|x|^{\alpha} 2x^2 + x^10}\big|= |x|^{1 - \alpha} \frac{3}{2 + x^8} \longrightarrow 0 \quad \text{as } x \to 0,
	\]
	whenever $\alpha < 1$.
\end{itemize}

In computations, one often uses Landau symbols as placeholders. Writing
\[
	f(x) = o(g(x)) \quad \text{as } x \to x_0
\]
means there is a function $h:D \to \mathbb{R}$ with $h(x) = o(g(x))$ as $x \to x_0$. Similarly for big-O.

\subsubsection*{Example}
Polynomial division gives, as $x \to +\infty$,
\[
	\frac{x^3 - 7x^2 + 6x + 2}{x^2} = x - 7 + O\left(\frac{1}{x}\right) = x - 7 + o(1) = x + O(1) = x + o(x).
\]

\subsection{Sequences of Functions}

\subsubsection{Pointwise Convergence}

\begin{definition}{Sequences of Functions}{seq_func}
	A \textbf{sequence} of real-valued on a subset $D \subseteq \mathbb{R}$ is a family of functions $f_n: D \to \mathbb{R}$ indexed by $\mathbb{N}$. The function $f_n$ is called the n-th \textbf{element} of the sequence. One often writes $(f_n)_{n \in \mathbb{N}}$, $(f_n)_{n=0}^{\infty}$, or $(f_n){n\geq 0}$ for a sequence of functions.
\end{definition}

\begin{definition}{Pointwise Convergence}
	Let $D \subseteq \mathbb{R}$, and let $(f_n)_{n=0}^{\infty}$ be a sequence of functions $f_n:D \to \mathbb{R}$. Let $f:D \to \mathbb{R}$ be another function. We say that $(f_n)_{n=0}^{\infty}$ \textbf{converges pointwise} to $f$, if for every $x \in D$, the sequence of real numbers $(f_n(x))_{n=0}^{\infty}$ converges to $f(x)$, i.e., for every $x \in D$ and for every $\varepsilon > 0$, there exists $n \in \mathbb{N}$ such that
	\[
		|f_n(x) - f(x)| < \varepsilon \qquad \forall n \geq N.
	\]
	In this case, $f$ is called the \textbf{pointwise limit} of the sequence $(f_n)_{n=0}^{\infty}$.
\end{definition}

\begin{remark}
	Note that, in the definition of pointwise convergence the index $N$ may depend on both $x$ and $\varepsilon$. In other words, for each point $x \in D$ we examine the convergence of $(f_n)_{n=0}^{\infty}$ to $f$ separately.
\end{remark}

\subsubsection{Uniform Convergence}

\begin{definition}{Uniform Convergence}{unif_conv}
	Let $D \subseteq \mathbb{R}$, and let $(f_n)_{n=0}^{\infty}$ be a sequence of functions $f_n:D \to \mathbb{R}$. Let $f:D \to \mathbb{R}$ be another function. We say that $(f_n)_{n=0}^{\infty}$ \textbf{converges uniformly} to $f$ on $D$ if, for every $\varepsilon > 0$, there exists $N \in \mathbb{N}$ such that
	\[
		|f_n(x) - f(x)| < \varepsilon \qquad \forall n\geq N,\; \forall x \in D.
	\]
\end{definition}

\begin{remark}
	Note that, in the definition of uniform convergence the index $N$ may only depend on $\varepsilon$, and therefore the condition has to hold for all $x \in D$ for the sequence of functions $(f_n)_{n=0}^{\infty}$ to converge uniformly to $f$ on $D$.
\end{remark}

\begin{remark}
	Let $D \subseteq \mathbb{R}$, and $(f_n)_{n=0}^{\infty}$ be a sequence of functions $f_n:D \to \mathbb{R}$ converging uniformly to $f:D\to \mathbb{R}$ on $D$. Then the sequence of functions $(f_n)_{n=0}^{\infty}$ also converges pointwise to $f$.
\end{remark}

\begin{theorem}{Continuity under Uniform Convergence}{cont_unif_conv}
	Let $D \subseteq \mathbb{R}$, and let $(f_n)_{n=0}^{\infty}$ be a sequence of continuous functions converging uniformly to $f:D \to \mathbb{R}$. Then $f$ is continuous.
\end{theorem}

\begin{proof}
	To prove that $f$ is continuous, we fix $\bar{x} \in D$ and show that $f$ is continuous at $\bar{x}$. Given $\varepsilon > 0$, the uniform convergence of $f_N$ to $f$ provides $N \in \mathbb{N}$ such that
	\[
		|f_N(y) - f(y)| < \frac{\varepsilon}{3} \qquad \forall y \in D.
	\]
	Also, since $f_N$ is continuous at $\bar{x}$, there exists $\delta > 0$ such that
	\[
		|x - \bar{x}| < \delta \quad \Rightarrow \quad |f_N(x) - f_N(\bar{x})| < \frac{\varepsilon}{3}.
	\]
	Then, for $|x - \bar{x}| < \delta$, we have
	\[
		|f(x) - f(\bar{x})| \leq |f(x) - f_N(x)| + |f_N(x) - f_N(\bar{x})| + |f_N(\bar{x}) - f(\bar{x})| < \frac{\varepsilon}{3} + \frac{\varepsilon}{3} + \frac{\varepsilon}{3} = \varepsilon.
	\]
	Which shows that $f$ is continuous at $\bar{x}$. Since, $\bar{x}$ is arbitrary, $f$ is continuous on $D$.
\end{proof}

Intuitively, uniform convergence allows us to \textit{exchange} the order of taking limits. More precisely, assume $(f_n)_{n=0}^{\infty}$ is a sequence of continuous functions converging pointwise to $f$. Then, by the pointwise convergence and the continuity of the functions $f_n$ we have
\[
	f(\bar{x}) = \lim_{n \to \infty} f_n(\bar{x}), \qquad f_n(\bar{x}) = \lim_{x \to \bar{x}} f_n(x), \qquad f(x) = \lim_{n \to \infty} f_n(x) \quad \forall x \in D. 
\]
Hence,
\[
	f(\bar{x}) = \lim_{n \to \infty} f_n(\bar{x}) = \lim_{n \to \infty} \bigg(\lim_{x \to \bar{x}}f_n(x)\bigg), \qquad \lim_{x \to \bar{x}} f(x) = \lim_{x \to \bar{x}} \bigg(\lim_{n \to \infty} f_n(x)\bigg).
\]
Note that the function $f$ is continuous at $\bar{x}$ if and only if $f(\bar{x}) = \lim_{x \to \bar{x}} f(x)$, which by the identities above is equivalent to
\[
	\lim_{x \to \bar{x}} \bigg( \lim_{n \to \infty} f_n(x)\bigg) = \lim_{n \to \infty} \bigg( \lim_{x \to \bar{x}} f_n(x)\bigg).
\]
As we have seen, for pointwise convergent this interchange may fail because $f$ need not be continuous. However, Theorem \ref{theo*cont_unif_conv} ensures that this equality holds under uniform convergence.





	%
% (c) 2025 Autor, ETH Zürich
%
% !TEX root = main.tex
% !TEX encoding = UTF-8
%

\section{Series and Power Series}
In this chapter we study series (infinite sums). They provide a framework to define many classical functions; in particular, we will use series to define trigonometric functions.

\subsection{Series of Real Numbers}

\begin{definition}{Convergent and Divergent Series}{conv_div_series}
	Let $(a_n)_{n=0}^{\infty}$ be a sequence of real numbers, and let $A \in \mathbb{R}$. We say that the series $\sum_{k=0}^{\infty} a_k$ \textbf{converges} to $A$ if
	\[
		\lim_{n \to \infty} \sum_{k=0}^{n} a_k = A.
	\]
	In other words, computing the infinite sum $\sum_{k=0}^{\infty} a_k$ means finding (if it exists) the limit of the \textbf{partial sums}
	\[
		s_n = \sum_{k=0}^{n} a_k, \qquad n \in \mathbb{N}.
	\]
	We call $a_n$ the $n$\textbf{-th term} (or $n$\textbf{-th summand}) of the series. If the limit exists, its value $A$ is the \textbf{sum of the series}. If the limit does not exist, the series is said to be \textbf{not convergent}. In particular, if the sequence of partial sums $(s_n)_{n=0}^{\infty}$ diverges to $+\infty$ (respectively, to $-\infty$), we say that the series \textbf{diverges to} $+\infty$ (respectively, \textbf{to} $-\infty$).
	This situation is therefore a specific case of a series that does not converge.
\end{definition}

\begin{proposition}{Necessary Condition for Convergence}{cond_for_conv}
	If the series $\sum_{k=0}^{\infty} a_k$  converges, then $a_n \to 0$ as $n \to \infty$.
\end{proposition}

\begin{proof}
	By assumption the partial sums $s_n = \sum_{k=0}^{n} a_k$ satisfy $s_n \to A \in \mathbb{R}$. Then for $n \geq 1$, we have 
	\[
		a_n = s_n - s_{n-1} \underset{n \to \infty}{\longrightarrow} A - A = 0. \qedhere
	\]
\end{proof}

\subsubsection*{Geometric Series}
For $q \in \mathbb{R}$, the geometric series $\sum_{k=0}^{\infty} q^k$ converges if and only if $|q| < 1$, and in this case
\[
	\sum_{k=0}^{\infty} q^k = \frac{1}{1 - q}.
\]
Indeed, if the series converges, then by Proposition \ref{prop*cond_for_conv} we must have $q^n \to 0$ as $n \to \infty$, hence $|q| < 1$. Conversely, for $|q| < 1$ one provides by induction that
\[
	s_n = \sum_{k=0}^{n} q^k = \frac{1 - q^{n+1}}{1 - q}\qquad \forall n \in \mathbb{N}, q \neq 1.
\]
Also since $|q| < 1$, $q^{n+1} \to 0$ as $n\to \infty$. Thus,
\[
	s_n = \frac{1 - q^{n+1}}{1 - q} \underset{n\to \infty}{\longrightarrow} \frac{1}{1 - q}.
\]

\subsubsection*{Harmonic Series}
The converse of Proposition \ref{prop*cond_for_conv} fails: the \textbf{harmonic series} $\sum_{k=1}^{\infty} \frac{1}{k}$ does not converge. To see this, consider $n = 2^{\ell}$ with $\ell \in \mathbb{N}$. Grouping terms gives
\begin{align*}
	\sum_{k=1}^{2^{\ell}} \frac{1}{k} &= 1 + \frac{1}{2} + \left(\frac{1}{3} + \frac{1}{4}\right) + \left(\frac{1}{5} + \hdots + \frac{1}{8}\right) + \hdots + \left(\frac{1}{2^{\ell - 1} + 1} + \hdots + \frac{1}{2^{\ell}}\right)\\
	&\geq 1 + \frac{1}{2} + \underbrace{\frac{1}{4} + \frac{1}{4}}_{=\frac{1}{2}} + \underbrace{\frac{1}{8} + \frac{1}{8} + \frac{1}{8} + \frac{1}{8}}_{=\frac{1}{2}} + \hdots + \underbrace{\frac{1}{2^{\ell}} + \frac{1}{2^{\ell}}}_{=\frac{1}{2}}\\
	&= 1 + \underbrace{\frac{1}{2} + \hdots + \frac{1}{2}}_{\ell \text{ - times}} = 1 + \frac{\ell}{2},
\end{align*}
which is unbounded as $\ell \to \infty$.

\begin{lemma}{Convergence of the Tail}{conv_tail}
	Let $\sum_{k=0}^{\infty} a_k$ be a series and fix $N \in \mathbb{N}$. Then $\sum_{k=0}^{\infty} a_k$ is convergent if and only if  $\sum_{k=N}^{\infty} a_k$ is convergent, and in that case
	\[
		\sum_{k=0}^{\infty} a_k = \sum_{k=0}^{N - 1} a_k + \sum_{k=N}^{\infty} a_k.
	\]
	The same equivalence holds for divergence to $+\infty$ or $-\infty$.
\end{lemma}

\begin{proof}
	For every $n \geq N$,
	\[
		\sum_{k=0}^{n} a_k = \sum_{k=0}^{N - 1} a_k + \sum_{k=N}^{n} a_k.
	\]
	Thus, the partial sums of $\sum_{k=0}^{\infty} a_k$ converge if and only if those of $\sum_{k = N}^{\infty} a_k$ do, and the identity in the statement follows by letting $n \to \infty$. The divergence case is analogous.
\end{proof}

\subsubsection{Series with Non-negative Elements}

\begin{proposition}{Non-negative Series: Convergence vs. Divergence}{non_neg_series_conv}
	Let $\sum_{k=0}^{\infty} a_k$ be a series with non-negative terms $a_k \geq 0$ for all $k \in \mathbb{N}$. Then the partial sums $s_n = \sum_{k=0}^{n} a_k$ form an increasing sequence. If $(s_n)_{n=0}^{\infty}$ is bounded, the series $\sum_{k=0}^{\infty} a_k$ converges; otherwise it diverges to $+\infty$.
\end{proposition}

\begin{proof}
	Since $a_{n+1} \geq 0$, we have $s_{n+1} = s_n + a_{n+1} \geq s_n$ for all $n \in \mathbb{N}$, so $(s_n)_{n=0}^{\infty}$ is increasing.
	
	If the sequence $(s_n)_{n=0}^{\infty}$ is bounded, then it converges by Theorem \ref{theo*conv_mono_seq}. If the partial sums are not bounded, then they diverge to $+\infty$. \qedhere
\end{proof}

\begin{remark}
	\label{rmk:series_sub_seq}
	If $\sum_{k=0}^{\infty} a_k$ has non-negative terms, then $(s_n)_{n=0}^{\infty}$ is bounded if and only if it has a bounded subsequence $(s_{n_k})_{k=0}^{\infty}$.
\end{remark}

\begin{corollary}{Comparison Test (Majorant/Minorant)}{maj_min}
	Let $\sum_{k=0}^{\infty} a_k$ and $\sum_{k=0}^{\infty} b_k$ be series with $0 \leq a_k \leq b_k$ for all $k \in \mathbb{N}$. Then
	\[
		0 \leq \sum_{k=0}^{\infty} a_k \leq \sum_{k=0}^{\infty} b_k,
	\]
	and in particular
	\begin{align*}
		\sum_{k = 0}^{\infty} b_k \text{ convergent} \quad &\Rightarrow \quad \sum_{k=0}^{\infty} a_k \text{ convergent},\\
		\sum_{k=0}^{\infty} a_k \text{ divergent to } +\infty \quad &\Rightarrow \quad \sum_{k=0}^{\infty} b_k \text{ divergent to } +\infty.
	\end{align*}
	These implications remain true if the inequalities $0 \leq a_n \leq b_n$ hold only for all $n \geq N$, for some $N \in \mathbb{N}$.
\end{corollary}

\begin{proof}
	From $a_k \leq b_k$ we get $\sum_{k=0}^{n} a_k \leq \sum_{k=0}^{n} b_k$ for all $n \in \mathbb{N}$. Therefore,
	\[
		\sum_{k=0}^{\infty} a_k = \lim_{n \to \infty} \sum_{k=0}^{n} a_k \leq \lim_{n \to \infty} \sum_{k=0}^{n} b_k = \sum_{k=0}^{\infty} b_k.
	\]
	The last part of the statement follows form Lemma \ref{lem*conv_tail}.\qedhere
\end{proof}
Under the assumptions of the corollary, $\sum_{k=0}^{\infty} b_k$ is called a \textit{majorant} of $\sum_{k=0}^{\infty}a_k$, and $\sum_{k=0}^{\infty}a_k$ a \textit{minorant} of $\sum_{k=0}^{\infty} b_k$. Hence the names \textbf{majorant} and \textbf{minorant criterion}.

\begin{proposition}{Cauchy Condensation Test}{cauchy_cond_test}
	Let $(a_k)_{k=0}^{\infty}$ be a decreasing sequence of non-negative numbers. Then
	\[
		\sum_{k=0}^{\infty} a_k \text{ converges} \quad \Leftrightarrow \quad \sum_{k=0}^{\infty} 2^k a_{2^k} \text{ converges}.
	\]
\end{proposition}

\begin{proof}
	Consider the partial sums of the series $\sum_{k=0}^{\infty} a_k$ starting form $k=2$ up to an index that is a power of 2. Since the terms $a_k$ are decreasing, the following inequalities hold:
	\begin{align*}
		\sum_{k=2}^{2^{n+1}} a_k &= a_2 + (a_3 + a_4) + (a_5 + \hdots + a_8) + \hdots + (a_{2^{n} + 1} + \hdots + a_{2^{n+1}})\\
		&\leq \underbrace{a_1}_{=1\cdot a_1} + \underbrace{(a_2 + a_2)}_{=2 \cdot a_2} + \underbrace{(a_4 + \hdots + a_4)}_{=4 \cdot a_4} + \hdots + \underbrace{(a_{2^n}) + \hdots + a_{2^{n}}}_{=2^n \cdot a_{2^n}}\\
		&= a_1 + 2a_2 + 4a_4 + \hdots + 2^n a_{2^n} = \sum_{k=0}^{n} 2^k a_{2^k},
	\end{align*}
	and similarly,
	\begin{align*}
		\sum_{k=2}^{2^{n+1}} a_k &= a_2 + (a_3 + a_4) + (a_5 + \hdots + a_8) + \hdots + (a_{2^{n} + 1} + \hdots + a_{2^{n+1}})\\
		&\geq \underbrace{a_2}_{=1\cdot a_2} + \underbrace{(a_4 + a_4)}_{=2 \cdot a_4} + \underbrace{(a_8 + \hdots + a_8)}_{=4 \cdot a_8} + \hdots + \underbrace{(a_{2^{n+1}}) + \hdots + a_{2^{n+1}}}_{=2^n \cdot a_{2^{n+1}}}\\
		&= \frac{1}{2}(2a_2 + 4a_4 + \hdots + 2^{n+1} a_{2^{n+1}}) = \frac{1}{2}\sum_{k=1}^{n+1} 2^k a_{2^k}.
	\end{align*}
	In other words,
	\[
		\sum_{k=0}^{n} 2^k a_{2^k} \geq \sum_{j=2}^{2^{n+1}} a_j \geq \frac{1}{2}\sum_{k=1}^{n+1} 2^k a_{2^k}. 
	\]
	By Remark \ref{rmk:series_sub_seq} and Corollary \ref{cor*maj_min}, the partial sums of one series are bounded if and only if those of the other are. Hence, the two series converge or diverge together. \qedhere
\end{proof}

\subsubsection{Conditional Convergence}

\begin{definition}{Absolute and Conditional Convergence}{abs_cond_conv}
	A series $\sum_{k=0}^{\infty} a_k$ is \textbf{absolutely convergent} if $\sum_{k=0}^{\infty} |a_k|$ converges. It is \textbf{conditionally convergent} if $\sum_{k=0}^{\infty} a_k$ converges but $\sum_{k=0}^{\infty} |a_k|$ diverges.
\end{definition}
A striking feature of conditionally convergent series is that their terms can be rearranged to obtain any prescribed limit.

\begin{theorem}{Riemann Rearrangement Theorem}{riemann_rearrange}
	Let $\sum_{n=0}^{\infty} a_n$ be a conditionally convergent series and let $A \in \mathbb{R}$. Then there exists a bijection $\varphi: \mathbb{N} \to \mathbb{N}$ such that
	\[
		A = \sum_{n=0}^{\infty} a_{\varphi(n)}.
	\]
\end{theorem}
The proof of this Theorem is extra material.

\subsubsection{Leibniz Criterion for Alternating Series}

\begin{definition}{Alternating Series}{alt_series}
	If $(a_k)_{k=0}^{\infty}$ is a sequence of non-negative numbers, the series
	\[
		\sum_{k=0}^{\infty} (-1)^{k} a_k
	\]
	is called the \textbf{alternating series} associated with the sequence $(a_k)_{k=0}^{\infty}$.
\end{definition}

\begin{proposition}{Leibniz Criterion}{leibniz_crit}
	Let $(a_k)_{k=0}^{\infty}$ be a monotonically decreasing sequence of non-negative numbers with $a_k \to 0$. Then the alternating series $\sum_{k=0}^{\infty} (-1)^k a_k$ converges, and for all $n \in \mathbb{N}$,
	\begin{equation}
		\label{eq:leibniz_crit}
		\sum_{k=0}^{2n+1} (-1)^k a_k \leq \sum_{k=0}^{\infty} (-1)^k a_k \leq \sum_{k=0}^{2n} (-1)^k a_k.
	\end{equation}
\end{proposition}

\begin{proof}
	Let $s_n = \sum_{k=0}^{n} (-1)^k a_k$. Since the sequence $(a_n)_{n=0}^{\infty}$ is decreasing and non-negative, we have
	\begin{align*}
		s_{2n+2} &= s_{2n} \underbrace{- a_{2n+1} + a_{2n+2}}_{\leq 0} \leq s_{2n},\\
		s_{2n+1} &= s_{2n-1} \underbrace{+ a_{2n} - a_{2n+1}}_{\geq 0} \geq s_{2n-1},\\
		s_{2n+2} &= s_{2n+1} \underbrace{+ a_{2n+2}}_{\geq 0} \geq s_{2n+1}
	\end{align*}
	for all $n\in \mathbb{N}$. In other words,
	\[
		s_1 \leq s_3 \leq \hdots \leq s_{2n-1} \leq s_{2n+1} \leq \hdots \leq s_{2n+2} \leq s_{2n} \leq \hdots \leq s_2 \leq s_0.
	\]
	This implies that the sequence $(s_{2n})_{n=0}^{\infty}$ is decreasing and bounded below, while the sequence $(s_{2n+1})_{n=0}^{\infty}$ is increasing and bounded form above. Thus, both limits $A = \lim_{n \to \infty} s_{n+1}$ and $B = \lim_{n \to \infty} s_{2n}$ exist and satify
	\begin{equation}
		\label{eq:leibniz_crit_ineq}
		s_1 \leq s_3 \leq \hdots \leq s_{2n-1} \leq s_{2n+1} \leq A \leq B \leq s_{2n+2} \leq s_{2n} \leq \hdots \leq s_2 \leq s_0.
	\end{equation}
	In particular,
	\[
		0 \leq B - A \leq s_{2n+2} - s_{2n-1} \qquad \forall n \in \mathbb{N},
	\]
	and because $a_{2n+2} \to 0$, we deduce that $A = B$.
	
	Also, Equation \eqref{eq:leibniz_crit_ineq} yields that $s_{2n+1} \leq A = B \leq s_{2n}$, which corresponds exactly to Equation \eqref{eq:leibniz_crit}. \qedhere
\end{proof}

\subsubsection*{Example (Alternating Harmonic Series)} 
The series
\[
	\sum_{n=1}^{\infty} \frac{(-1)^{n+1}}{n} = 1 - \frac{1}{2} + \frac{1}{3} - \frac{1}{4} + \hdots 
\]
converges by Proposition \ref{prop*leibniz_crit}, whereas $\sum_{n=0}^{\infty} \frac{1}{n}$ diverges. Hence, the alternating harmonic series is only conditionally convergent.

\subsection{Absolute Convergence}
In this section we will look at absolutely convergent series and prove some convergence criteria. As before, unless otherwise specified, all sequences consist of real numbers.

\subsubsection{Criteria for Absolute Covergence}
We begin by restating the concept of a Cauchy sequence in the context of convergent series.

\begin{theorem}{Cauchy Criterion for Series}{cauchy_crit_series}
	The series $\sum_{k=0}^{\infty} a_k$ converges if and only if, for every $\varepsilon > 0$, there exists $N \in \mathbb{N}$ such that for all $n > m \geq N$,
	\[
		\left|\sum_{k=m+1}^{n} a_k\right| < \varepsilon.
	\]
\end{theorem}

\begin{proof}
	By definition, the series $\sum_{k=0}^{\infty} a_k$ converges if and only if the sequence of partial sums
	\[
		s_n = \sum_{k=0}^{n} a_k
	\]
	converges. By Theorem \ref{theo*conv_cauchy_seq}, this occurs if and only if $(s_n)_{n=0}^{\infty}$ is a Cauchy sequence, i.e., $|s_n - s_m| < \varepsilon$ for all $n, m \geq N$. Since $s_n - s_m = 0$ when $n=m$, and the expression is symmetric in $n$ and $m$, it suffices to consider the case $n > m$. In this case,
	\[
		s_n - s_m = \sum_{k=m+1}^{n} a_k,
	\]
	which proves the claim.\qedhere
\end{proof}

We can now prove that absolutely convergent series do indeed converge.

\begin{proposition}{Absolute Convergence Implies Convergence}{abs_conv_conv}
	If a series $\sum_{n=0}^{\infty}a_n$ converges absolutely, then it converges and satisfies the generalized triangle inequality
	\[
		\left|\sum_{n=0}^{\infty} a_n\right| \leq \sum_{n=0}^{\infty} |a_n|.
	\]
\end{proposition}

\begin{proof}
	Since $\sum_{n=0}^{\infty} |a_n|$ converges, by the Cauchy criterion (Theorem \ref{theo*cauchy_crit_series}) there exists $N \in \mathbb{N}$ such that, for all $n > m \geq N$,
	\[
	\sum_{k=m+1}^{n} |a_k| < \varepsilon.
	\]
	By the triangle inequality,
	\[
	\left|\sum_{k=m+1}^{n} a_k\right| \leq \sum_{k=m+1}^{n} |a_k| < \varepsilon,
	\]
	so $\sum_{n=0}^{\infty} a_n$ also satisfies the Cauchy criterion and therefore converges.
	
	Moreover, again by the triangle inequality,
	\[
	\left|\sum_{k=0}^{n} a_k\right| \leq \sum_{k=0}^{n} |a_k| \leq \sum_{k=0}^{\infty} a_k \qquad \forall n \in \mathbb{N},
	\]
	and taking the limit as $n \to \infty$ gives the desired inequality.\qedhere
\end{proof}

We now establish two classical criteria guaranteeing absolute convergence. In their proof, we repeatedly use the following fact:
\begin{remark}
	\label{rmk:for_root_crit}
	If a sequence $(x_n)_{n=0}^{\infty}$ converges to $\alpha \in \mathbb{R}$, then Proposition \ref{prop*lim_ineq} implies the following facts:
	\begin{enumerate}
		\item[(i)] for any $q > \alpha$ there exists $N \in \mathbb{N}$ such that $x_n < q$ for all $n \geq N$;
		\item[(ii)] for any $r < \alpha$ there exists $N \in \mathbb{N}$ such that $x_n > r$ for all $n \geq N$.
	\end{enumerate}
\end{remark}

\begin{proposition}{Cauchy Root Criterion}{cauchy_root_crit}
	Given a sequence $(a_n)_{n=0}^{\infty}$, define
	\[
		\alpha = \limsup_{n \to \infty} \sqrt[n]{|a_n|} \in \mathbb{R} \cup \{\infty\}.
	\]
	Then,
	\[
		\alpha < 1 \;\Rightarrow \; \sum_{n=0}^{\infty} a_n \text{ converges absolutely,} \qquad \alpha > 1 \;\Rightarrow\; \sum_{n=0}^{\infty} a_n \text{ does not converge}.
	\]
\end{proposition}

\begin{proof}
	Suppose $\alpha < 1$ and set $q = \frac{1 + \alpha}{2}$, so that $q \in (\alpha, 1)$. By definition,
	\[
	\limsup_{n \to \infty} \sqrt[n]{|a_n|} = \lim_{n \to \infty} \sup_{k\geq n} \sqrt[k]{|a_k|}.
	\]
	Thus, $x_n = \sup_{k\geq n} \sqrt[k]{|a_k|} \to \alpha$. Since $\alpha < q$, Remark \ref{rmk:for_root_crit}(i) implies the existence of $N \in \mathbb{N}$ such that
	\[
	x_N = \sup_{k\geq N} \sqrt[k]{|a_k|} < q \qquad \forall k \geq N,
	\]
	therefore,
	\[
	|a_k| < q^k \qquad \forall k \geq N.
	\]
	Since $q < 1$, $\sum_{k=N}^{\infty} |a_k|$ converges by comparison with the geometric series, so $\sum_{n=0}^{\infty} a_n$ converges absolutely.
	
	If $\alpha > 1$, since the limsup is an accumulation point (Theorem \ref{theo*limsup_acc_pt}), Proposition \ref{prop*subseq_acc_pt} implies the existence of a subsequence $(a_{n_k})_{k=0}^{\infty}$ such that $\lim_{k \to \infty} \sqrt[n_k]{|a_{n_k}|} = \alpha$. Hence, thanks to Remark \ref{rmk:for_root_crit}(ii) with $r=1$, $\sqrt[n_k]{|a_{n_k}|} > 1$ for all $k$ large, or equivalently, $|a_{n_k}| > 1$ for large $k$. In particular the sequence $(a_n)_{n=0}^{\infty}$ does not converge to 0. Recalling Proposition \ref{prop*cond_for_conv}, this implies that the series $\sum_{n=0}^{\infty} a_n$ does not converge.\qedhere
\end{proof}

\begin{remark}
	If $\alpha = 1$, the root criterion is inconclusive.
\end{remark}

\begin{proposition}{D'Alambert's Quotient Criterion}{quot_crit}
	Given a sequence $(a_n)_{n=0}^{\infty}$ with $a_n \neq 0$ for all $n$, assume that
	\[
		\lim_{n \to \infty} \frac{|a_{n+1}|}{|a_n|} = \alpha \in [0, \infty).
	\]
	Then
	\[
		\alpha < 1 \; \Rightarrow \; \sum_{n=0}^{\infty} a_n \text{ converges absolutely,} \qquad \alpha > 1 \;\Rightarrow\; \sum_{n=0}^{\infty} a_n \text{ does not converge.}
	\]
\end{proposition}

\begin{proof}
	The proof parallels that of the root criterion.
	
	If $\alpha < 1$. set $q = \frac{1 + \alpha}{2} \in (\alpha, 1)$. Since $\frac{|a_{n+1}|}{|a_n|} \to \alpha$ and $\alpha < q$, by Remark \ref{rmk:for_root_crit}(i) there exists $N \in \mathbb{N}$ such that
	\[
		\frac{|a_{k+1}|}{|a_k|} < q \qquad \forall k \geq N.
	\]
	This gives
	\[
		|a_k| = \frac{|a_k|}{|a_{k-1}|}\cdot \frac{|a_{k-1}|}{|a_{k-2}|} \cdot \hdots \cdot \frac{|a_{N+1}|}{|a_N|} \cdot |a_{N}|
		<
		q^{k-N} |a_N| = \frac{|a_N|}{q^N} q^k \qquad \forall k \geq N.
	\]
	Since $q < 1$, the geometric comparison test shows that $\sum_{n=0}^{\infty} a_n$ converges absolutely.
	
	If $\alpha > 1$, then Remark \ref{rmk:for_root_crit}(ii) with $r = 1$ implies the existence of $N \in \mathbb{N}$ such that
	\[
		\frac{|a_{k+1}|}{|a_k|} > 1 \qquad \forall k \geq N.
	\]
	In particular,
	\[
		|a_k| = \frac{|a_k|}{|a_{k-1}|}\cdot \frac{|a_{k-1}|}{|a_{k-2}|} \cdot \hdots \cdot \frac{|a_{N+1}|}{|a_N|} \cdot |a_{N}|
		> |a_N| \qquad \forall k \geq N.
	\]
	Hence, $(a_n)_{n=0}^{\infty}$ does not tend to 0, and by Proposition \ref{prop*cond_for_conv} the series does not converge.\qedhere
\end{proof}

\subsubsection{Reordering Series}
\begin{theorem}{Rearrangement of Absolutely Convergent Series}{rearr_abs_conv}
	Let $\sum_{n=0}^{\infty} a_n$ be an absolutely convergent series, and let $\varphi: \mathbb{N} \to \mathbb{N}$ be a bijection. Then $\sum_{n=0}^{\infty} a_{\varphi(n)}$ is absolutely convergent, and
	\begin{equation}
		\label{eq:reordered_series}
		\sum_{n=0}^{\infty} a_n = \sum_{n=0}^{\infty} a_{\varphi(n)}.
	\end{equation}
\end{theorem}

\begin{proof}
	Fix $\varepsilon > 0$. Since $\sum_{n=0}^{\infty} |a_n|$ converges, there exists $N \in \mathbb{N}$ such that
	\[
		\sum_{k=N+1}^{\infty} |a_k| < \frac{\varepsilon}{2}.
	\]
	Let
	\[
		M = \max\{\varphi^{-1}(0), \hdots , \varphi^{-1}(N)\}.
	\]
	Equivalently, $M \in \mathbb{N}$ is the smallest number such that
	\[
		\{a_0, \hdots , a_N\} \subseteq \{a_{\varphi(0)}, \hdots , a_{\varphi(M)}\}
	\]
	Then
	\[
		\{a_0, \hdots , a_N\} \subseteq \{a_{\varphi(0)}, \hdots , a_{\varphi(n)}\} \qquad \forall n \geq M.
	\]
	Therefore,
	\[
		\sum_{\ell = 0}^{n} a_{\varphi(\ell)} - \sum_{k=0}^{N} a_k = \sum_{\underset{\varphi(\ell) > N}{0 \leq \ell \leq n}} a_{\varphi(\ell)}.
	\]
	Moreover, since all indices $\varphi(\ell) > N$ with $0\leq \ell \leq n$ correspond to terms among $\{|a_k| \;|\; k \geq N+1\}$, we have
	\[
	\sum_{\underset{\varphi(\ell) >N}{0 \leq \ell \leq n}} |a_{\varphi(\ell)}| \leq \sum_{k = N+1}^{\infty} |a_k|.
	\]
	This implies that, for $n \geq M$, we can estimate
	\begin{align*}
		\bigg|\sum_{\ell = 0}^{n} a_{\varphi(\ell)} - \sum_{k=0}^{\infty} a_k\bigg| &= \bigg|\sum_{\ell = 0}^{n} a_{\varphi(\ell)} - \sum_{k=0}^{N} a_k - \sum_{k=N+1}^{\infty} a_k\bigg| = \bigg|\sum_{\underset{\varphi(\ell) > N}{0 \leq \ell \leq n}} a_{\varphi(\ell)} - \sum_{k=N+1}^{\infty} a_k\bigg|\\
		&\leq \sum_{\underset{\varphi(\ell) > N}{0 \leq \ell \leq n}} |a_{\varphi(\ell)}| + \sum_{k=N+1}^{\infty} |a_k| \leq 2\,\sum_{k=N+1}^{\infty} |a_k| < \varepsilon.
	\end{align*}
	This shows that
	\[
	\sum_{\underset{\varphi(\ell) > N}{0 \leq \ell \leq n}} a_{\varphi(\ell)} \longrightarrow \sum_{k=0}^{\infty} a_k \quad \text{as } n\to \infty,
	\]
	which proves the identity \eqref{eq:reordered_series}. Applying the same reasoning to $\sum_{n=0}^{\infty} |a_n|$ shows that $\sum_{\ell = 0}^{\infty} |a_{\varphi(\ell)}| = \sum_{n=0}^{\infty} |a_n| < \infty$, hence $\sum_{\ell = 0}^{\infty} a_{\varphi(\ell)}$ is absolutely convergent.
	\qedhere
\end{proof}

\subsubsection{Product of Series}

\begin{theorem}{Product Theorem}{prod_theo}
	Let $\sum_{n=0}^{\infty} a_n$ and $\sum_{n=0}^{\infty} b_n$ be absolutely convergent series, and let $\alpha: \mathbb{N} \to \mathbb{N} \times \mathbb{N}$ be a bijection. Writing $\alpha(n) = (\alpha_1(n), \alpha_2(n))$, one has
	\begin{equation}
		\label{eq:product_series}
		\left(\sum_{n=0}^{\infty} a_n\right)\left(\sum_{n=0}^{\infty} b_n\right) = \sum_{n=0}^{\infty} a_{\alpha_1(n)} b_{\alpha_2(n)},
	\end{equation}
	and the series on the right converges absolutely.
\end{theorem}

\begin{proof}
	Consider first a bijection $\alpha: \mathbb{N} \to \mathbb{N} \times \mathbb{N}$, written as $\alpha(n) = (\alpha_1(n), \alpha_2(n))$, such that
	\[
		\{a(k) \;|\; 0\leq k < n^2\} = \{0, 1, \hdots, n-1\} \qquad \forall n \in\mathbb{N}. 
	\]
	For example, $(\alpha(n))_{n=0}^{\infty}$ could traverse the grid as shown in the lecture.
	Then, for every $n \in \mathbb{N}$,
	\[
	\sum_{k=0}^{n^2 - 1} |a_{\alpha_1(k)}| |b_{\alpha_2(k)}| = \left(\sum_{\ell = 0}^{n-1} |a_{\ell}|\right) \left(\sum_{m=0}^{n-1} |b_m|\right).
	\]
	Since the right-hand side is bounded by
	\[
	\left(\sum_{\ell = 0}^{\infty} |a_{\ell}|\right) \left(\sum_{m=0}^{\infty} |b_m|\right),
	\]
	we have
	\[
	\sup_{n \in \mathbb{N}} \sum_{k=0}^{n^2 - 1} |a_{\alpha_1(k)}| |b_{\alpha_2(k)}| \leq 	\left(\sum_{\ell = 0}^{\infty} |a_{\ell}|\right) \left(\sum_{m=0}^{\infty} |b_m|\right) < \infty.
	\]
	This implies that the series $\sum_{k=0}^{\infty} a_{\alpha_1(k)} b_{\alpha_2(k)}$. In particular, since it converges, its value can be computed along every subsequence, therefore
	\[
	\sum_{k=0}^{\infty} a_{\alpha_1(k)} b_{\alpha_2(k)} = \lim_{n \to \infty} \sum_{k=0}^{n^2 - 1} a_{\alpha_1(k)} b_{\alpha_2(k)}
	\]
	Now writing the identity
	\[
	\sum_{k=0}^{n^2 - 1} a_{\alpha_1(k)} b_{\alpha_2(k)} = \left(\sum_{\ell = 0}^{n-1} a_{\ell}\right) \left(\sum_{m=0}^{n-1} b_m\right),
	\]
	and taking the limit as $n \to \infty$, Proposition \ref{prop*lim_op}(2.) gives
	\begin{align*}
		\sum_{k=0}^{\infty} a_{\alpha_1(k)} b_{\alpha_2(k)} &= \lim_{n \to \infty} \sum_{k=0}^{n^2 - 1} a_{\alpha_1(k)} b_{\alpha_2(k)} = \\
		&= \left( \lim_{n \to \infty}\sum_{\ell = 0}^{n-1} a_{\ell}\right) \left( \lim_{n \to \infty} \sum_{m=0}^{n-1} b_m\right) = \left(\sum_{\ell = 0}^{\infty} a_{\ell}\right) \left(\sum_{m=0}^{\infty} b_m\right),
	\end{align*}
	which proves Equation \eqref{eq:product_series} for this specific bijection $\alpha$.
	
	For an arbitrary bijection $\beta: \mathbb{N} \to \mathbb{N} \times \mathbb{N}$, define $\varphi = \alpha^{-1} \circ \beta: \mathbb{N} \to \mathbb{N}$. Then $\beta = \alpha \circ \varphi$ and writing $\beta(n) = (\beta_1(n), \beta_2(n)) = (\alpha_1(\varphi(n)), \alpha_2(\varphi(n)))$, the rearrangement Theorem \ref{theo*rearr_abs_conv} yields
	\[
	\sum_{n=0}^{\infty} a_{\beta_1(n)} b_{\beta_2(n)} = \sum_{n=0}^{\infty} a_{\alpha_1(\varphi(n))} b_{\alpha_2(\varphi(n))} = \sum_{n=0}^{\infty} a_{\alpha_1(n)} b_{\alpha_2(n)} = \left(\sum_{n=0}^{\infty} a_n\right) \left(\sum_{n=0}^{\infty} b_n\right). \qedhere
	\] 
\end{proof}

\begin{corollary}{Cauchy Product}{cauchy_prod}
	If $\sum_{n=0}^{\infty} a_n$ and $\sum_{n=0}^{\infty} b_n$ are absolutely convergent, then
	\[
		\left(\sum_{n=0}^{\infty} a_n\right) \left(\sum_{n=0}^{\infty} b_n\right) = \sum_{n = 0}^{\infty} \left(\sum_{k=0}^{n} a_{n-k} b_k\right),
	\]
	and the series on the right converges absolutely.
\end{corollary}

\begin{proof}
	Consider the bijection $\alpha: \mathbb{N} \to \mathbb{N} \times \mathbb{N}$ defined as
	\begin{align*}
		&\alpha(0) = (0,0), \quad \alpha(1) = (1,0), \quad \alpha(2) = (0,1), \quad \alpha(3) = (2,0), \quad \alpha(4) = (1,1), \hdots ,\\
		&\alpha(20) = (0, 5), \hdots , \alpha(31) = (4, 3), \hdots , \alpha(49) = (5,4), \hdots \text{etc.}
	\end{align*}
	By Theorem \ref{theo*prod_theo},
	\[
		\left(\sum_{n=0}^{\infty} a_n\right) \left(\sum_{n=0}^{\infty} b_n\right) = \sum_{n=0}^{\infty} a_{\alpha_1(n)} b_{\alpha_2(n)}.
	\]
	Listing the terms explicitly and grouping them by diagonals, we obtain
	\begin{align*}
				\sum_{n=0}^{\infty} a_{\alpha_1(n)} b_{\alpha_2(n)} &= a_0b_0 + (a_0 b_1 + a_1 b_0) + (a_2 b_0 + a_1 b_1 + a_0 b_2)\\
				&+ (a_3 b_0 + a_2 b_1 + a_1 b_2 + a_0 b_3) + \hdots\\
				&= \sum_{n=0}^{\infty} \bigg(\sum_{\underset{j + k = n}{j,k \geq 0}} a_j b_k\bigg) = \sum_{n=0}^{\infty} \bigg(\sum_{k=0}^{n} a_{n-k} b_k\bigg).
	\end{align*}
	Finally, absolute convergence follows from the triangle inequality and Theorem \ref{theo*prod_theo}, i.e.,
	\[
		\sum_{n=0}^{\infty} \bigg|\sum_{k=0}^{n} a_{n-k} b_k\bigg| \leq \sum_{n=0}^{\infty} \sum_{k=0}^{n} |a_{n-k} b_k| = \sum_{n=0}^{\infty} |a_{\alpha_1(n)}| |b_{\alpha_2(n)}| < \infty. \qedhere
	\]
\end{proof}

\subsection{Series of Complex Numbers}
To define the notion of a convergent series in $\mathbb{C}$, it is sufficient to consider separately the corresponding series of its real and imaginary parts in $\mathbb{R}$.

\begin{definition}{Series of Complex Numbers}{series_cmplx}
	Let $(z_n)_{n=0}^{\infty} = (x_n + iy_n)_{n=0}^{\infty}$ be a sequence of complex numbers, and let $Z = A + iB \in \mathbb{C}$. The series $\sum_{n=0}^{\infty} z_n$ is said to \textbf{converge} to $Z$ if both real series $\sum_{n=0}^{\infty} x_n$ and $\sum_{n=0}^{\infty} y_n$ converge, with limits $A$ and $B$, respectively, i.e.,
	\[
		\sum_{n=0}^{\infty} x_n = A, \qquad \sum_{n=0}^{\infty} y_n = B.
	\]
	We say that
	$\sum_{n=0}^{\infty} z_n$ \textbf{converges absolutely} if the series of moduli $\sum_{n=0}^{\infty} |z_n|$ converges.
\end{definition}

Whenever the series $\sum_{n=0}^{\infty} z_n$ and $\sum_{n=0}^{\infty} w_n$ converge absolutely, their sum and product are given (exactly as in the real case) by
\begin{subequations}
	\begin{align}
		\label{eq:sum_cmplx_series}
		\sum_{n=0}^{\infty} z_n + \sum_{n=0}^{\infty} w_n &= \sum_{n=0}^{\infty} z_n + w_n\\
		\label{eq:prod_cmplx_series}
		\left(\sum_{n=0}^{\infty} z_n\right) \left(\sum_{n=0}^{\infty} w_n\right) &= \sum_{n=0}^{\infty} \left(\sum_{k=0}^{n} z_{n-k} w_k\right).
	\end{align}
\end{subequations}

\begin{remark}
	\label{rmk:conv_real_im}
	Let $(z_n)_{n=0}^{\infty} = (x_n + iy_n)_{n=0}^{\infty}$ be a sequence of complex numbers, and assume that the series $\sum_{n=0}^{\infty} |z_n|$ converges. Since
	\[
		0\leq |x_n| \leq |z_n|, \qquad 0 \leq |y_n| \leq |z_n|, \qquad \forall n\in \mathbb{N},
	\]
	the Majorant Criterion (Corollary \ref{cor*maj_min}) implies that both $\sum_{n=0}^{\infty} |x_n|$ and $\sum_{n=0}^{\infty} |y_n|$ converge. Hence, the series of real and imaginary parts are absolutely convergent.
	
	Conversely since $|z_n| \leq |x_n| + |y_n|$, the absolute convergence of $\sum_{n=0}^{\infty} x_n$ and $\sum_{n=0}^{\infty} y_n$ also implies the absolute convergence of $\sum_{n=0}^{\infty} z_n$. Therefore, absolute convergence in $\mathbb{C}$ is equivalent to absolute convergence of the real and imaginary parts.
\end{remark}

\subsection{Power Series}
\label{sec:pwr_series}
Our next goal is to investigate power series. These are series where the terms are powers of the variable $x \in \mathbb{R}$ (or $z \in \mathbb{C}$, if one considers complex power series) multiplied by coefficients.

\subsubsection{Radius of Convergence}

\begin{definition}{Power Series}{pwr_series}
	A \textbf{power series} with real coefficients is a series of the form
	\[
		\sum_{n=0}^{\infty} a_n x^n,
	\]
	where $(a_n)_{n=0}^{\infty}$ is a sequence in $\mathbb{R}$ and $x \in \mathbb{R}$. Here $x$ is the \textbf{variable} and $a_n$ is the \textbf{coefficient} of $x^n$.
	By convention, we set $x^0 = 1$ for all $x \in \mathbb{R}$, including $x=0$. In other words, the first term of the power series is always $a_0$.
\end{definition}
Addition and Multiplication of power series are given by
\begin{align*}
		\sum_{n=0}^{\infty}a_n x^n + \sum_{n=0}^{\infty} b_n x^n &= \sum_{n=0}^{\infty} (a_n + b_n) x^n,\\
		\left(\sum_{n=0}^{\infty} a_n x^n\right) \left(\sum_{n=0}^{\infty} b_n x^n\right) &= \sum_{n=0}^{\infty} \left(\sum_{k=0}^{n} a_{n-k} x^{n-k} b_k x^k\right) = \sum_{n=0}^{\infty} \left(\sum_{k=0}^{\infty} a_{n-k} b_k\right)x^n,
\end{align*}
where the product formula follows from Corollary \ref{cor*cauchy_prod}.

A power series is a polynomial whenever only finitely many coefficients are \textit{non-zero}.

\begin{definition}{Radius of Convergence}{radius_conv}
	Let $\sum_{n=0}^{\infty} a_n x^n$ be a power series, and define
	\[
		\rho := \limsup_{n \to \infty} \sqrt[n]{|a_n|}.
	\]
	The \textbf{radius of convergence} is defined as 
	\[
		R := \begin{cases}
			0, \qquad &\rho = \infty,\\
			\rho^{-1}, \qquad &0  < \rho < \infty,\\
			\infty, \qquad &\rho = 0.
		\end{cases}
	\]
\end{definition}

In the following, when we write $R \in [0, \infty]$, we mean that $R$ is either a non-negative real number or $R = \infty$.

\begin{theorem}{Convergence of Power Series}{conv_pwr_series}
	Let $\sum_{n=0}^{\infty}a_n x^n$ be a power series with radius of convergence $R \in (0, \infty]$. Then $\sum_{n=0}^{\infty}a_nx^n$ converges absolutely for all $x \in \mathbb{R}$ with $|x| < R$, and does not converge for all $x \in \mathbb{R}$ with $|x| > R$. In particular, for $x \in (-R, R)$, we can define the function $f(x) = \sum_{n=0}^{\infty} a_n x^n$.
\end{theorem}

\begin{proof}
	Let $x \in \mathbb{R}$, and write $\rho = \limsup_{n \to \infty} \sqrt[n]{|a_n|}$ as in Definition \ref{def*radius_conv}. Then
	\[
		\limsup_{n \to \infty} \sqrt[n]{|a_n x^n|} = \left(\limsup_{n \to \infty} \sqrt[n]{|a_n|}\right) |x| = \rho |x|.
	\]
	By the root criterion (see Proposition \ref{prop*cauchy_root_crit}), the series $\sum_{n=0}^{\infty}a_n x^n$ converges absolutely if$\rho |x| < 1$, and does not converge if $\rho |x| > 1$ (in particular if $\rho = 0$ it converges for all $x \in \mathbb{R}$). Since $R = \frac{1}{\rho}$, the result follows. \qedhere
\end{proof}

\begin{theorem}{Continuity of Power Series}{cont_pwr_series}
	Let $\sum_{n=0}^{\infty} a_n x^n$ be a power series with radius of convergence $R \in (0, \infty]$, and define polynomials $f_n(x) = \sum_{k=0}^{n} a_k x^k$. For any $r \in (0, R)$, the sequence $(f_n)_{n=0}^{\infty}$ converges uniformly to $f$ on $[-r, r]$. In particular, the power series defines a continuous function $f:(-R, R) \to \mathbb{R}$.
\end{theorem}

\begin{proof}
	By Theorem \ref{theo*conv_pwr_series} with $x = r$, the series $\sum_{n=0}^{\infty} |a_n| r^n$ converges, where $r < R$. Hence, for every $\varepsilon > 0$, there exists $N \in \mathbb{N}$ such that
	\[
	\sum_{k=N+1}^{\infty} |a_k| r^k < \varepsilon.
	\]
	Thus, for all $x \in [-r, r]$ and all $n \geq N$,
	\[
	|f_n(x) - f(x)| = \left|\sum_{k=n+1}^{\infty} a_k x^k\right| \leq \sum_{k=n+1}^{\infty} |a_k||x^k| \leq \sum_{k=N+1}^{\infty} |a_k| r^k < \varepsilon.
	\]
	This shows that $(f_n)_{n=0}^{\infty}$ converges uniformly to $f$ on $[-r, r]$. Since each $f_n$ is continuous (being a polynomial), Theorem \ref{theo*cont_unif_conv} implies that $f$ is continuous on $[-r, r]$. As $r < R$ is arbitrary, $f$ is continuous on $(-R, R)$.\qedhere
\end{proof}

\subsubsection*{Example}
In general, the partial sums $f_n(x) = \sum_{k=0}^{n} a_k x^k$ do \textit{not} converge uniformly to $f(x) = \sum_{k=0}^{\infty} a_k x^k$ on the whole interval $(-R, R)$.

To see this, consider the geometric series $\sum_{n=0}^{\infty} x^n$. Its radius of Convergence is $R = 1$, and on $(-1, 1)$ we have $f(x) = \sum_{n=0}^{\infty} x^n = \frac{1}{1 - x}$. If the convergence on $(-1, 1)$ were uniformly, then applying the notion of uniform convergence with $\varepsilon = 1$ would give $N \in \mathbb{N}$ such that, for all $n \geq N$ and $x \in (-1, 1)$,
\[
	\left|\sum_{k=0}^{n} x^k - \frac{1}{1 - x}\right| < 1.
\]
Taking $n = N$ and using the triangle inequality, we would get
\[
	\left|\frac{1}{1 - x}\right| < 1 + \left|\sum_{k=0}^{N} x^k\right| \leq 1 + \sum_{k=0}^{N} |x^k| \leq 1 + (N + 1) = N + 2 \qquad \forall x \in (-1, 1),
\]
a contradiction since $\lim_{x\to 1^{-}} \frac{1}{1 - x} = \infty$.


\begin{proposition}{Radius of Convergence of Sum and Product}{radius_conv_sum_prod}
	Let $R \geq 0$, and let $\sum_{n=0}^{\infty} a_n x^n$ and $\sum_{n=0}^{\infty} b_n x^n$ be power series with radius of convergence of at least $R$. Then their sum and their Cauchy product also have radius of convergence of at least $R$.
\end{proposition}

\begin{proof}
	By linearity and Corollary \ref{cor*cauchy_prod}, the absolute convergence of $\sum_{n=0}^{\infty} a_nx^n$ and $\sum_{n=0}^{\infty} b_nx^n$ for $|x| < R$ implies that
	\[
		\sum_{n=0}^{\infty} (a_n + b_n) x^n, \qquad \left(\sum_{n=0}^{\infty} a_n x^n\right) \left(\sum_{n=0}^{\infty} b_n x^n\right) = \sum_{n=0}^{\infty} \left(\sum_{k=0}^{n} a_{n-k} b_k\right) x^n
	\]
	both converge absolutely for $|x| < R$. Since a power series cannot converge for $|x| > R$, each has radius of convergence of at least $R$. \qedhere
\end{proof}

\subsubsection{Complex Power Series}

\begin{definition}{Complex Power Series}
	A \textbf{complex power series} with complex coefficients is a series of the form
	\[
		\sum_{n=0}^{\infty} a_n z^n,
	\]
	where $(a_n)_{n=0}^{\infty}$ is a sequence in $\mathbb{C}$ and $z \in \mathbb{C}$. Again by convention, we set $z^0 = 1$ for all $z \in \mathbb{C}$, including $z = 0$.
\end{definition}
Addition and multiplication are defined as in the real case.

\begin{theorem}{Convergence of Complex Power Series}{conv_cmplx_pwr_series}
	Let $\sum_{n=0}^{\infty} a_nz^n$ be a complex power series with radius of convergence $R \in (0, \infty]$. Then the series $\sum_{n=0}^{\infty} a_nz^n$ converges absolutely for all $z \in \mathbb{C}$ with $|z| < R$, and diverges for all $z \in \mathbb{C}$ with $|z| > R$. In particular, for $|z| < R$ one can define the (complex-valued) function
	\[
		f(z) = \sum_{n=0}^{\infty} a_n z^n.
	\]
\end{theorem}

\subsection{Exponential and Trigonometric Functions}

\subsubsection{The Exponential Map as a Power Series}
We now show that the exponential map can also be defined via the \textbf{exponential series}
\begin{equation}
	\label{eq:exp}
	\exp(x) = e^x = \sum_{n=0}^{\infty} \frac{x^n}{n!}.
\end{equation}
Since $\frac{n!}{(n+1)!} = \frac{1}{(n+1)} \to 0$ as $n \to \infty$, it follows directly from the quotient criterion that this series has infinite radius of convergence. Hence, Theorem \ref{theo*cont_pwr_series} implies that the right-hand side of Equation \eqref{eq:exp} defines a continuous function on $\mathbb{R}$.

\begin{proposition}{Exponential Map as a Power Series}{exp_pwr_series}
	For every $x \in \mathbb{R}$,
	\[
		\sum_{k=0}^{\infty} \frac{x^k}{k!} = \lim_{n \to \infty} \left(1 + \frac{x}{n}\right)^n.
	\]
\end{proposition}

\begin{definition}{The Complex Exponential Map}{cmplx_exp}
	The \textbf{complex exponential map} is the function $\exp:\mathbb{C} \to \mathbb{C}$ defined by
	\[
		\exp(z) = e^z = \sum_{n=0}^{\infty} \frac{z^n}{n!}, \qquad z \in \mathbb{C}.
	\]
\end{definition}

\begin{theorem}{Properties of the Complex Exponential}{prop_cmplx_exp}
	The complex exponential map $\exp: \mathbb{C} \to \mathbb{C}$ is continuous, and for all $z, w \in \mathbb{C}$,
	\[
		e^{z + w} = e^z e^w, \quad |e^z| = e^{\operatorname{Re}(z)}.
	\]
	In particular, $|e^{ix}| = 1$ for all $x \in \mathbb{R}$.
\end{theorem}

\subsubsection{Sine and Cosine}
Given $x \in \mathbb{R}$, we split the power series of $e^{ix}$ into its even and odd terms, i.e.,
\[
	e^{ix} = \sum_{n=0}^{\infty} \frac{i^n}{n!}x^n = \sum_{n = 0}^{\infty} \frac{i^{2n}}{(2n)!} x^{2n} + \sum_{n=0}^{\infty} \frac{i^{2n+1}}{(2n+1)!}x^{2n+1}.
\]
Since $i^{2n} = (-1)^n$ and $i^{2n+1} = i (-1)^n$, we obtain
\[
	e^{ix} = \sum_{n=0}^{\infty} \frac{(-1)^n}{(2n)!}x^{2n} + i \sum_{n=0}^{\infty} \frac{(-1)^{n}}{(2n + 1)!} x^{2n+1}.
\]
This motivates the following definition of the \textbf{sine} and \textbf{cosine functions}, i.e.,
\begin{equation}
	\label{eq:sin_cos}
	\sin(x) := \sum_{n=0}^{\infty} \frac{(-1)^n}{(2n + 1)!}x^{2n+1}, \qquad \cos(x) := \sum_{n=0}^{\infty} \frac{(-1)^n}{(2n)!}x^{2n},
\end{equation}
so that the identity
\[
	e^{ix} = \cos(x) + i\sin(x)
\]
holds for all $x \in \mathbb{R}$.

As for the exponential series, radius of convergence of the power series in Equation \eqref{eq:sin_cos} is infinite. Therefore, by Theorems \ref{theo*conv_pwr_series} and \ref{theo*cont_pwr_series}, $\sin$ and $\cos$ are continuous on $\mathbb{R}$.

Since $(-x)^{2n+1} = -x^{2n+1}$ and $(-x)^{2n} = x^{2n}$ for all $n \in \mathbb{N}$, it follows directly from Equation \eqref{eq:sin_cos} that
\begin{align*}
	\sin(-x) &= -\sin(x) \text{ $\sin$ is an \textbf{odd} function},\\
	 \cos(-x) &= \cos(x) \text{ $\cos$ is an \textbf{even} function}.
\end{align*}

\begin{theorem}{From the Complex Exponential to Sine and Cosine}{cmplx_exp_sin_cos}
	For all $x \in \mathbb{R}$, the relations
	\[
		e^{ix} = \cos(x) + i\sin(x), \qquad \sin(x) = \frac{e^{ix} - e^{-ix}}{2i}, \qquad \cos(x) = \frac{e^{ix} + e^{-ix}}{2} 
	\]
	hold. For all $x, y \in \mathbb{R}$ the trigonometric addition formulas are
	\begin{align*}
		\sin(x + y) &= \sin(x)\cos(y) + \cos(x)\sin(y),\\
		 \cos(x + y) &= \cos(x)\cos(y) - \sin(x)\sin(y).
	\end{align*}
\end{theorem}

\subsubsection{The Circle Number}
\begin{theorem}{Existence of $\pi$ as the First Zero of Sine}{exist_pi_sin}
	There exists exactly one number $\pi \in (0, 4)$ such that $\sin(\pi) = 0$.
	For this number it holds that
	\[
		e^{i\frac{\pi}{2}} = i, \qquad e^{i\pi} = -1, \qquad e^{i2\pi} = 1.
	\]
\end{theorem}

\begin{corollary}{Periodicity of Sine and Cosine}{period_sin_cos}
	\begin{align*}
		\sin(x + \frac{\pi}{2}) &= \cos(x), \qquad \cos(x + \frac{\pi}{2}) = -\sin(x)\\
		\sin(x + \pi) &= -\sin(x), \qquad \cos(x + \pi) = -\cos(x),\\
		\sin(x + 2\pi) &= \sin(x), \qquad \cos(x + 2\pi) = \cos(x).
	\end{align*}
\end{corollary}

\subsubsection{Polar Coordinates and Multiplication of Complex Numbers}
Using the complex exponential function, we can express complex numbers in \textbf{polar coordinates}, i.e., in the form
\[
	z = re^{i\theta} = r \cos(\theta) + ir \sin(\theta),
\]
where $r = |z|$ is the distance of $z$ from the origin, and $\theta$ is the angle between the positive real axis $\mathbb{R}_{\geq0}$ and the segment from 0 to $z$. In other words, if $z = x + iy$, then
\[
	x = r\cos(\theta), \qquad y = r\sin(\theta), \qquad r = \sqrt{x^2 + y^2}.
\]
If $z \neq 0$, the angle $\theta$ is uniquely determined and is called the \textbf{argument} of $z$, denoted $\theta = \arg(z)$. The set of all complex numbers with absolute value 1 is
\[
	\mathbb{S}^1 = \{z \in \mathbb{C} \;|\; |z| = 1\} = \{e^{i\theta} \;|\; \theta \in [0, 2\pi)\},
\]
and is called the \textbf{unit circle} in $\mathbb{C}$.

\begin{proposition}{Existence of Polar Coordinates}{exist_polar_coord}
	For every $z \in \mathbb{C} \setminus \{0\}$, there exists uniquely determined real numbers $r > 0$ and $\theta \in [0, 2\pi)$ such that $z = r e^{i\theta}$.
\end{proposition}

In polar coordinates, multiplication of complex numbers takes a simple geometric form: if $z = re^{i\varphi}$ and $w = se^{i\psi}$, then
\[
	zw = rse^{i(\varphi + \psi)}.
\]
Thus, when multiplying two complex numbers, their magnitudes multiply and their arguments add.

\subsubsection{Other Trigonometric and Hyperbolic Functions}
In addition to sine and cosine functions, several related \textbf{trigonometric functions} are defined.

The \textbf{tangent} and \textbf{cotangent} functions are defined by
\[
	\tan(x) = \frac{\sin(x)}{\cos(x)}, \qquad \cot(x) = \frac{\cos(x)}{\sin(x)},
\]
for all $x \in \mathbb{R}$ such that the denominator are non-zero.

The \textbf{hyperbolic sine} and \textbf{hyperbolic cosine} are defined by the power series
\[
	\sinh(x) = \sum_{n=0}^{\infty} \frac{x^{2n+1}}{(2n+1)!}, \qquad \cosh(x) = \sum_{n=0}^{\infty} \frac{x^{2n}}{(2n)!}.
\]
Equivalently,
\[
	\sinh(x) = \frac{e^{ix} - e^{-ix}}{2}, \qquad \cosh(x) = \frac{e^{ix} + e^{-ix}}{2},
\]
and hence $e^{x} = \sinh(x) + \cosh(x)$ for all $x \in \mathbb{R}$.

The \textbf{hyperbolic tangent} and \textbf{hyperbolic cotangent} are defined by
\[
	\tanh(x) = \frac{\sinh(x)}{\cosh(x)} = \frac{e^{ix} - e^{-ix}}{e^{ix} + e^{-ix}}, \qquad \coth(x) = \frac{\cosh(x)}{\sinh(x)} = \frac{e^{ix} + e^{-ix}}{e^{ix} - e^{-ix}},
\]
where $\coth(x)$ is defined for all $x \in \mathbb{R}\setminus \{0\}$ (since $\sinh(x) \neq 0$ for all $x \neq 0$)

The functions $\sinh$ and $\tanh$ are odd, while $\cosh$ is even. Also, they satisfy the addition formulas
\begin{align*}
	\sinh(x+y) &= \sinh(x)  \cosh(y) + \cosh(x) \sinh(y),\\
	\cosh(x+y) &= \cosh(x) \cosh(y) + \sinh(x) \sinh(y),
\end{align*}
and the \textbf{hyperbolic identity}
\[
	\cosh^2(x) - \sinh^2(x) = 1 \qquad \forall x \in \mathbb{R}.
\]



	%
% (c) 2025 Autor, ETH Zürich
%
% !TEX root = main.tex
% !TEX encoding = UTF-8
%

\section{Differential Calculus}
In this chapter we deal with differential calculus in one variable. This is of fundamental importance for understanding functions on $\mathbb{R}$.

\subsection{The Derivative}

\subsubsection{Definition and Geometrical Interpretation}
In this section $D \subseteq \mathbb{R}$ denotes a non-empty set with no isolated points, i.e., every $x \in D$ is an accumulation point of $D \setminus \{x\}$. A typical example is a non-empty interval containing more than one point.

\begin{definition}{Derivative}{derivative}
	Let $f:D \to \mathbb{R}$ be a function and $x_0 \in D$. We say that $f$ is \textbf{differentiable} at $x_0$ if the limit
	\begin{equation}
		\label{eq:diff_quot}
		f'(x_0) = \lim_{\underset{x \neq x_0}{x \to x_0}} \frac{f(x) - f(x_0)}{x - x_0} = \lim_{\underset{h \neq 0}{h \to 0}} \frac{f(x_0 + h) - f(x_0)}{h}
	\end{equation}
	exists. In this case we call $f'(x_0)$ the \textbf{derivative} of $f$ at $x_0$. If $f$ is differentiable at every point of $D$, we say that $f$ is \textbf{differentiable} on $D$, and we call the resulting function $f':D \to \mathbb{R}$ the \textbf{derivative} of $f$.
\end{definition}

To simplify notation, we will often write
\[
	f'(x_0) = \lim_{x \to x_0} \frac{f(x) - f(x_0)}{x - x_0} = \lim_{h \to 0} \frac{f(x_0 + h) - f(x_0)}{h}	
\]
without explicitly mentioning that $x \neq x_0$ and $h \neq 0$.
Note that the condition
\[
	f'(x_0) = \lim_{x \to x_0} \frac{f(x) - f(x_0)}{x - x_0}
\]
can be rewritten as
\[
	\lim_{x \to x_0} \frac{f(x) - f(x_0) - f'(x_0)(x - x_0)}{x - x_0} = 0,
\]
or equivalently, using the little-$o$ notation from Definition \ref{def*lil_o},
\begin{equation}
	\label{eq:derivative_lil_o}
	f(x) - f(x_0) - f'(x_0)(x - x_0) = o(x - x_0).
\end{equation}

\begin{remark}
	\label{rmk:diff_imp_cont}
	If $f:D \to \mathbb{R}$ is differentiable at $x_0$, then $f$ is also continuous at $x_0$. Indeed, using Equation \ref{eq:derivative_lil_o},
	\[
		\lim_{x \to x_0} f(x) = \lim_{x \to x_0} (f(x_0) + f'(x_0)(x - x_0) + o(x - x_0)) = f(x_0),
	\]
	hence $f$ is continuous at $x_0$.
\end{remark}

An alternative notation for the derivative of $f$ is $\frac{df}{dx}$. If $x_0 \in D$ is a right accumulation point of $D$, then $f$ is \textbf{differentiable from the right} at $x_0$ if the \textbf{right derivative}
\[
	f'_{+}(x_0) = \lim_{x \to x_0^{+}} \frac{f(x) - f(x_0)}{x - x_0} = \lim_{h \to 0^{+}} \frac{f(x_0 + h) - f(x_0)}{h}
\]
exists. \textbf{Differentiablility from the left} and the \textbf{left derivative} $f'_{-}(x_0)$ are defined analogously using the limit $x \to x_0^{-}$.

If $f:D \to \mathbb{R}$ is differentiable at $x_0 \in D$, the function $x \mapsto f(x_0) + f'(x_0)(x - x_0)$ is called the \textbf{affine approximation} of $f$ at $x_0$.

\begin{definition}{Higher Derivatives}{high_derivatives}
	Let $f:D \to \mathbb{R}$ be a function. We define the \textbf{higher derivatives} of $f$, if they exist, by
	\[
		f^{(0)} = f, \qquad f^{(1)} = f', \qquad f^{(2)} = f'',\; \hdots \;, f^{(n+1)} = (f^{(n)})'
	\]
	for all $n \in \mathbb{N}$. If $f^{(n)}$ exists, we say that $f$ is $n$\textbf{-times differentiable}. If the $n$-th derivative is also continuous, we say that $f$ is $n$\textbf{-times continuously differentiable}. We denote the set of $n$-times continuously differentiable functions on $D$ by $C^n(D)$.
\end{definition}

Equivalently, $C^0(D)$ is the set of real-valued continuous function on $D$, and $C^1(D)$ is the set of all differentiable functions  whose derivative is continuous (these are called \textbf{continuously differentiable} or of \textbf{class} $C^1$). Recursively, for $n \geq 1$, we define
\[
	C^n(D) = \{f:D \to \mathbb{R} \;|\; f \text{ is differentiable and } f' \in C^{n - 1}(D)\},
\]
and we say that $f \in C^{n}(D)$ is of \textbf{class} $C^n$.

\begin{definition}{Smooth Functions}{smooth_func}
	We define
	\[
		C^{\infty}(D) = \bigcap_{n = 0}^{\infty} C^n(D) = \{f:D \to \mathbb{R} \;|\; f \text{ is differentiable infinitely many times}\},
	\]
	and call functions $f \in C^{\infty}(D)$ \textbf{smooth} or of \textbf{class} $C^{\infty}$.
\end{definition}

The exponential function $\exp: \mathbb{R} \to \mathbb{R}$ is smooth.

\subsubsection{Differentiation Rules}
As with continuous functions, we rarely reprove differentiability from first principles for each new example. Instead, we use general rules that reduce differentiability of compound expressions to that of simpler ones.

\begin{proposition}{Derivative of Sum and Product}{derivative_sum_prod}
	Let $D \subseteq \mathbb{R}$ and $x_0 \in D$ be an accumulation point of $D \setminus \{x_0\}$. Let $f, g:D \to \mathbb{R}$ be differentiable at $x_0$. Then $f + g$ and $f \cdot g$ are differentiable at $x_0$, and
	\begin{subequations}
		\begin{align}
			\label{eq:sum_rule}
			(f + g)'(x_0) &= f'(x_0) + g'(x_0),\\
			\label{eq:prod_rule}
			(f \cdot g)'(x_0) &= f'(x_0)g(x_0) + f(x_0)g'(x_0).
		\end{align}
	\end{subequations}
	In particular, for any $\alpha \in \mathbb{R}$, the scalar multiple $\alpha f$ is differentiable at $x_0$ and $(\alpha f)'(x_0) = \alpha f'(x_0)$.
\end{proposition}

\begin{proof}
	Using the properties of limit discussed in Section \ref{sec:lim_vicinity_pt}, we have
	\begin{align*}
		\lim_{x \to x_0} \frac{(f + g)(x) - (f + g)(x_0)}{x - x_0} &= \lim_{x \to x_0} \left(\frac{f(x) - f(x_0)}{x - x_0} + \frac{g(x) g(x_0)}{x -x_0}\right)\\
		&=\lim_{x \to x_0} \frac{f(x) - f(x_0)}{x - x_0} + \lim_{x \to x_0} \frac{g(x) - g(x_0)}{x - x_0}\\
		&= f'(x_0) + g'(x_0),
	\end{align*}
	and
	\begin{align*}
		\lim_{x \to x_0} \frac{(fg)(x) - (fg)(x_0)}{x - x_0} &= \lim_{x \to x_0} \frac{(f(x) - f(x_0))g(x) + f(x_0)(g(x) - g(x_0)) }{x - x_0}\\
		&= \lim_{x \to x_0} \left(\frac{f(x) - f(x_0)}{x - x_0} g(x)\right) + f(x_0)\, \lim_{x \to x_0} \frac{g(x) - g(x_0)}{x - x_0}\\
		&= \lim_{x \to x_0} \left(\frac{f(x) - f(x_0)}{x - x_0}\right) \cdot \left( \lim_{x \to x_0} g(x) \right) + f(x_0)\, \lim_{x \to x_0} \frac{g(x) - g(x_0)}{x - x_0}\\
		&=
		f'(x_0)g(x_0) + f(x_0)g'(x_0),
	\end{align*}
	where we used that $g$ is continuous at $x_0$ (see Remark \ref{rmk:diff_imp_cont}) to conclude $\lim_{x \to x_0} g(x) = g(x_0)$. \qedhere
\end{proof}

\begin{corollary}{Higher Order Derivatives of the Sum and Product}{higher_derivative_prod}
	Let $f,g:D \to \mathbb{R}$ be $n$-times differentiable. Then $f + g$ and $f \cdot g$ are also $n$-times differentiable, and 
	\begin{align*}
		(f + g)^{(n)} &= f^{(n)} + g^{(n)},\\
		(fg)^{(n)} &= \sum_{k = 0}^{n} \binom{n}{k} f^{(k)} g^{(n - k)}.
	\end{align*}
	In particular, for every $\alpha \in \mathbb{R}$, $(\alpha f)^{(n)} = \alpha f^{(n)}$.
\end{corollary}

\begin{proof}
	For $n = 1$ this is Proposition \ref{prop*derivative_sum_prod}. The general case for $n \geq 1$ follows by induction. \qedhere
\end{proof}

\begin{corollary}{Derivatives of Polynomials}{derivative_poly}
	Polynomial functions are differentiable on all of $\mathbb{R}$. Moreover $(1)' = 0$ and $(x^n)' = nx^{n-1}$ for all $n \geq 1$.
\end{corollary}

\begin{proof}
	We argue by induction. For $n = 0$, we have
	\[
		(1)' = \lim_{x \to x_0} = \frac{1 - 1}{x - x_0} = 0.
	\]
	For $n = 1$, we have
	\[
		(x)' = \lim_{x \to x_0} \frac{x - x_0}{x - x_0} = 1.
	\]
	
	For $n > 1$, assume $(x^n)' = nx^{n-1}$. Then by Equation \eqref{eq:prod_rule}, since $x^{n+1} = x\cdot x^n$, we have
	\[
		(x^{n+1})' = (x \cdot x^n)' = 1 \cdot x^n + x \cdot n x^{n-1} = (n + 1)x^n.
	\]
	This proves the inductive step and establishes the result. Finally, the linearity of the derivative (see Equation \eqref{eq:sum_rule}) yields the differentiability of any polynomial. \qedhere
\end{proof}

\subsubsection*{Example}
With $\alpha = \pm 1$ and $\alpha = \pm i$, we have
\[
	(e^x)' = e^x, \qquad (e^{-x})' = -e^{-x}, \qquad (e^{ix})' = ie^{ix}, \qquad (e^{-ix})' = -ie^{-ix}.
\]
By Theorem \ref{theo*cmplx_exp_sin_cos}, we have
\[
	\sin'(x) = \frac{(e^{ix})' - (e^{-ix})'}{2i} = \frac{e^{ix} + e^{ix}}{2} = \cos(x),
\]
and analogously, $\cos'(x) = -\sin(x)$. Similarly $\sinh'(x) = \cosh(x)$ and $\cosh'(x) = \sinh(x)$.

\begin{theorem}{Chain Rule}{chain_rule}
	Let $D, E \subseteq \mathbb{R}$ and let $x_0 \in D$ be an accumulation point of $D \setminus \{x_0\}$. Let $f: D \to E$ be differentiable at $x_0$ such that $y_0 = f(x_0)$ is an accumulation point of $E \setminus \{y_0\}$, and let $g:E \to \mathbb{R}$ be differentiable at $y_0$. Then $g \circ f:D \to \mathbb{R}$ is differentiable at $x_0$ and 
	\[
		(g \circ f)'(x_0) = g'(f(x_0))f'(x_0).
	\]
\end{theorem}

\begin{proof}
	Observer that we can write
	\begin{align*}
		g(y) &= g(y_0) [g(y) - g(y_0)]\\
		&= g(y_0) + g'(y_0)(y - y_0) + [g(y) - g(y_0) - g'(y_0)(y - y_0)]\\
		&= g(y_0) + g'(y_0)(y - y_0) + \omega(y)(y - y_0),
	\end{align*}
	where $\omega: E \to \mathbb{R}$ is defined as
	\[
		\omega(y) = \begin{cases}
			\dfrac{g(y) - g(y_0)}{y - y_0} - g'(y_0) \qquad &\text{for }y \in E \setminus \{y_0\},\\
			0 \qquad &\text{for } y = y_0.
		\end{cases}
	\]
	Since $g$ is continuous at $y_0$, it follows that $\omega(y) \to 0$ as $y \to y_0$; hence, the function $\omega$ is continuous at $y_0$. Substituting $y = f(x)$ and using $y_0 = f(x_0)$, we get
	\[
		g(f(x)) = g(f(x_0)) + g'(f(x_0))[f(x) - f(x_0)] + \omega(f(x))[f(x) - f(x_0)],
	\]
	therefore
	\begin{align*}
		 \lim_{x \to x_0} \frac{g(g(x)) - g(f(x_0))}{x - x_0} &= \lim_{x \to x_0} \bigg( g'(f(x_0))\frac{f(x) - f(x_0)}{x - x_0} + \omega(f(x)) \frac{f(x) - f(x_0)}{x - x_0} \bigg)\\
		 &= g'(f(x_0)) f'(x_0) + \underbrace{\omega(f(x_0))}_{= 0} f'(x_0) = g'(f(x_0)) f'(x_0),
	\end{align*}
	where we used the continuity of $\omega$ at $y_0 = f(x_0)$ to deduce that $\omega(f(x)) \to \omega(f(x_0))$ as $x \to x_0$. \qedhere
\end{proof}

\begin{corollary}{Quotient Rule}{quot_rule}
	Let $D \subseteq \mathbb{R}$. let $x_0$ be an accumulation point of $D \setminus \{x_0\}$, and let $f,g:D \to \mathbb{R}$ be differentiable at $x_0$. If $g(x_0) \neq 0$, then $\frac{f}{g}$ is differentiable at $x_0$, and
	\[
		\left(\frac{f}{g}\right)'(x_0) = \frac{f'(x_0)g(x_0) - f(x_0)g'(x_0)}{g(x_0)^2}.
	\]
\end{corollary}

\begin{proof}
	Consider the function $\psi: \mathbb{R} \setminus \{0\} \to \mathbb{R}$ given by $\psi(y) = \frac{1}{y}$. This function is differentiable, with $\phi'(y) = -\frac{1}{y^2}$. Then, by the chain rule (Theorem \ref{theo*chain_rule}), $\frac{1}{g} = \psi \circ g$ is differentiable at $x_0$, with
	\[
		\bigg( \frac{1}{g}\bigg)'(x_0) = \psi'(g(x_0)) g'(x_0) = -\frac{g'(x_0)}{g(x_0)^2}.
	\]
	Applying now the product rule (Proposition \ref{prop*derivative_sum_prod}), $\frac{f}{g} = f \cdot \frac{1}{g}$ is differentiable at $x_0$, and
	\[
		\left(\frac{f}{g}\right)'(x_0) = \left(f \cdot \frac{1}{g}\right)'(x_0) = f'(x_0)\frac{1}{g(x_0)} - f(x_0) \frac{g'(x_0)}{g(x_0)^2} = \frac{f'(x_0)g(x_0) - f(x_0)g'(x_0)}{g^(x_0^2)}. \qedhere 
	\]
\end{proof}

\begin{theorem}{Derivative of the Inverse}{derivative_inv}
	Let $D, E \subseteq\mathbb{R}$, and let $f:D \to E$ be a continuous bijection whose inverse $f^{-1}:E \to D$ is also continuous. Let $\bar{x} \in D$ be an accumulation point of $D \setminus \{\bar{x}\}$, and assume $f$ is differentiable at $\bar{x}$ with $f'(\bar{x}) \neq 0$. Then $f^{-1}$ is differentiable at $\bar{y} = f(\bar{x})$ and
	\[
		(f^{-1})'(\bar{y}) = \frac{1}{f'(\bar{x})} = \frac{1}{f'(f^{-1}(\bar{y}))}.
	\] 
\end{theorem}

\begin{proof}
	To compute $(f^{-1})'(\bar{y})$, take a sequence $(y_n)_{n=0}^{\infty} \subseteq E \setminus \{\bar{y}\}$ with $y_n \to \bar{y}$ and set $x_n = f^{-1}(y_n)$. Then
	\[
		\frac{f^{-1}(y_n) - f^{-1}(\bar{y})}{y_n - \bar{y}} = \frac{x_n - \bar{x}}{f(x_n) - f(\bar{x})} = \left(\frac{f(x_n) - f(\bar{x})}{x_n - \bar{x}}\right)^{-1}.
	\]
	By the continuity of $f^{-1}$ we have $f^{-1}(y_n) = x_n \to \bar{x} = f^{-1}(\bar{y})$ as $y_n \to \bar{y}$. Thus, since $f$ is differentiable at $\bar{x}$ with $f'(\bar{x}) \neq 0$, Proposition \ref{prop*lim_op}(4) implies
	\[
		\left(\frac{f(x_n) - f(\bar{x})}{x_n - \bar{x}}\right)^{-1} \longrightarrow \frac{1}{f'(\bar{x})},
	\]
	proving that
	\[
		\lim_{n \to \infty} \frac{f^{-1}(y_n) - f^{-1}(\bar{y})}{y_n - \bar{y}} = \frac{1}{f'(\bar{x})}.
	\]
	Since the sequence $(y_n)_{n=0}^{\infty}$ was arbitrary, Lemma \ref{lem*lim_and_seq} gives
	\[
		\lim_{y \to \bar{y}} \frac{f^{-1}(y) - f^{-1}(\bar{y})}{y - \bar{y}} = \frac{1}{f'(\bar{x})},
	\]
	as desired. \qedhere
\end{proof}

\subsection{Main Theorems of Differential Calculus}

\subsubsection{Local Extrema}

\begin{definition}{Local Extrema}{local_extrema}
	Let $D \subseteq \mathbb{R}$ and $x_0 \in D$. We say that a function $f:D \to \mathbb{R}$ has a \textbf{local maximum} at $x_0$, if there exists $\delta > 0$ such that
	\[
		f(x) \leq f(x_0) \qquad \forall x \in D \cap (x_0 - \delta, x_0 + \delta).
	\]
	If the inequality is strict (i.e. $f(x) < f(x_0)$ for all $x \in (x_0 - \delta, x_0 + \delta) \setminus \{x_0\}$), then $f$ has a \textbf{strict local maximum} at $x_0$. A (\textbf{strict}) \textbf{local minimum} is defined analogously. We call $x_0$ a \textbf{local extremum} if $f$ has either a local maximum or a local minimum at $x_0$.
\end{definition}

\begin{proposition}{Local Extrema vs. First Derivative}{local_extrema_deriv}
	Let $D \subseteq \mathbb{R}$ and $f:D \to \mathbb{R}$. Suppose $x_0 \in D$ is a local extremum of $f$, that $f$ is differentiable at $x_0$ and that $x_0$ is both a right-hand and a left-hand accumulation point of $D$. Then
	\[
		f'(x_0) = 0.
	\]
\end{proposition}

\begin{proof}
	W.l.o.g, assume that $f$ has a local maximum at $x_0$ (otherwise replace $f$ by $-f$). We first note that for $x$ close to $x_0$ and to the right of it, we have $f(x) - f(x_0) \leq 0$ and $x - x_0 > 0$. Hence,
	\[
		f'_{+}(x_0) = \lim_{x \to 0^+} \frac{f(x) - f(x_0)}{x - x_0} \leq 0.
	\]
	Similarly for $x$ close to $x_0$ and to the left of it, we have $f(x) - f(x_0) \leq 0$ and $x - x_0 < 0$, so
	\[
		f'_{-}(x_0) = \lim_{x \to x_0^-} \frac{f(x) - f(x_0)}{x - x_0} \geq 0.
	\]
	Since $f$ is differentiable at $x_0$, the two one sided derivatives coincide, i.e.,
	\[
		 f'(x_0) = f'_{+}(x_0) = f'_{-}(x_0) = 0,
	\]
	and therefore $f'(x_0) = 0$. \qedhere
\end{proof}

\begin{corollary}{Local Extrema in an Inteval}{local_extrema_inteval}
	Let $I \subseteq \mathbb{R}$ be an interval and $f:I \to \mathbb{R}$. If $x_0 \in I$ is a local extrema, then at least one of the following statements holds:
	\begin{enumerate}
		\item $x_0$ is an endpoint of $I$,
		\item $f$ is not differentiable at $x_0$,
		\item $f$ is differentiable at $x_0$ and $f'(x_0) = 0$.
	\end{enumerate}
	In particular, all local extrema of a differentiable function on an open interval are zeros of the derivative.
\end{corollary}

\subsubsection{The Mean Value Theorem}

\begin{theorem}{Rolle's Theorem}{rolle_theo}
	Let $f:[a,b] \to \mathbb{R}$ be continuous on $[a,b]$ and differentiable on $(a,b)$. If $f(a) = f(b)$, then there exists $\xi \in (a,b)$ with $f'(\xi) = 0$.
\end{theorem}

\begin{proof}
	By Theorem \ref{theo*extreme_val_theo} $f$ attains both its maximum and minimum on $[a,b]$ at some points $x_0, x_1 \in [a,b]$. By Proposition \ref{prop*local_extrema_deriv} any interior extremum has zero derivative. Thus, we consider two cases:
	\begin{enumerate}
		\item[(i)] If either $x_0$ or $x_1$ lies in $(a,b)$ we are done.
		\item[(ii)] If both $x_0$ and $x_1$ are endpoints, since $f(a) = f(b)$, then $\min f = \max f = f(a) = f(b)$, therefore $f$ is constant. In particular, $f'(x) = 0$ for all $x \in (a,b)$ and the result follows also in this case. \qedhere
	\end{enumerate}
\end{proof}

\begin{corollary}{Non-Vanishing Derivative Implies Different Endpoint Values}{non_zero_deriv}
	Let $f:[a,b] \to \mathbb{R}$ be continuous on $[a,b]$ and differentiable on $(a,b)$. If $f'(x) \neq 0$ for all $x \in (a,b)$, then $f(a) \neq f(b)$.
\end{corollary}

\begin{proof}
	Assume, by contradiction, that $f(a) = f(b)$. Then by Rolle's Theorem, there would exists $\xi \in (a,b)$ such that $f'(\xi) = 0$, which would contradict the assumption that $f'$ never vanishes. \qedhere
\end{proof}

\begin{theorem}{Mean Value Theorem}{mean_val_theo}
	Let $f:[a,b] \to \mathbb{R}$ be continuous on $[a,b]$ and differentiable on $(a,b)$. Then there exists $\xi \in (a,b)$ such that
	\[
		f'(\xi) = \frac{f(b) - f(a)}{b - a}.
	\]
\end{theorem}

\begin{proof}
	Define $g:[a,b] \to \mathbb{R}$ by
	\[
		g(x) = f(x) - \frac{f(b) - f(a)}{b - a} (x - a).
	\]
	Then $g$ is continuous on $[a,b]$ and differentiable on $(a,b)$, and satisfies
	\[
		g(a) = f(a), \qquad g(b) = f(b) - (f(b) - f(a)) = f(a).
	\]
	By Rolle's Theorem \ref{theo*rolle_theo}, there exists $\xi \in (a,b)$ such that
	\[
		0 = g'(\xi) = f'(\xi) - \frac{f(b) - f(a)}{b - a},
	\]
	proving the result. \qedhere
\end{proof}

\begin{corollary}{Lipschitz Continuity vs. Bounded Derivative}
	Let $f:[a,b] \to \mathbb{R}$ be continuous on $[a,b]$ and differentiable on $(a,b)$. Then $f$ is Lipschitz continuous on $[a,b]$ if and only if $ f'$ is bounded on $(a,b)$.
\end{corollary}

\begin{proof}
	Suppose first that $f$ is Lipschitz continuous on $[a,b]$ with constant $L$. This implies that given $x, x_0 \in (a,b)$ with $x \neq x_0$,
	\[
		\left|\frac{f(x) - f(x_0)}{x - x_0}\right| \leq L.
	\]
	Taking the limit as $x \to x_0$ gives $|f'(x_0)| \leq L$, so $f'$ is bounded on $(a,b)$.
	
	Conversely, suppose that $f'$ is bounded on $(a,b)$, say $f'(z) \leq M$ for all $z \in (a,b)$. Then given $x, y \in [a,b]$ with $x < y$, the Mean Value Theorem \ref{theo*mean_val_theo} applied on the interval $[x, y]$ yields $\xi \in (x,y) \subseteq (a,b)$ such that
	\[
		f(y) - f(x) = f'(\xi)(y - x),
	\]
	therefore
	\[
		|f(y) - f(x)| = |f'(\xi)| |y - x| \leq M |y - x|.
	\]
	Since $x, y \in [a,b]$ are arbitrary, this shows that $f$ is Lipschitz continuous on $[a,b]$ with Lipschitz constant $M$. \qedhere
\end{proof}

\begin{theorem}{Cauchy Mean Value Theroem}{cauchy_mean_val}
	Let $f, g: [a,b] \to \mathbb{R}$ be continuous on $[a,b]$ and differentiable on $(a,b)$. Then there exists $\xi \in (a,b)$ such that
	\begin{equation}
		\label{eq:cauchy_mean_val}
		g'(\xi)(f(b) - f(a)) = f'(\xi)(g(b) - g(a)).
	\end{equation}
	If, in addition $g'(x) \neq 0$ for all $x \in (a,b)$, then $g(a) \neq g(b)$ and
	\[
		\frac{f'(\xi)}{g'(\xi)} = \frac{f(b) - f(a)}{g(b) - g(a)}.
	\]
\end{theorem}

\begin{proof}
	Define the function $F:[a,b] \to \mathbb{R}$ as
	\[
		F(x) = g(x)(f(b) - f(a)) - f(x)(g(b) - g(a)).
	\]
	Then
	\begin{align*}
		F(a) &= g(a)(f(b) - f(a)) - f(a)(g(b) - g(a)) = g(a)f(b) - f(a)g(b)\\
		F(b) &= g(b)(f(b) - f(a)) - f(b)(g(b) - g(a)) = g(a)f(b) - f(a)g(b).
	\end{align*}
	Thus, by Rolle's Theorem \ref{theo*rolle_theo}, there exists $\xi \in (a,b)$ such that
	\[
		F'(\xi) = 0 = g'(\xi)(f(b) - f(a)) - f'(\xi)(g(b) - g(a)),
	\]
	which is Equation \ref{eq:cauchy_mean_val}.
	
	If $g'(x) \neq 0$ for all $x \in (a,b)$, then Corollary \ref{cor*non_zero_deriv} yields $g(a) \neq g(b)$. Dividing Equation \ref{eq:cauchy_mean_val} accordingly gives the second formula. \qedhere
\end{proof}

\subsubsection{L'Hopital's Rule}
\begin{theorem}{L'Hopital's Rule}{hopital}
	Let $f, g: (a,b) \to \mathbb{R}$ be differentiable. Suppose:
	\begin{enumerate}
		\item $g(x) \neq 0$ and $g'(x) \neq 0$ for all $x \in (a,b)$,
		\item $\lim_{x \to a^{+}} f(x) = \lim_{x \to a^{+}} g(x) = 0$,
		\item the limit $L = \lim_{x \to a^{+}} \dfrac{f'(x)}{g'(x)}$ exists.
	\end{enumerate}
	Then $\lim_{x \to a^{+}} \dfrac{f(x)}{g(x)}$ exists and equals $L$.
\end{theorem}

\begin{proof}
	By (2), we can extend $f$ and $g$ continuously to $[a, b)$ by setting $f(a) = g(a) = 0$. Fix $\varepsilon > 0$. By (3), there exists $\delta > 0$ such that
	\[
		\frac{f'(\xi)}{g'(\xi)} \in (L - \varepsilon, L + \varepsilon) \qquad \forall x \in (a, a + \delta).
	\]
	Now, for any $x \in (a, a + \delta)$, we can apply Cauchy's Mean Value Theorem \ref{theo*cauchy_mean_val} to $f$ and $g$ on $[a, x]$ to find some $\xi_x \in (a,x)$ with
	\[
		\frac{f(x)}{g(x)} = \frac{f(x) - f(a)}{g(x) - g(a)} = \frac{f'(\xi_x)}{g'(\xi_x)}.
	\]
	Since $\xi_x \in (a, x) \subseteq (a, a + \delta)$, it follows that
	\[
		\frac{f(x)}{g(x)} = \frac{f'(\xi_x)}{g'(\xi_x)} \in (L - \varepsilon, L + \varepsilon) \qquad \forall x \in (a, a + \delta).
	\]
	Because $\varepsilon > 0$ is arbitrary, this proves that $\lim_{x \to a^+} \dfrac{f(x)}{g(x)} = L$.
\end{proof}

\begin{theorem}{L'Hopital's Rule for Improper Limits}{hopital_improper_lim}
	Let $f,g:(a,b) \to \mathbb{R}$ be differentiable. Suppose:
	\begin{enumerate}
		\item $g(x) \neq 0$ and $g'(x) \neq 0$ for all $x \in (a,b)$,
		\item $\lim_{x \to a^+} |f(x)| = \lim_{x \to a^{+}} |g(x)| = \infty$,
		\item the limit $L = \lim_{x \to a^{+}} \dfrac{f'(x)}{g'(x)}$ exists.
	\end{enumerate}
	Then $\lim_{x \to a^{+}} \dfrac{f(x)}{g(x)}$ exists and equals $L$.
\end{theorem}

\begin{theorem}{L'Hopital's Rule at Infinity}{hopital_infinity}
	Let $R > 0$ and $f,g:(R, \infty) \to \mathbb{R}$ be differentiable. Suppose:
	\begin{enumerate}
		\item $g(x) \neq 0$ and $g'(x) \neq 0$ for all $x \in (R, \infty)$,
		\item either $\lim_{x \to \infty} f(x) = \lim_{x \to \infty} g(x) = 0$ or $\lim_{x \to \infty} |f(x)| = \lim_{x \to \infty} |g(x)| = \infty$,
		\item the limit $L = \lim_{x \to \infty} \dfrac{f'(x)}{g'(x)}$ exists.
	\end{enumerate}
	Then $\lim_{x \to \infty} \dfrac{f(x)}{g(x)}$ exists and equals $L$.
\end{theorem}

\subsubsection{Monotonicity and Convexity via Differential Calculus}
Here $I$ always denotes a non-trivial interval (non-empty and not a single point).

\begin{proposition}{Monotonicity vs. First Derivative}{mono_deriv}
	Let $I \subseteq \mathbb{R}$ be an interval and let $f:I \to \mathbb{R}$ be differentiable. Then
	\[
		f' \geq 0 \quad \Leftrightarrow \quad f \text{ is increasing}.
	\]
\end{proposition}

\begin{proof}
	If $f$ is increasing $f(x + h) - f(x) \geq 0$ for $h > 0$, and $f(x + h) - f(x) \leq 0$ for $h < 0$. Hence, in both cases, $\frac{f(x+h) - f(x)}{h} \geq 0$, therefore
	\[
		f'(x) = \lim_{h \to 0} \frac{f(x + h) - f(x)}{h} \geq 0.
	\]
	Conversely, assume $f$ is not increasing. Then there exists $x_1 < x_2$ with $f(x_1) > f(x_2)$. By the Mean Value Theorem \ref{theo*mean_val_theo}, there exists $\xi \in (x_1, x_2)$ with
	\[
		f'(\xi) = \frac{f(x_2) - f(x_1)}{x_2 - x_1} < 0,
	\]
	so $f' \not \geq 0$ on $I$. \qedhere
\end{proof}

\begin{remark}
	If $f' > 0$, the same argument shows that $f$ is strictly increasing. However, the converse fails: the function $f(x) = x^3$is strictly increasing but $f'(0) = 0$.
\end{remark}

\begin{corollary}{Constant Functions vs. First Derivative}{const_deriv}
	Let $I \subseteq \mathbb{R}$ be an interval and $f:I \to \mathbb{R}$. Then $f$ is constant if and only if $f$ is differentiable and $f'(x) = 0$ for all $x \in I$.
\end{corollary}

\begin{proof}
	The derivative of a constant function is 0.
	
	Conversely, if $f' = 0$, then $f' \geq 0$ and $-f' \geq 0$, so by Proposition \ref{prop*mono_deriv} both $f$ and $-f$ are increasing, hence $f$ is constant. \qedhere
\end{proof}

\begin{definition}{Convex Functions}{convex_func}
	Let $I \subseteq \mathbb{R}$ and $f:I \to \mathbb{R}$. We call $f$ \textbf{convex}, if for all $a,b \in I$ with $a < b$ and all $t \in (0, 1)$, it holds that
	\begin{equation}
		\label{eq:convex_func}
		f((1-t)a + tb) \leq (1-t)f(a) + tf(b).
	\end{equation}
	We call $f$ \textbf{strictly convex} if the inequality in \eqref{eq:convex_func} is strict. A function $g:I \to \mathbb{R}$ is (\textbf{strictly}) \textbf{concave} if $-g$ is (strictly) convex.
	
	An equivalent definition of a convex function is the following: $f:I \to \mathbb{R}$ is convex if for all $a,b \in I$ with $a < b$ and all $x \in (a,b)$, we have
	\begin{equation}
		\label{eq:convex_func_2}
		\frac{f(x) - f(a)}{x - a} \leq \frac{f(b) - f(x)}{b - x},
	\end{equation}
	and strictly convex if the inequality is strict.
\end{definition}

\begin{proposition}{Convexity vs. Monotonicity of the First Derivative}{convex_deriv}
	Let $I \subseteq \mathbb{R}$ and let $f:I \to \mathbb{R}$ be differentiable. Then $f$ is convex if and only if $f'$ is increasing.
\end{proposition}

\begin{proof}
	Assume $f'$ is increasing. Then, for $a < b$ and $x \in (a, b)$, the Mean Value Theorem \ref{theo*mean_val_theo} applied on the intervals $[a, x]$ and $[x, b]$ yields $\xi \in (a, x)$ and $\zeta \in (x, b)$ such that
	\[
		f'(\xi) = \frac{f(x) - f(a)}{x - a}, \qquad f'(\zeta) = \frac{f(b) - f(x)}{b - x}.
	\]
	Since $f'$ is increasing, we have $f'(\xi) \leq f'(\zeta)$, so Equation \eqref{eq:convex_func_2} follows. Since $a < b \in I$ and $x \in (a, b)$ are arbitrary, $f$ is convex.
	
	Conversely, assume $f$ is convex. Given $a < b$, consider $h > 0$ small enough, such that $a + h < b - h$ and apply Equation \eqref{eq:convex_func_2} twice: first, applying it on the interval $(a, b - h)$ with $x = a + h$ we get
	\[
		\frac{f(a + h) - f(a)}{h} \leq \frac{f(b - h) - f(a + h)}{(b- h) - (a + h)};
	\]
	then, applying it on the interval $(a + h, b)$ with $x = b - h$ we get
	\[
		\frac{f(b - h) - f(a + h)}{(b - h) - (a + h)} \leq \frac{f(b) - f(b - h)}{h}.
	\]
	Combining these two inequalities, we deduce that, for small $h > 0$,
	\[
		\frac{f(a + h) - f(a)}{h} \leq \frac{f(b) - f(b - h)}{h}.
	\]
	Letting $h \to 0^+$ gives $f'(a) \leq f'(b)$. Since $a < b$ are arbitrary, $f'$ is increasing. \qedhere
\end{proof}

\begin{corollary}{Convexity vs. Second Derivative}{convex_second_deriv}
	Let $I \subseteq \mathbb{R}$ and let $f:I \to \mathbb{R}$ be twice differentiable. Then $f$ is convex if and only if $f'' \geq 0$.
\end{corollary}

\begin{proof}
	By Proposition \ref{prop*convex_deriv} $f$ is convex if and only if $f'$ is increasing. Applying Proposition \ref{prop*mono_deriv} to $f'$, we have that $f'$ is increasing if and only if $f'' \geq 0$. \qedhere
\end{proof}

\subsection{Example: Differentiation of Trigonometric Functions}
In this section we will list the derivatives of various trigonometric functions that can be looked up, when solving integrals, in order to find a certain primitive.
\subsubsection{Sine and Arc Sine}
The function $\sin:[-\frac{\pi}{2}, \frac{\pi}{2}] \to [-1, 1]$ satisfies
\[
	\sin'(x) = \cos(x), \qquad \cos'(x) = -\sin(x).
\]
Its inverse function $\arcsin:[-1, 1]\to [-\frac{\pi}{2}, \frac{\pi}{2}]$ has the derivative
\[
	\arcsin'(x) = \frac{1}{\cos(\arcsin(x))} = \frac{1}{\sqrt{1 - \sin^2(\arcsin(x))}} = \frac{1}{\sqrt{1 - x^2}},
\]
where we used the derivative of the inverse (Theorem \ref{theo*derivative_inv}) to calculate the above derivative.

\subsubsection{Cosine and Arc Cosine}
Similarly the function $\cos:[0, \pi] \to [-1, 1]$ has the inverse function $\arccos:[-1, 1] \to [0, \pi]$, which has the derivative
\[
	\arccos'(x) = \frac{1}{-\sin(\arccos(x))} = -\frac{1}{\sqrt{1 - \cos^2(\arccos(x))}} = -\frac{1}{\sqrt{1 - x^2}}.
\]

\subsubsection{Tangent and Arc Tangent}
The function $\tan:(-\frac{\pi}{2}, \frac{\pi}{2}) \to \mathbb{R}$ has the derivative
\[
	\tan'(x) = \left(\frac{\sin(x)}{\cos(x)}\right)' = \frac{\cos(x)\cos(x) - \sin(x)(-\sin(x))}{\cos^2(x)} = \frac{1}{\cos^2(x)}.
\]
Its inverse function $\arctan:\mathbb{R} \to (-\frac{\pi}{2}, \frac{\pi}{2})$ has the derivative
\[
	\arctan'(x) = \frac{1}{\dfrac{1}{\cos^2(\arctan(x))}} = \cos^2(\arctan(x)).
\]
Since 
\[
	\tan^2(x) = \frac{\sin^2(x)}{\cos^2(x)} = \frac{1 - \cos^2(x)}{\cos^2(x)} = \frac{1}{\cos^2(x)} - 1,
\]
it follows that
\[
	\arctan'(x) = \cos^2(\arctan(x)) = \frac{1}{1 + \tan^2(\arctan(x))} = \frac{1}{1 + x^2}.
\]

The cotangent and its inverse behave similarly, i.e. 
\[
	\cot'(x) = \left(\frac{\cos(x)}{\sin(x)}\right)' = \frac{-\sin(x)\sin(x) - \cos(x)\cos(x)}{\sin^2(x)} = \frac{-1}{\sin^2(x)},
\]
and
\[
	\operatorname{arccot}'(x) = -\frac{1}{1 + x^2}.
\]

\subsubsection{Hyperbolic Functions}
We now perform the analogous analysis for the hyperbolic trigonometric functions
\[
	\sinh(x) = \frac{e^x - e^{-x}}{2}, \qquad \cosh(x) = \frac{e^x + e^{-x}}{2}, \qquad \tanh(x) = \frac{\sinh(x)}{\cosh(x)}.
\]
The function $\sinh: \mathbb{R} \to \mathbb{R}$ is bijective and satisfies
\[
	\sinh'(x) = \cosh(x), \qquad \sinh''(x) = \sinh(x).
\] 
Its inverse $\operatorname{arsinh}:\mathbb{R} \to \mathbb{R}$ is called the inverse hyperbolic sine. Its derivative is given by
\[
	\operatorname{arsinh}'(x) = \frac{1}{\cosh(\operatorname{arsinh}(x))} = \frac{1}{\sqrt{1 + \sinh^2(\operatorname{arsinh}(x))}} = \frac{1}{\sqrt{1 + x^2}}.
\]

The hyperbolic cosine satisfies
\[
	\cosh'(x) = \sinh(x), \qquad \cosh''(x) = \cosh(x).
\]
Its inverse $\operatorname{arcosh}:[1,\infty] \to [0, \infty]$ has the derivative
\[
	\operatorname{arcosh}'(x) = \frac{1}{\sinh(\operatorname{arcosh}(x))} = \frac{1}{\sqrt{\cosh^2(\operatorname{arcosh}(x)) - 1}} = \frac{1}{\sqrt{x^2 - 1}}.
\]

The inverse hyperbolic tangent $\operatorname{artanh}:(-1, 1) \to \mathbb{R}$ has the derivative
\[
	\operatorname{artanh}'(x) = \frac{1}{1 - x^2}.
\]

Notice that the derivatives containing the inverse functions of either the tangent or hyperbolic tangent do not have a root in the denominator. One can memorize this by the sentence: ''You can't get tan under a roof''.




	%
% (c) 2025 Autor, ETH Zürich
%
% !TEX root = main.tex
% !TEX encoding = UTF-8
%

\section{The Riemann Integral}

\subsection{Step Functions and their Integral}
\subsubsection{Decompositions and Step Functions}

\begin{definition}{Partitions}{partitions}
	Two sets $A, B$ are called \textbf{disjoint} if $A \cap B = \emptyset$. For a collection $\mathcal{A}$ of sets, we say that the sets in $\mathcal{A}$ are pairwise disjoint, if for all $A_1, A_2 \in \mathcal{A}$ with $A_1 \neq A_2$ it holds that $A_1 \cap A_2 = \emptyset$.
	
	Let $X$ be a set. A Partition of $X$ is a family $\mathcal{P}$ of non-empty pairwise disjoint subsets of $X$ such that
	\[
		X = \bigcup_{P \in \mathcal{P}} P.
	\]
\end{definition}

\begin{definition}{Decomposition of an Interval}{decomp_interval}
	A \textbf{decomposition} of $[a, b]$ is a finite sequence of points
	\[
		a = x_0 < x_1 < \; \hdots \; < x_{n-1} < x_n = b,
	\]
	with $n \in \mathbb{N}$. The points $x_0, \hdots , x_n$ are called the \textbf{division points} of the decomposition.
\end{definition}

Formally, a decomposition of $[a,b]$ is a finite subset of $[a,b]$ containing $a$ and $b$, together with the ordering of its elements. Each decomposition induces a natural partition of $[a,b]$, i.e.,
\[
	[a,b] = \{x_0\} \cup (x_0, x_1) \cup \{x_1\} \cup \; \hdots \; \cup (x_{n-1}, x_n) \cup \{x_n\},
\]
which we will use implicitly from now on.

A decomposition
\[
	a = y_0  < y_1 < \; \hdots \; < y_m = b
\]
is called a \textbf{refinement} of the decomposition
\[
	a = x_0 < x_1 < \; \hdots \; < x_n = b
\]
if
\[
	\{x_0, x_1, \hdots, x_n\} \subseteq \{y_0, y_1, \hdots , y_m\}.
\]
The notion of refinement defines a partial order on the set of all decompositions of $[a,b]$. Note that any two decompositions of $[a,b]$ admit a common refinement given by the union of all division points.

\begin{definition}{Step Functions}{step_func}
	A function $f:[a,b] \to \mathbb{R}$ is called a \textbf{step function} if there exists a decomposition
	\[
		a = x_0 < x_1 < \; \hdots \; < x_n = b
	\]
	such that, for each $k = 1, \hdots , n$ the restriction of $f$ to the open interval $(x_{k-1}, x_k)$ is constant. In this case, we say that $f$ is a step function \textit{with respect to} the decomposition $a= x_0  < x_1 < \; \hdots \; < x_n = b$.
\end{definition}

\begin{proposition}{Linearity of the Space of Step Functions}{lin_step_func}
	Let $f, g:[a, b] \to \mathbb{R}$ be step functions, and $\alpha, \beta \in \mathbb{R}$. Then $\alpha f + \beta g$ is also a step function.
\end{proposition}

\begin{proof}
	Let $f$ be a step function with respect to the decomposition $a = x_0 < x_1 <\; \hdots \; < x_n = b$, and let $g$ be a step function with respect to the decomposition $a = y_0 < y_1 <\; \hdots \; < y_m = b$. The union of all division points $\{x_0, x_1, \hdots , x_n\} \cup \{y_0, y_1, \hdots , y_m\}$ defines a new decomposition
	\[
		a = z_0 < z_1 < \; \hdots \; < z_N = b
	\]
	that is a common refinement of the two. Since both $f$ and $g$ are constant on each open interval $(z_{k - 1}, z_k)$, so is the function $\alpha f + \beta g$. Thus, $\alpha f + \beta g$ is also a step function with respect to this decomposition. \qedhere
\end{proof}

\begin{remark}
	As in the proof of Proposition \ref{prop*lin_step_func}, one can show that the product of two step functions is again a step function.  Moreover, step functions are bounded, since they take only finitely many values.
\end{remark}

\subsubsection{The Integral of a Step Function}

\begin{definition}{Integral of a Step Function}{integral_step_func}
	Let $f:[a,b] \to \mathbb{R}$ be a step function with respect to a decomposition
	\[
		a = x_0 < x_1 <\; \hdots \; < x_n = b.
	\]
	We define the \textbf{integral} of $f$ on $[a, b]$ as the real number
	\begin{equation}
		\label{eq:int_step_func}
		\int_{a}^{b} f(x)\, dx = \sum_{k=1}^{n} c_k (x_{k} - x_{k-1}),
	\end{equation}
	where $c_k$ denotes the constant value of $f$ on the interval $(x_{k - 1}, x_k)$.
\end{definition}

\begin{proposition}{Linearity of the Integral of Step Functions}{lin_int_step_func}
	Let $f,g:[a,b] \to \mathbb{R}$ be step functions, and let $\alpha , \beta \in \mathbb{R}$. Then
	\[
		\int_{a}^{b} (\alpha f + \beta g)(x) \, dx = \alpha \int_{a}^{b} f(x) \, dx + \beta \int_{a }^{b} g(x) \, dx.
	\]
\end{proposition}

\begin{proof}
	As in the proof of Proposition \ref{prop*lin_step_func}, we can find a decomposition $a = x_0 < x_1 <\; \hdots \; < x_n = b$ such that both $f$ and $g$ (and hence $\alpha f + \beta g$) are constant on the interval $(x_{k-1}, x_k)$. If $f$ takes the values $c_k$ and $g$ the values $d_k$ on $(x_{k-1}, x_k)$, then $\alpha f + \beta g$ takes the value $\alpha c_k + \beta d_k$. Thus,
	\begin{align*}
		\int_{a}^{b} (\alpha f + \beta g)(x) \, dx &= \sum_{k=1}^{n} (\alpha c_k + \beta d_k) (x_k - x_{k-1})\\
		&= \alpha \sum_{k=1}^{n} c_k (x_k - x_{k-1}) + \beta \sum_{k=1}^{n} d_k (x_k - x_{k-1})\\
		&= \alpha \int_{a}^{b} f(x) \, dx + \beta \int_{a}^{b} g(x) \, dx,
	\end{align*}
	as claimed. \qedhere
\end{proof}

\begin{proposition}{Monotonicity of the Integral of Step Functions}{mono_int_step_func}
	Let $f,g:[a,b] \to \mathbb{R}$ be step functions such that $f \leq g$. Then
	\[
		\int_{a}^{b} f(x) \, dx \leq \int_{a}^{b} g(x) \, dx.
	\]
\end{proposition}

\begin{proof}
	As in the proof of Proposition \ref{prop*lin_step_func}, we can find a decomposition $a = x_0 < x_1 <\; \hdots \; < x_n = b$ such that both $f$ and $g$ are constant on each interval $(x_{k-1}, x_k)$. Writing $c_k$ and $d_k$ for their respective values, the assumption $f \leq g$ implies that $c_k \leq d_k$ for all $k = 1, \hdots , n$. Hence
	\[
		\int_{a}^{b} f(x) \, dx = \sum_{k=1}^{n} c_k (x_k - x_{k-1}) \leq \sum_{k=1}^{n} d_k (x_k - x_{k-1}) = \int_{a}^{b} g(x) \, dx. \qedhere
	\]
\end{proof}

Applying Proposition \ref{prop*mono_int_step_func} with $g \equiv 0$, we deduce the following corollary.

\begin{corollary}{Positivity of the Integral of Step Functions}{pos_int_step_func}
	If $f:[a,b] \to \mathbb{R}$ is a step functions such that $f(x) \geq 0$ for all $x \in [a,b]$, then
	\[
		\int_{a}^{b} f(x) \, dx \geq 0.
	\]
\end{corollary}

\subsection{Definition and First Properties of the Riemann Integral}
As in the last section, we consider functions on a compact interval $[a,b] \subseteq \mathbb{R}$. To alleviate notation, we write $\mathcal{SF}$ for the set of step functions on $[a,b]$. Also, we often write $\int_{a}^{b} f\, dx$ in place of $\int_{a}^{b} f(x) \, dx$.

\subsubsection{Integrability of Real-Valued Functions}
Before defining lower and upper sums, we recall a simple but useful property of the supremum and infimum of two related sets. We use this fact several times in what follows.

\begin{definition}{Relation Between Supremum and Infimum}{relation_sup_inf}
	Let $A, B \subseteq \mathbb{R}$ be non-empty sets such that $s \leq t$ for all $s \in A$ and $t \in B$. Then
	\begin{equation}
		\label{eq:relation_sup_inf}
		\sup A \leq \inf B.
	\end{equation}
	Moreover,
	\begin{equation}
		\label{eq:relation_sup_inf_2}
		\sup A = \inf B \quad \Leftrightarrow \quad \forall \varepsilon > 0 \; \exists s \in A \, \exists t \in B \text{ such that }t - s < \varepsilon.
	\end{equation}
\end{definition}


\begin{definition}{Lower and Upper Sums}{lower_upper_sums}
	Let $f:[a, b] \to \mathbb{R}$ be a function. Define the sets of \textbf{lower sums} $\mathcal{L}(f) \subseteq \mathbb{R}$ and \textbf{upper sums} $\mathcal{U}(f) \subseteq \mathbb{R}$ by 
	\[
		\mathcal{L}(f) = \biggl\{\int_{a}^{b} \ell \, dx \; \bigg|\; \ell \in \mathcal{SF} \text{ and } \ell \leq f\biggr\}, \qquad \mathcal{U}(f) = \biggl\{\int_{a}^{b} u \, dx \;\bigg|\; u \in \mathcal{SF} \text{ and } f \leq u\biggr\}.
	\]
\end{definition}
If $f$ is bounded, then these sets are non-empty. Indeed, if $|f| \leq M$, then the constant step functions
\[
	\ell(x) = -M \quad \forall x \in [a,b], \qquad u(x) = M \quad \forall x\in [a,b],
\]
satisfy $\ell \in \mathcal{L}(f)$ and $u \in \mathcal{U}(f)$.

For $\ell, u \in \mathcal{SF}$ with $\ell \leq f \leq u$, Proposition \ref{prop*mono_int_step_func} gives
\[
	\int_{a}^{b} \ell \, dx \leq \int_{a}^{b } u \, dx.
\]
This implies that $s \leq t$ for all $s \in \mathcal{L}(f)$ and $t \in \mathcal{U}(f)$, so Equation \eqref{eq:relation_sup_inf} yields
\[
	\sup \mathcal{L}(f) \leq \inf \mathcal{U}(f).
\]

\begin{definition}{Riemann Integral}{riemann_integral}
	A bounded function $f:[a,b] \to \mathbb{R}$ is \textbf{Riemann integrable} if $\sup \mathcal{L}(f) = \inf \mathcal{U}(f)$. In this case, this common value is called the \textbf{Riemann integral} of $f$, and we write
	\[
		\int_{a}^{b} f \, dx = \sup \mathcal{L}(f) = \inf \mathcal{U}(f).
	\]
	We call $a$ the \textbf{lower (integration) limit} and $b$ the \textbf{upper (integration) limit}, and the function $f$ the \textbf{integrand} of the integral $\int_{a}^{b} f\, dx$. If $f \geq 0$ is Riemann integrable, we interpret the number $\int_{a}^{b} f\, dx$ as the \textbf{area} of the set
	\[
		\{(x, y) \in \mathbb{R}^2 \;|\; a \leq x \leq b, \quad 0 \leq y \leq f(x)\}.
	\]
\end{definition}

\begin{proposition}{Riemann Integrability Condition}{int_cond}
	Let $f:[a,b] \to \mathbb{R}$ be bounded. Then $f$ is Riemann integrable if and only if for every $\varepsilon > 0$ there exists step functions $\ell, u \in \mathcal{SF}$ such that
	\[
		\ell \leq f \leq u \qquad \text{and} \qquad \int_{a}^{b} (u - \ell) < \varepsilon.
	\]
	In this case,
	\[
		\left|\int_{a}^{b} f \, dx - \int_{a}^{b} \ell \, dx\right| < \varepsilon, \qquad \left|\int_{a}^{b} u \, dx - \int_{a}^{b} f \, dx\right| < \varepsilon.	
	\]
\end{proposition}

\begin{proof}
	By Equation \eqref{eq:relation_sup_inf_2} applied with $A = \sup \mathcal{L}(f)$ and $B = \inf \mathcal{U}(f)$ we obtain
	\begin{align*}
		f \text{ is Riemann integrable} \quad &\Leftrightarrow \quad \sup \mathcal{L}(f) = \mathcal{U}(f)\\
		&\Leftrightarrow \quad \forall \varepsilon > 0 \; \exists s \in \mathcal{L}(f) \, \exists t \in \mathcal{U}(f):\; t - s < \varepsilon\\
		&\Leftrightarrow \quad \forall \varepsilon > 0\; \exists \ell, u \in \mathcal{SF}: \; \ell \leq f \leq u \text{ and } \int_{a}^{b} u \, dx - \int_{a}^{b} \ell \, dx < \varepsilon\\
		&\Leftrightarrow \quad \forall \varepsilon > 0\; \exists \ell, u \in \mathcal{SF}: \; \ell \leq f \leq u \text{ and } \int_{a}^{b} (u - \ell) \, dx < \varepsilon,
	\end{align*}
	where we used Proposition \ref{prop*lin_int_step_func} to deduce that
	\[
		\int_{a}^{b}u \, dx - \int_{a}^{b} \ell \, dx = \int_{a}^{b}(u-\ell)\, dx
	\]
	Finally, the concluding inequalities follow from
	\[
		\int_{a}^{b} \ell \, dx \leq \int_{a}^{b} f\, dx \leq \int_{a}^{b} u \, dx \qquad \text{and} \qquad \int_{a}^{b}(u - \ell )\, dx < \varepsilon. \qedhere
	\]
\end{proof}

\subsubsection*{Example}
Not all functions are Riemann integrable. Indeed, consider the function $f:[0,1] \to \mathbb{R}$ defined by
\[
	f(x) = \begin{cases}
		1, \qquad &x \in \mathbb{Q},\\
		0, \qquad &x \notin \mathbb{Q}.
	\end{cases}
\]
We claim that $f$ is not Riemann integrable.

Let $u \in \mathcal{SF}$ with $f \leq u$, and let $0 = x_0 < \; \hdots \; < x_1 = 1$ be a decomposition such that $u$ is constant $c_k$ on $(x_{k-1}, x_{k})$. Since $\mathbb{Q}$ is dense in $\mathbb{R}$, there exists $x \in (x_{k-1}, x_k) \cap \mathbb{Q}$, hence $1 = f(x) \leq u(x) = c_k$, so $c_k \geq 1$. Therefore,
\[
	\int_{0}^{1} u(x)\, dx = \sum_{k=1}^{n} c_k (x_k - x_{k-1}) \geq \sum_{k=1}^{n} (x_k - x_{k-1}) = x_n - x_0 = 1.
\]
Thus $\inf \mathcal{U}(f) = 1$, and taking $u \equiv 1$ gives $\inf \mathcal{U}(f)$. A similar argument with lower sums shows that $\sup \mathcal{L}(f) = 0$. Hence, $f$ is not Riemann integrable.

\begin{theorem}{Linearity of the Riemann Integral}{lin_int}
	If $f,g:[a,b] \to \mathbb{R}$ are integrable and $\alpha , \beta \in \mathbb{R}$, then $\alpha f + \beta g$ is integrable, and
	\[
		\int_{a}^{b} (\alpha f + \beta g) \, dx = \alpha \int_{a}^{b} f \, dx + \beta \int_{a}^{b} g\, dx
	\]
\end{theorem}

\begin{proof}
	Given $\varepsilon > 0$, Proposition \ref{prop*int_cond} yields step functions $\ell_1, \ell_2, u_1, u_2$ with
	\[
		\ell_1 \leq f \leq u_1, \quad \ell_2 \leq g \leq u_2,  \quad \int_{a}^{b} (u_1 - \ell_1) \, dx < \varepsilon, \quad \int_{a}^{b}(u_2 - \ell_2) \, dx < \varepsilon,
	\]
	and
	\[
		\left|\int_{a}^{b}f \, dx - \int_{a}^{b} \ell_1 \, dx\right| < \varepsilon, \qquad \left|\int_{a}^{b} g \, dx - \int_{a}^{b} \ell_2 \, dx \right| < \varepsilon.
	\]
	
	Assume first $\alpha, \beta \geq 0$. Then
	\[
		\alpha \ell_1 + \beta \ell_2 \leq \alpha f + \beta g \leq \alpha u_1 + \beta u_2,
	\]
	and
	\[
		\int_{a}^{b}[(\alpha u_1 + \beta u_2) - (\alpha \ell_1 + \beta \ell_2)] \, dx = \alpha \int_{a}^{b} (u_1 - \ell_1) \, dx + \beta \int_{a}^{b} (u_2 - \ell_2) \, dx < (\alpha + \beta) \varepsilon.
	\]
	Since $\varepsilon > 0$ is arbitrary, this proves that $\alpha f + \beta g$ is integrable. Moreover, by the triangle inequality and Proposition \ref{prop*lin_int_step_func}, we have that
	\begin{align*}
		\bigg|\int_{a}^{b} (\alpha f + \beta g) \, dx - \alpha \int_{a}^{b} f\, dx - \beta \int_{a}^{b} g \, dx\bigg| &\leq \bigg|\int_{a}^{b} (\alpha f + \beta g)\, dx - \int_{a}^{b} (\alpha \ell_1 + \beta \ell_2) \, dx\bigg|\\
		&+ \bigg|\underbrace{\int_{a}^{b} (\alpha \ell_1 + \beta \ell_2) \, dx - \alpha \int_{a}^{b} \ell_1\, dx - \beta \int_{a}^{b} \ell_2 \, dx}_{=0}\bigg|\\
		&+ \alpha \bigg|\int_{a}^{b} \ell_1 \, dx - \int_{a}^{b} f\, dx\bigg| + \beta \bigg|\int_{a}^{b} \ell_2 \, dx - \int_{a}^{b} g\, dx\bigg|\\
		&\leq (\alpha + \beta) \varepsilon + \alpha \varepsilon + \beta \varepsilon = 2 (\alpha + \beta) \varepsilon.
	\end{align*}
	Since $\varepsilon > 0$ is arbitrary, the linearity follows.
	
	The case when one of $\alpha, \beta$ is negative is analogous, but one needs to reverse the corresponding inequalities. For instance, if $\alpha \geq 0$ and $\beta < 0$, then
	\[
		\alpha \ell_1 + \beta u_2 \leq \alpha f + \beta g \leq \alpha u_1 + \beta \ell_2,
	\]
	and
	\[
		\int_{a}^{b}[(\alpha u_1 + \beta \ell_2) - (\alpha \ell_1 + \beta u_2)]\, dx = \alpha \int_{a}^{b} (u_1 - \ell_1) \, dx + |\beta| \int_{a}^{b} (u_2 - \ell_2) \, dx < (\alpha + |\beta|)\varepsilon.
	\]
	This implies again that $\alpha f + \beta g$ is integrabl, and the linearity identity holds similarly. \qedhere
\end{proof}

\begin{proposition}{Monotonicity of the Riemann Integral}{mono_int}
	Let $f,g:[a,b] \to \mathbb{R}$ be integrable. If $f \leq g$, then
	\[
		\int_{a}^{b} f \, dx \leq \int_{a}^{b} g \, dx.
	\]
\end{proposition}
	
\begin{proof}
	Since $f \leq g$, for any step function $\ell$ with $\ell \leq f$ we have $\ell \leq g$. This implies that $\mathcal{L}(f) \subseteq \mathcal{L}(g)$, therefore
	\[
		\int_{a}^{b} f\, dx = \sup \mathcal{L}(f) \leq \sup \mathcal{L}(g) = \int_{a}^{b} g\, dx. \qedhere
	\]
\end{proof}

\begin{definition}{Positive and Negative Parts}{pos_neg_parts}
	Given a function $f:D \to \mathbb{R}$, we define its \textbf{positive part} $f^+:D \to \mathbb{R}$ and \textbf{negative part} $f^-:D \to \mathbb{R}$ by
	\[
		f^+(x) = \max\{0, f(x)\}, \qquad f^-(x) = -\min\{0, f(x)\}.
	\]
	These satisfy
	\[
		f = f^+ - f^-, \quad |f| = f^+ + f^-, \quad f^+ = \frac{|f| + f}{2}, \quad f^- = \frac{|f| - f}{2}.
	\]
	Moreover, for any functions $f, g:D \to \mathbb{R}$,
	\[
		f \leq g \quad \Rightarrow \quad f^+ \leq g^+, \qquad f \leq g \quad \Rightarrow \quad f^- \geq g^-.
	\]
\end{definition}

\begin{remark}
	\label{rmk:prop_pos_part}
	For any real numbers $z_1, z_2 \in \mathbb{R}$, one has
	\begin{equation}
		\label{eq:ineq_pos_part}
		(z_1 - z_2)^+ \geq z_1^+ - z_2^+.
	\end{equation}
	Indeed, since $z^+ \geq z$ and $z^+ \geq 0$ for all $z \in \mathbb{R}$, by applying these inequalities with $z = z_1 - z_2$ and $z = z_2$ we obtain
	\[
		z_1 = (z_1 - z_2) + z_2 \leq (z_1 - z_2)^+ + z_2^+ \quad \text{and} \quad 0 \leq (z_1 - z_2)^+ + z_2^+.
	\]
	Hence $(z_1 - z_2)^+ + z_2^+$ is greater or equal to both $z_1$ and 0, and therefore
	\[
		z_1^+ = \max\{z_1, 0\} \leq (z_1 - z_2)^+ + z_2^+,
	\]
	which yields Equation \eqref{eq:ineq_pos_part} after rearranging.
\end{remark}

\begin{theorem}{Triangle Inequality for the Riemann Integral}{triangle_ineq_int}
	Let $f,g:[a,b] \to \mathbb{R}$ be integrable. Then $f^+$, $f^-$, and $|f|$ are integrable, and
	\[
		\left|\int_{a}^{b} f \, dx\right| \leq \int_{a}^{b} |f|\, dx.
	\]
\end{theorem}

\begin{proof}
	Fix $\varepsilon > 0$. Since $f$ is integrable, there exists step functions $\ell \leq f \leq u$ with $\int_{a}^{b}(u - \ell)\, dx < \varepsilon$. Then $\ell^+$ and $u^+$ are step functions with $\ell^+ \leq f^+ \leq u^+$.
	
	Since $u - \ell \geq 0$, we have $(u - \ell) = (u - \ell)^+$. Moreover, applying Equation \eqref{eq:ineq_pos_part} with $z_1 = u(x)$ and $z_2 = \ell(x)$, we obtain
	\[
		(u(x) - \ell(x))^+ \geq u(x)^+ - \ell(x)^+ \qquad \forall x \in [a,b].
	\]
	Hence
	\[
		\int_{a}^{b}(u^+ - \ell^+) \, dx \leq \int_{a}^{b} (u - \ell)^+\, dx = \int_{a}^{b} (u - \ell) \, dx < \varepsilon,
	\]
	so $f^+$ is integrable. By Theorem \ref{theo*lin_int}, also $f^- = f^+ - f$ and $|f| = 2f^+ - f$ are integrable. Finally,
	\[
		\left|\int_{a}^{b} f\, dx\right| = \left|\int_{a}^{b}f^+ \, dx - \int_{a}^{b} f^- \, dx\right| \leq \int_{a}^{b} f^+ \, dx + \int_{a}^{b} f^- \, dx = \int_{a}^{b} |f| \, dx. \qedhere
	\]
\end{proof}

\begin{remark}
	\label{rmk:int_seperate}
	Let $a < b < c$. A function $f:[a,c] \to \mathbb{R}$ is integrable if and only if $f|_{[a,b]}$ and $f|_{[b,c]}$ are integrable, and
	\[
		\int_{a}^{c} f \, dx = \int_{a}^{b} f|_{[a,b]} \, dx + \int_{b}^{c} f|_{[b,c]} \, dx.
	\]
\end{remark}

\subsection{Integrability Theorems}

\subsubsection{Integrability of Monotone Functions}
As before we work on a compact interval $[a,b] \subseteq \mathbb{R}$. Note that very monotone function $f:[a,b] \to \mathbb{R}$ is bounded; for instance, if $f$ is increasing then $f(a)$ is a lower bound and $f(b)$ is an upper bound.

\begin{theorem}{Monotone Functions are Integrable}{mono_func_int}
	Every monotone function $f:[a,b] \to \mathbb{R}$ is Riemann integrable.
\end{theorem}

\begin{proof}
	W.l.o.g, $f$ is increasing (otherwise replace $f$ by $-f$ and use Theorem \ref{theo*lin_int}). We want to apply Proposition \ref{prop*int_cond}. Given $\varepsilon > 0$, we need to construct step functions $\ell, u \in \mathcal{SF}$ such that $\ell \leq f \leq u$ and $\int_{a}^{b}(u - \ell) <, dx < \varepsilon$.
	
	Fix $n \in \mathbb{N}$ (to be chosen later) and the uniform partition
	\[
		a = x_0 < x_1 < \; \hdots \; < x_n = b, \qquad x_k = a + \frac{k}{n}(b - a).
	\]
	Define the step functions $ell, u:[a,b] \to \mathbb{R}$ as
	\begin{align*}
		\ell(x) = f(x_{k-1}) \quad \text{and} \quad u(x) = f(x_k) \qquad &\text{for } x \in (x_{k-1}, x_k), \quad k = 1, \hdots, n,\\
		\ell(x) = u(x) = f(x) \qquad &\text{for } x \in \{x_0, \hdots , x_n\}.
	\end{align*}
	Note that, since $f$ is increasing, $\ell \leq f \leq u$. Moreover, for each $k$ we have $u - \ell = f(x_k) - f(x_{k-1})$ on $(x_{k-1}, x_k)$. Recalling that $x_k - x_{k-1} = \frac{b - a}{n}$, this yields
	\begin{align*}
		\int_{a}^{b} (u - \ell) \, dx &= \sum_{k=1}^{n} (f(x_k) - f(x_{k-1})) (x_k - x_{k-1}) = \frac{b - a}{n} \sum_{k=1}^{n} (f(x_k) - f(x_{k-1})) \\
		&= \frac{b - a}{n} (f(x_n) - f(x_0)) = \frac{b - a}{n} (f(b) - f(a)).
	\end{align*}
	Choosing $n \in \mathbb{N}$ so large that $\frac{b - a}{n}(f(b) - f(a)) < \varepsilon$, Proposition \ref{prop*int_cond} implies that $f$ is Riemann integrable. \qedhere
\end{proof}

\begin{definition}{Piecewise Monotone Functions}{piecewise_mono_func}
	A function $f:[a,b] \to \mathbb{R}$ is \textbf{piecewise monotone} if there exists a decomposition
	\[
		a = x_0 < x_1 < \; \hdots \; < x_n = b
	\]
	such that $f|_{(x_{k-1}, x_k)}$ is monotone for every $k = 1, \hdots , n$.
\end{definition}

\begin{corollary}{Piecewise Monotone Functions are Integrable}{piecewise_mono_func_int}
	Every bounded piecewise monotone function $f:[a,b] \to \mathbb{R}$ is Riemann integrable.
\end{corollary}

\subsubsection{Integrability of Continuous Functions}
Using boundedness and uniform continuity on compact intervals (Theorems \ref{theo*boundedness} and \ref{theo*unif_cont_compact_int}), we can prove that continuous functions are integrable.

\begin{theorem}{Continuous Functions are Integrable}{cont_func_int}
	Every continuous function $f:[a,b] \to \mathbb{R}$ is Riemann integrable.
\end{theorem}

\begin{proof}
	Let $f:[a,b] \to \mathbb{R}$ be continuous and fix $\varepsilon > 0$. By uniform continuity \ref{theo*unif_cont_compact_int}, there exists $\delta >0$ such that
	\begin{equation}
		\label{eq:unif_cont}
		|x - y| < \delta \quad \Rightarrow \quad |f(x) - f(y)|< \varepsilon \qquad \forall x,y \in [a,b].
	\end{equation}
	Choose a partition $a = x_0 < x_1 < \; \hdots \; < x_n = b$ with $x_k - x_{k-1} < \delta$. For each $k$ set 
	\[
		c_k = \min\{f(x) \;|\; x_{k-1} \leq x \leq x_k\}, \qquad d_k = \max\{f(x) \,|\; x_{k-1} \leq x \leq x_k\},
	\]
	which exists by Theorem \ref{theo*extreme_val_theo}, and let $y_k, z_k \in [x_{k-1}, x_k]$ satisfy $f(y_k) = c_k$ and $f(z_k) = d_k$. Then since, $|y_k - z_k| \leq x_k - x_{k-1} < \delta$, Equation \eqref{eq:unif_cont} yields $d_k - c_k < \varepsilon$.
	
	Define now the step functions $\ell, u:[a,b] \to \mathbb{R}$ as
	\begin{align*}
		\ell(x) = c_k, \quad \text{and} \quad u(x) = d_k \qquad &\text{for } x \in (x_{k-1}, x_k), \quad k = 1,\hdots , n,\\
		\ell(x) = u(x) = f(x) \qquad &\text{for } x \in \{x_0, \hdots , x_n\}.
	\end{align*}
	Then $\ell \leq f \leq u$ and $u - \ell = d_k - c_k$ on $(x_{k-1} , x_k)$, hence
	\[
		\int_{a}^{b} (u - \ell) \, dx = \sum_{k=1}^{n} (d_k - c_k)(x_k - x_{k-1}) < \varepsilon \sum_{k=1}^{n} (x_k - x_{k-1}) = \varepsilon (b - a).
	\]
	Since $\varepsilon > 0$ is arbitrary, $f$ is integrable. \qedhere
\end{proof}

\begin{definition}{Piecewise Continuous Functions}{piecewise_cont_func}
	A function $f:[a,b] \to \mathbb{R}$ is \textbf{piecewise continuous} if there exists a decomposition
	\[
		a = x_0 < x_1 < \; \hdots \; < x_n = b
	\]
	such that $f|_{(x_{k-1}, x_k)}$ is continuous for all $k$ and both one sided limits $\lim_{x \to x_{k-1}^+} f(x)$ and $\lim_{x \to x_k^-} f(x)$ exist. Equivalently, each $f|_{(x_{k-1}, x_k)}$ extends to a continuous function on $[x_{k-1}, x_k]$.
\end{definition}

\begin{corollary}{Piecewise Continuous Functions are Integrable}{piecewise_cont_func_int}
	Every piecewise continuous function $f:[a,b] \to \mathbb{R}$ is Riemann integrable.
\end{corollary}

\subsubsection{Integration and Sequences of Functions}
Let $(f_n)_{n=0}^{\infty}$ with $f_n:[a,b] \to \mathbb{R}$ be a sequence of integrable functions. Assume that $f_n$ converges pointwise or uniformly to $f:[a,b] \to \mathbb{R}$. Is $f$ integrable? And if so, does
\[
	\lim_{n \to \infty} \int_{a}^{b} f_n \, dx = \int_{a}^{b} f\, dx
\]
hold?

In general, the pointwise limit of integrable functions need not to be integrable. Also, as the following example shows, even when the pointwise limit of $f$ is integrable, one may have that $\lim_{n \to \infty} \int f_n \neq \int f$.

\subsubsection*{Example}
Let $D = [0,1]$ and define $f_n:D \to \mathbb{R}$ by
\[
	f_n(x) = \begin{cases}
		n^2x \qquad &\text{if } 0 \leq x \leq \frac{1}{2n},\\
		n^2(\frac{1}{n} - x) \qquad &\text{if } \frac{1}{2n} \leq x \leq \frac{1}{n},\\
		0 \qquad &\text{if } \frac{1}{n} \leq x \leq 1.
	\end{cases}
\]
Each $f_n$ is continuous (hence integrable). Also, its graph is a triangle of base $\frac{1}{n}$ and height $\frac{n}{2}$, so 
\[
	\int_{0}^{1} f_n(x) \, dx = \frac{1}{2}\cdot \frac{1}{n} \cdot \frac{n}{2} = \frac{1}{4}.
\]
Moreover, $f_n(0) = 0$ for all $n$, and for every $x > 0$ we have $f_n(x) = 0$ for all $n > 1/x$, hence $f_n(x) \to 0$. Thus, $f_n$ converges pointwise to the constant function $f = 0$, but
\[
	\int_{0}^{1}f_n(x) \, dx = \frac{1}{4} \neq 0 = \int_{0}^{1} f(x) \, dx.
\]

On the other hand, as the next result shows, uniform convergence is sufficient for both integrability of the limit and interchange of limit and integral.

\begin{theorem}{Uniform Convergence and Riemann Integrals Commute}{unif_conv_int_commute}
	Let $(f_n)_{n=0}^{\infty}$, with $f_n:[a,b] \to \mathbb{R}$, be a sequence of integrable functions converging uniformly to $f:[a,b] \to \mathbb{R}$. Then $f$ is integrable and
	\begin{equation}
		\label{eq:unif_conv_int}
		\int_{a}^{b}f \, dx = \lim_{n \to \infty} \int_{a}^{b} f_n \, dx.
	\end{equation}
\end{theorem}

\begin{proof}
	Fix $\varepsilon > 0$. By uniform convergence, there exists $N \in \mathbb{N}$ such that $|f_n - f| < \varepsilon$ on $[a,b]$ for all $n \geq N$.
	
	Since $f_N$ is integrable, there exists step functions $\ell, u \in \mathcal{SF}$ with $\ell \leq f_N \leq u$ and $\int_{a}^{b} (u - \ell) \, dx < \varepsilon$. Set $\hat{\ell} = \ell - \varepsilon$ and $\hat{u} = u + \varepsilon$. Then $\hat{\ell}, \hat{u} \in \mathcal{SF}$. Also, since $|f_N - f| < \varepsilon$,
	\[
	\hat{\ell} = \ell - \varepsilon \leq f_N - \varepsilon \leq f \leq f_N + \varepsilon \leq u + \varepsilon = \hat{u}
	\]
	and (because $\hat{u} - \hat{\ell} = u - \ell + 2\varepsilon$)
	\[
	\int_{a}^{b} (\hat{u} - \hat{\ell}) \, dx = \int_{a}^{b} (u - \ell) \, dx + 2\varepsilon(b-a) < \varepsilon + 2\varepsilon(b-a).
	\]
	As $\varepsilon > 0$ is arbitrary, Proposition \ref{prop*int_cond} yields that $f$ is integrable.
	
	Moreover, using monotonicity (Proposition \ref{prop*mono_int}) and the triangle inequality for the Riemann integral (Theorem \ref{theo*triangle_ineq_int}),
	\[
		\left|\int_{a}^{b}f \, dx - \int_{a}^{b}f_n \, dx\right| = \left|\int_{a}^{b} (f - f_n) \, dx\right| \leq \int_{a}^{b} |f - f_n|\, dx \leq \varepsilon (b-a) \qquad \forall n \geq N,
	\]
	proving Equation \eqref{eq:unif_conv_int}. \qedhere
\end{proof}
	%
% (c) 2025 Autor, ETH Zürich
%
% !TEX root = main.tex
% !TEX encoding = UTF-8
%

\section{The Derivative and the Riemann Integral}

\subsection{The Fundamental Theorem of Calculus}
Throughout this section we fix a compact interval $I \subseteq\mathbb{R}$ that is non-emtpy and contains more than one point. For brevity, we write \textit{integrable} for \textit{Riemann integrable}.

\subsubsection{The Fundamental Theorem}
\begin{definition}{Primitive Function}{primitive_func}
	Let $I \subseteq \mathbb{R}$ be an interval and $f:I \to \mathbb{R}$ a function. Any differentiable function $F : I \to \mathbb{R}$ such that $F' = f$ is called a \textbf{primitive} (or \textbf{antiderivative}) of $f$.
\end{definition}

\begin{remark}
	A primitive may not always exist.
\end{remark}

The next result is known as the \textbf{Fundamental Theorem of (Integral and Differential) Calculus}, going back to Leibniz, Newton and Barrow.

\begin{theorem}{Fundamental Theorem of Calculus}{fund_thm_calc}
	Let $f:[a,b] \to \mathbb{R}$ be continuous. Then:
	\begin{enumerate}
		\item[(i)] For every $C \in \mathbb{R}$, the function $F:[a,b] \to \mathbb{R}$ defined by
		\begin{equation}
			\label{eq:primitive}
			F(x) = \int_{a}^{x} f(t) \, dt + C
		\end{equation}
		is a primitive of $f$.
		\item[(ii)] Every primitive $F:[a,b] \to \mathbb{R}$ of $f$ has the form \ref{eq:primitive} for some constant $C$.
	\end{enumerate}
\end{theorem}

\begin{proof}
	By Theorem \ref{theo*cont_func_int}, $f$ is integrable.
	
	Let $F$ be defined as in Equation \ref{eq:primitive}. To prove (i), we fix $x_0 \in [a,b]$ and we want to show that $F'(x_0) = f(x_0)$. To this aim, fix $\varepsilon > 0$. By continuity, there exists $\delta > 0$ such that
	\begin{equation}
		\label{eq:fund_cont}
		z \in [a,b], \quad |z - x_0| < \delta \quad \Rightarrow \quad |f(z) - f(x_0)| < \varepsilon.
	\end{equation}
	Now, given $x \in (x_0, x_0 + \delta) \cap [a,b]$, it follows from Remark \ref{rmk:int_seperate} that
	\begin{align*}
		\left|\frac{F(x) - F(x_0)}{x - x_0} - f(x_0)\right| &= \left|\frac{1}{x - x_0} \bigg(\int_{a}^{x} f(t) \, dt - \int_{a}^{x_0} f(t) \, dt\bigg) - f(x_0)\right|\\
		&= \left|\frac{1}{x- x_0} \int_{x_0}^{x} f(t) \, dt - f(x_0)\right|.
	\end{align*}
	Also,
	\[
		f(x_0) = f(x_0) \frac{1}{x - x_0} \int_{x_0}^{x} \, dt = \frac{1}{x - x_0} \int_{x_0}^{x} f(x_0) \, dt.
 	\]
 	Combining these two equations and using Theorem \ref{theo*triangle_ineq_int}, we get
 	\begin{align*}
 		\left|\frac{F(x) - F(x_0)}{x - x_0} - f(x_0)\right| &= \left|\frac{1}{x - x_0} \int_{x_0}^{x} f(t) \, dt - \frac{1}{x - x_0} \int_{x_0}^{x} f(x_0)\, dt\right|\\
 		&= \left|\frac{1}{x - x_0} \int_{x_0}^{x} (f(t) - f(x_0)) \, dt\right| \\
 		&\leq \frac{1}{x - x_0} \int_{x_0}^{x} |f(t) - f(x_0)| \, dt.
 	\end{align*}
 	Note now that, in the last integral, $t \in [x_0, x] \subseteq [x_0, x_0 + \delta) \cap [a,b]$. Hence, it follows form Equation \ref{eq:fund_cont} that $|f(t) - f(x_0)| < \varepsilon$, therefore
 	\[
 		\left|\frac{F(x) - F(x_0)}{x - x_0} - f(x_0)\right| < \frac{1}{x - x_0} \int_{x_0}^{x} \varepsilon \, dt = \varepsilon. 
 	\]
 	Similarly, if $x \in (x_0 - \delta, x_0) \cap [a,b]$, then
 	\[
 		\left|\frac{F(x) - F(x_0)}{x - x_0} - f(x_0)\right| = \left|\frac{1}{x_0 - x} \int_{x}^{x_0} (f(t) - f(x_0)) \, dt\right| \leq \frac{1}{x_0 - x} \int_{x}^{x_0} |f(t) - f(x_0)| \, dt < \varepsilon.
 	\]
 	In summary, we proved that
 	\[
 		x \in [a,b], \quad |x - x_0| < \delta \quad \Rightarrow \quad \left|\frac{F(x) - F(x_0)}{x - x_0} - f(x_0)\right| < \varepsilon,
 	\]
 	therefore
 	\[
 		F'(x_0) = \lim_{x \to x_0} \frac{F(x) - F(x_0)}{x - x_0} = f(x_0),
 	\]
 	as desired.
 	
 	We now prove (ii). Let $F$ be a primitive of $f$. Then, since $\left(\int_{a}^{x} f(t) \, dt\right)' = f(x)$ (by (i)), we have that
 	\[
 		\left(F(x) - \int_{a}^{x} f(t)\, dt\right)' = F'(x) - f(x) = f(x) - f(x) = 0 \qquad  \forall x \in (a,b).
 	\]
 	By Corollary \ref{cor*const_deriv}, this implies that $F(x) - \int_{a}^{x} f(t)\, dt$ is constant on $[a,b]$, concluding the proof of (ii). \qedhere
\end{proof}

\begin{corollary}{Integral vs. Derivative}{int_vs_deriv}
	If $F:[a,b] \to \mathbb{R}$ is continuously differentiable, then for all $x \in [a,b]$,
	\[
		F(x) = F(a) + \int_{a}^{x} F'(t)\, dt.
	\]
\end{corollary}

\begin{proof}
	Since $F$ is a primitive of $F'$, Theorem \ref{theo*fund_thm_calc} yields $F(x) = \int_{a}^{x} F'(t)\, dt + C$. Evaluating at $x = a$ gives $C = F(a)$. \qedhere
\end{proof}

\begin{corollary}{Riemann Integral and Primitives}{int_primitive}
	If $f:[a,b] \to \mathbb{R}$ is continuous and $F$ is a primitive of $f$, then
	\[
		\int_{a}^{b} f(t)\, dt = F(b) - F(a).
	\]
\end{corollary}

\begin{proof}
	Apply Corollary \ref{cor*int_vs_deriv} with $F' = f$ and $x = b$. \qedhere
\end{proof}

\subsubsection{Integration by Parts and Substitution}
Given a Function $h:[a,b] \to \mathbb{R}$, we use the notation $\big[h(x)\big]_a^b := h(b) - h(a)$.

\begin{theorem}{Integration by Parts}{int_parts}
	If $f,g:[a,b] \to \mathbb{R}$ are continuously differentiable, then
	\[
		\int_{a}^{b} f(x) g'(x) \, dx = \big[f(x)g(x)\big]_a^b - \int_{a}^{b} f'(x)g(x)\, dx.
	\]
\end{theorem}

\begin{proof}
	By Proposition \ref{prop*prod_rule}, $(fg)' = f'g + fg'$. Rearranging and integrating, thanks to Corollary $\ref{cor*int_primitive}$, we get
	\[
		\int_{a}^{b} fg'\, dx = \int_{a}^{b} (fg)'\, dx - \int_{a}^{b}f g'\, dx = \big[fg\big]_a^b - \int_{a}^{b} fg'\, dx. \qedhere
	\]
\end{proof}

As a convention, for any $h:[a,b] \to \mathbb{R}$,
\begin{equation}
	\label{eq:convention_int}
	\int_{b}^{a} h(x)\, dx = -\int_{a}^{b} h(x)\, dx.
\end{equation}

\begin{theorem}{Integration by Substitution, 1st Form}{int_sub1}
	Let $I,J \subseteq \mathbb{R}$ be intervals, $f:I \to J$ be continuously differentiable, and $g:J \to \mathbb{R}$ be continuous. For any $[a,b] \subseteq I$,
	\[
		\int_{a}^{b} g(f(x))f'(x) \, dx = \int_{f(a)}^{f(b)} g(y) \, dy.
	\]
\end{theorem}

\begin{proof}
	Fix $y_0 \in J$ and set $G(y) = \int_{y_0}^{y} g(t)\, dt$. Since $G' = g$, by the chain rule (Theorem \ref{theo*chain_rule}) we get $(G \circ f)' = G'(f)f' = g(f)f'$. Integrating this identity and using Corollary \ref{cor*int_primitive} yields
	\begin{align*}
		\int_{a}^{b} g(f(x))f'(x)\, dx &= \int_{a}^{b} (G\circ f)'(x)\, dx = G(f(b)) - G(f(a))\\
		&= \int_{y_0}^{f(b)} g(t)\, dt - \int_{y_0}^{f(a)} g(t)\, dt = \int_{f(a)}^{f(b)} g(t)\, dt.\qedhere
	\end{align*}
\end{proof}

Before stating the nect result, we note the following: If $h:[a,b]\to \mathbb{R}$ is continuously differentiable with $h' \neq 0$, then $h'$ has constant sign on $[a,b]$, so $h$ is strictly monotone and invertible ; $h^{-1}$ is continuous by Theorem \ref{theo*inv_func_theo} and differentiable on $(h(a), h(b))$ by Theorem \ref{theo*derivative_inv}. Furthermore, since$(h^{-1})' = \frac{1}{h'\circ h^{-1}}$, also $(h^{-1})'$ is continuous.

\begin{theorem}{Integration by Substitution, 2nd Form}{int_sub2}
	Let $I,J \subseteq \mathbb{R}$ be intervals, $f:I \to J$ be $C^1$, and $g:J \to \mathbb{R}$ be continuous. Let $[a,b] \subseteq I$ and assume $f'(x) \neq 0$ on $[a,b]$. If $f^{-1}:[f(a), f(b)] \to \mathbb{R}$ denotes the inverse of $f|_{[a,b]}$, then
	\[
		\int_{a}^{b} g(f(x))\, dx = \int_{f(a)}^{f(b)} g(y)(f^{-1})'(y) \, dy.
	\]
\end{theorem}

\begin{proof}
	In order to apply Theorem \ref{theo*int_sub1}, we first observe that
	\[
		\int_{a}^{b} g(f(x))\, dx = \int_{a}^{b} \frac{g(f(x))}{f'(x)}f'(x)\, dx = \int_{a}^{b}\frac{g(f(x))}{f' \circ f^{-1}(f(x))}f'(x)\, dx.	
	\]
	So we can apply Theorem \ref{theo*int_sub1} with $\frac{g}{f' \circ f^{-1}}$ in place of $g$ to get
	\[
		\int_{a}^{b} g(f(x)) \, dx = \int_{f(a)}^{f(b)} \frac{g(y)}{f'(f^{-1}(y))}\, dy.
	\]
	Since $\frac{1}{f'\circ f^{-1}} = (f^{-1})'$, the result follows. \qedhere
\end{proof}

\subsubsection{Improper Integrals}
A function $f:I \to \mathbb{R}$ is \textbf{locally integrable} if $f|_{[a,b]}$ is integrable for every compact interval $[a,b]\subseteq I$.

\begin{definition}{Improper Integrals}{improper_int}
	Let $I \subseteq \mathbb{R}$ be a non-empty interval and $f:I \to \mathbb{R}$ be locally integrable. Set $c = \inf I \in \mathbb{R}\cup \{-\infty\}$ and $d = \sup I \in \mathbb{R}\cup \{\infty\}$, and fix $x_0 \in I$. We define the \textbf{improper integral} of $f$ on $I$ by
	\[
		\int_{c}^{d} f(x)\, dx := \lim_{a \to c^+}\int_{a}^{x_0}f(x) \, dx + \lim_{b \to d^-} \int_{x_0}^{b} f(x)\, dx,
	\]
	whenever both limits exist and the sum is well-defined (we do not allow the indeterminate form $\infty - \infty$). Here the first limit is taken over $a \in I$ with $c < a < x_0$ (interpreting $a \to -\infty$ if $c = -\infty$) and the second over $b \in I$ with $x_0 < b < d$ (interpreting $b \to +\infty$ if $d = \infty$). If the value is finite we say the integral \textit{converges}; if it is $\pm \infty$ we say it \textit{diverges to} $\pm \infty$; otherwise, it \textit{does not converge}. When defined, the value is independent of the choice of $x_0$. 
\end{definition}


	%
% (c) 2025 Autor, ETH Zürich
%
% !TEX root = main.tex
% !TEX encoding = UTF-8
%

\section{Ordinary Differential Equations}

\subsection{Ordinary Differential Equations (ODEs)}
In this section we study \textit{ordinary differential equations} (ODEs). They describe how an unknown quantity depends on one real variable (often denoted by $x$ or $t$) and how this dependence is constrained by relations involving derivatives.

It is convenient to fix some notation from the beginning:
\begin{itemize}
	\item[$\bullet$] we usually denote the \textit{unknown function} by $u$;
	\item[$\bullet$] the independent variable is denoted by $x$ (or by $t$ when it represents time);
	\item[$\bullet$] letters like $f$, $g$, $a_0$, $a_1$ will typically denote given functions or constants appearing in the equation (data of the problem).
\end{itemize}
Although derivatives are usually denoted using $'$ (so $u'$, $u''$, etc.), it is common to use a dot to denote derivatives with respect to time (so $\dot{u}$, $\ddot{u}$, etc.).

\begin{definition}{ODEs}{ode}
	An \textbf{ordinary differential equation (ODE)} is a relation involving a function $u:\mathbb{R}\to \mathbb{R}$ of a real variable $x \in \mathbb{R}$ and its derivatives. The general form of an $n$-th order ODE is
	\begin{equation}
		\label{eq:ode}
		G(x, u(x), u'(x), u''(x), \hdots , u^{(n)}(x)) = 0,
	\end{equation}
	where $G:\mathbb{R}^{n+2} \to \mathbb{R}$ is a given function.
\end{definition}

In many examples the independent variable is time $t$ and the equation describes the evolution of a system, but we keep the generic notation $x$ unless we want to stress the time interpretation.

ODEs can be classified according to several criteria:
\begin{enumerate}
	\item \textit{Order:} An ODE is of order $n$ if $u^{(n)}$ is the highest derivative appearing in the equation. For instance:
	\begin{enumerate}
		\item $u'' + u = 0 \; \rightsquigarrow$ second order.
		\item $u^{(3)} = x^2u + x \; \rightsquigarrow$ third order.
		\item $(u')^2 + u - x^3 = 0 \; \rightsquigarrow$ first order.
	\end{enumerate}
	\item \textit{Linearity:} An ODE is \textit{linear} if it is linear in $u$ and its derivatives. Otherwise, it is \textit{nonlinear}. Here ''linear'' means that $u, u', u'', \hdots$ appears only to the first power and are not multiplied with each other.
	\begin{enumerate}
		\item $u'' + u = 0\; \rightsquigarrow$ linear.
		\item $u'' + u^2 = 0\; \rightsquigarrow$ nonlinear (because of $u^2$).
		\item $u'' + u' u = 0\; \rightsquigarrow$ nonlinear (because of the product $u' u$).
		\item $u^{(3)} = x^2 u + x \; \rightsquigarrow$ linear.
		\item $(u')^2 + u - x^3 = 0\; \rightsquigarrow$ nonlinear (because of $(u')^2$).
	\end{enumerate}
	\item \textit{Homogeneity (for linear ODEs):} For a linear ODE, we say it is \textit{homogeneous} if all terms involve the function or its derivatives. Equivalently, if $u$ is a solution then $Au$ is a solution for alll $A \in \mathbb{R}$. If there is an additional term that does not depend on $u$ (a ''forcing term''), the equation is \textit{non-homogeneous}.
	\begin{enumerate}
		\item $u'' + u = 0\; \rightsquigarrow$ homogeneous.
		\item $u^{(3)} = x^2u + x\; \rightsquigarrow$ non-homogeneous (because of the term $+x$).
		\item $u^{(3)} = x^2u\; \rightsquigarrow$ homogeneous.
	\end{enumerate}
\end{enumerate}

So far, we have only considered single equations, but one can also study \textit{systems} of ODEs with several unknown functions $u_1, \hdots, u_n$. We will not go into this now, but many ideas are similar.

In addition, solutions are often required to satisfy extra conditions such as $u(0) = 0$ (prescribed position at time 0) and/or $u'(0) = 1$ (prescribed velocity at time 0). When these conditions are imposed at a single time (typically $t = 0$), they are called \textbf{initial conditions}. More general conditions (for instance at two different points, such as $u(0) = 0$ and $u(1) = 1$) are called \textbf{boundary conditions}.

Later we shall see that, under suitable assumptions on the data (for example on a function $f$ appearing in the equation), prescribing initial conditions often leads to a unique solution. This is the content of the Cauchy-Lipschitz (or Picard-Lindelöf) theorem.

\subsubsection{Linear First Order ODEs}
We now consider linear first-order ODEs. Throughout this subsection we fix a non-empty interval $I \subseteq \mathbb{R}$ that is not a single point, and we study equations of the form
\[
	u'(x) + f(x)u(x) = g(x),
\]
where $f$ and $g$ are given continuous functions on $I$, and $u$ is the unknown.

We start with the homogeneous case $g \equiv 0$.

\begin{theorem}{Homogeneous Linear 1st Order ODEs}{homo_lin_1order_ode}
	Let $f:I \to \mathbb{R}$ be continuous and consider the homogeneous first-order linear ODE
	\begin{equation}
		\label{eq:homo_lin_1order_ode}
		u'(x) + f(x)u(x) = 0 \qquad \forall x \in I.
	\end{equation}
	Let $F:I \to \mathbb{R}$ be a primitive of $f$. Then all $C^1$ solutions $u:I \to \mathbb{R}$ of Equation \eqref{eq:homo_lin_1order_ode} are of the form
	\[
		u(x) = Ae^{-F(x)}, \qquad A \in \mathbb{R}.
	\]
	In other words, the set of solutions of Equation \eqref{eq:homo_lin_1order_ode} form a one-dimensional linear subspace of $C^1(I)$.
\end{theorem}

\begin{proof}
	Given $A \in \mathbb{R}$, define $u(x) = Ae^{-F(x)}$. Then
	\[
		u'(x) = -F'(x)Ae^{-F(x)} = -f(x)Ae^{-F(x)} = -f(x)u(x) \qquad \forall x \in I,
	\]
	so $u$ solves the ODE.
	
	Conversely, let $u \in C^1(I)$ solve Equation \eqref{eq:homo_lin_1order_ode} and set $v(x) = e^{F(x)}u(x)$. Then
	\[
		v'(x) = \big(e^{F(x)}\big)'u(x) = e^{F(x)}u'(x) = e^{F(x)}\left(f(x)u(x) - f(x)u(x)\right) = 0 \qquad \forall x \in I.
	\]
	By Corollary \ref{cor*const_deriv}, we deduce that $v(x) = A$ for some $A \in \mathbb{R}$, hence $u(x) = Ae^{-F(x)}$. \qedhere
\end{proof}

\begin{remark}
	In the previous result, solutions are written using a primitive $F$ of $f$. Since primitives are defined up to an additive constant, we can replace $F$ by $F + C$ for any $C \in \mathbb{R}$. This amounts to replacing $Ae^{-F(x)}$ by $Ae^{-C}e^{-F(x)}$, and since $A \in \mathbb{R}$ is arbitrary, this does not change the set of solutions.
\end{remark}

Next, we consider the non-homogeneous linear first-order ODE
\begin{equation}
	\label{eq:non_homo_lin_1order_ode}
	u'(x) + f(x)u(x) = g(x) \qquad \forall x \in I,
\end{equation}
where $f,g:I \to \mathbb{R}$ are given continuous functions.

To motivate the solution formula, we use the method of \textbf{variation of constants}. For the homogeneous equation we know that every solution has the form $Ae^{-F(x)}$ with $A \in \mathbb{R}$. The idea is to \textit{replace this constant by a function} and look for a solution of the non-homogeneous equation of the form
\[
	u(x) = H(x)e^{-F(x)},
\]
for some function $H \in C^1(I)$. With this choice,
\[
	u'(x) = H'(x)e^{-F(x)} - F'(x)H(x)e^{-F(x)} = H'(x)e^{-F(x)} - f(x)u(x).
\]
Thus, $u$ solves Equation \eqref{eq:non_homo_lin_1order_ode} if and only if
\[
	H'(x)e^{-F(x)} = g(x),
\]
which means that $H$ must be a primitive of the function $g(x)e^{F(x)}$.

This motivates the general solution formula:

\begin{theorem}{Non-Homogeneous Linear 1st Order ODEs}{non_homo_lin_1order_ode}
	Let $f,g:I \to \mathbb{R}$ be continuous, and consider the non-homogeneous first-order linear ODE \eqref{eq:non_homo_lin_1order_ode}. Let $F:I \to \mathbb{R}$ be a primitive of $f$, and let $H:I \to \mathbb{R}$ be a primitive of $ge^F$. Then every $C^1$ solution $u:I \to \mathbb{R}$ of Equation \eqref{eq:non_homo_lin_1order_ode} is of the form
	\[
		u(x) = H(x)e^{-F(x)} + Ae^{-F(x)}, \qquad A \in \mathbb{R}.
	\]
	In particular, the set of solutions of Equation \eqref{eq:non_homo_lin_1order_ode} is a one-dimensional affine subspace of $C^1(I)$.
\end{theorem}

\begin{proof}
	If $H$ is a primitive of $ge^F$, then $H + A$ is also a primitive for any constant $A$. Hence, by the same computation as the one performed above, it follows that
	\[
		u(x) = (H(x) + A)e^{-F(x)}
	\]
	solves Equation \eqref{eq:non_homo_lin_1order_ode}. Indeed
	\begin{align*}
		u'(x) &= (H(x) + A)'e^{-F(x)} - F'(x)(H(x) + A)e^{-F(x)}\\
		&= H'(x)e^{-F(x)} - f(x)u(x) = g(x) - f(x)u(x).
	\end{align*}
	
	Conversely, let $u$ be any solution of Equation \eqref{eq:non_homo_lin_1order_ode}, and set $v(x) = u(x) - H(x)e^{-F(x)}$. Then,
	\begin{align*}
		v'(x) &= u'(x) - H'(x)e^{-F(x)} + F'(x)H(x)e^{-F(x)}\\
		&= -f(x)u(x) + g(x) - g(x)e^{F(x)}e^{-F(x)} + f(x)H(x)e^{-F(x)}\\
		&= -f(x)u(x) + g(x) - g(x) + f(x)(u(x) - v(x)) = -f(x)v(x).
	\end{align*}
	Thus $v$ solves the homogeneous Equation \eqref{eq:homo_lin_1order_ode}. By Theorem \ref{theo*homo_lin_1order_ode}, we have $v(x) = Ae^{-F(x)}$ for some constant $A$. Therefore,
	\[
		u(x) = v(x) + H(x)e^{-F(x)} = Ae^{-F(x)} + H(x)e^{-F(x)},
	\]
	which proves the result.\qedhere
\end{proof}

The previous results give explicit formulas to solve every linear first-order ODE of the form $u' + fu = g$. In concrete situations, the difficulty lies in computing a primitive $F$ of $f$ and the a primitive of $g(x)e^{F(x)}$.

As we have seen, solutions depend on a free parameter $A \in \mathbb{R}$. This allows us to impose an initial condition of the form $u(x_0) = u_0$, which uniquely determines $A$.

\subsubsection*{Example}
We solve the ODE
\begin{equation}
	\label{eq:example_ode}
	u'(x) - 2x\, u(x) = e^{x^2}, \qquad u(0) = 1,
\end{equation}
on $\mathbb{R}$. Here $u$ is the unknown, and $f(x) = -2x$, $g(x) = e^{x^2}$ are given. According to Theorem \ref{theo*non_homo_lin_1order_ode}, we first find a primitive of $f$, i.e.,
\[
	F(x) = -x^2.
\]
Then we consider
\[
	g(x)e^{F(x)} = e^{x^2}e^{-x^2} = 1,
\]
whose primitive is $H(x) = x$. Thus $u$ must be of the form
\[
	u(x) = (x + A)e^{x^2}.
\]
Imposing the condition $u(0) = 1$ gives $A = 1$, hence
\begin{equation}
	\label{eq:example_ode_sol}
	u(x) = (x + 1)e^{x^2}.
\end{equation}

\begin{remark}
	If one forgets the formula form Theorem \ref{theo*non_homo_lin_1order_ode}, it is enough to remember the following procedure for solving $\eqref{eq:non_homo_lin_1order_ode}$. We start from
	\[
		u'(x) + f(x)u(x) = g(x),
	\]
	and multiply both sides by a function $e^{w(x)}$, i.e.,
	\[
		u'(x)e^{w(x)} + f(x)u(x)e^{w(x)} = g(x)e^{w(x)}.
	\]
	We look for $w$ such that the left-hand side is the derivative of $u(x)e^{w(x)}$, i.e.,
	\[
		\big(u(x)e^{w(x)}\big)' = u'(x)e^{w(x)} + w(x)u(x)e^{w(x)}.
	\]
	So we require
	\[
		w'(x) = f(x).
	\]
	If we choose $w = F$ to be any primitive of $f$ (the additive constant does not matter), then
	\[
		\big(u(x)e^{F(x)}\big)' = g(x)e^{F(x)},
	\]
	and therefore
	\[
		u(x)e^{F(x)} = \int ge^F + A,
	\]
	for some $A \in \mathbb{R}$. Thus, if $H$ is a primitive of $ge^F$, this reads
	\[
		u(x)e^{F(x)} = H(x) + A \qquad \Rightarrow \qquad u(x) = H(x)e^{-F(x)} + Ae^{-F(x)}.
	\]
\end{remark}

\subsubsection{Autonomous First Order ODEs}
We next study \textit{autonomous} first-order ODEs, where the rate of change of the unknown depends only on its current value, and not explicitly on $x$, i.e.,
\begin{equation}
	\label{eq:auto_1order_ode}
	u'(x) = f(u(x)),
\end{equation}
where $f:\mathbb{R} \to \mathbb{R}$ is a given continuous function, and $u$ is the unknown. The function $f$ tells us how $u$ should change depending on its current value.

A standard way to solve such equations is the method of \textbf{separation of variables}. If $u(x) = C \in \mathbb{R}$ for some $C$ such that $f(C) = 0$, then the constant function $u = C$ is a solution of the ODE. Otherwise, if $f(u(x)) \neq 0$, we can divide both sides by $f(u(x))$ and get
\[
	\frac{u'(x)}{f(u(x))} = 1.
\]
Integrating and using the substitution formula, we obtain
\begin{equation}
	\label{eq:auto_1order_ode2}
	\int \frac{1}{f(u)}\, du = \int \frac{1}{f(u(x))}u'(x)\, dx = \int 1\, dx = x + A,
\end{equation}
where $A$ is a constant of integration.

If $H$ is a primitive of $1/f$, this reads
\[
	H(u(x)) = x + A \qquad \Rightarrow \qquad u(x) = H^{-1}(x + A).
\]
Since we assumed $f(u(x)) \neq 0$ on the interval under consideration, we have $H' = \frac{1}{f} \neq 0$ there, so $H$ is strictly monotone and therefore invertible on that interval.

The method of separation of variables applies also to equations of the form
\[
	u'(x) = f(u(x))g(x), 
\]
where $f,g:\mathbb{R} \to \mathbb{R}$ are continuous. Assuming $f(u(x))\neq 0$ on a suitable interval, we divide and get
\[
	\frac{u'(x)}{f(u(x))} = g(x).
\]
Integrating gives
\[
	\int \frac{1}{f(u)}\, du = \int g(x) \, dx + A.
\]
If $H$ is a primitive of $1/f$ and $G$ is a primitive of $g$, then
\[
	H(u(x)) = G(x) + A \qquad \Rightarrow \qquad u(x) = H^{-1}(G(x) + A).
\]
We summarize this discussion in the next theorem:

\begin{theorem}{Separable Equations}{sep_eq}
	Let $I,J \subseteq \mathbb{R}$ be intervals, let $f:J \to \mathbb{R}$ and $g:I \to \mathbb{R}$ be continuous, and assume $f(y)\neq 0$ for all $y \in J$. Let $H$ be a primitive of $1/f$ on $J$, and let $G$ be a primitive of $g$ on $I$. Then every $C^1$ solution $u:I \to J$ of 
	\begin{equation}
		\label{eq:sep_eq}
		u'(x) = f(u(x))g(x)
	\end{equation}
	is of the form
	\[
		u(x) = H^{-1}(G(x) + A), \qquad A \in \mathbb{R}.
	\]
\end{theorem}

\begin{proof}
	Since $f$ never vanishes, $H' = 1 /f$ has constant sign. Hence, $H$ is strictly monotone and therefore invertible.
	
	If $u(x) = H^{-1}(G(x) + A)$, then by the chain rule (Theorem \ref{theo*chain_rule}) and the formula for the derivative of the inverse (Theorem \ref{theo*derivative_inv}), we get
	\[
		u'(x) = (H^{-1})'(G(x) + A)G'(x) = \frac{1}{H'\circ H^{-1}(G(x) + A)}g(x) = \frac{1}{H'(u(x))}g(x) = f(u(x))g(x),
	\]
	so $u$ solves Equation \eqref{eq:sep_eq}.
	
	Conversely, suppose $u$ solves Equation \eqref{eq:sep_eq}. Then
	\[
		(H \circ u(x))' = H'(u(x))u'(x) = \frac{1}{f(u(x))}f(u(x))g(x) = g(x) = G'(x).
	\]
	Thus, $H(u(x)) - G(x)$ has derivative zero, and is therefore constant, i.e.,
	\[
		H(u(x)) = G(x) + A, \qquad A \in \mathbb{R}.
	\]
	Applying $H^{-1}$ gives
	\[
		u(x) = H^{1}(G(x) + A),
	\]
	which completes the proof.\qedhere
\end{proof}

\subsubsection{Homogeneous Linear Second Order ODEs with Constant Coefficients}
\label{sec:homo_lin_2order_ode}
We now move to second-order linear ODEs. These are considerably more difficult to solve than first-order ones in general, so we start with the simplest case: homogeneous equations with constant coefficients,
\begin{equation}
	\label{eq:homo_lin_2order_ode}
	u''(x) + a_1u'(x) + a_0u(x) = 0,
\end{equation}
where $a_0, a_1 \in \mathbb{R}$ are given constants and $u$ is the unknown. Such equations already cover many important applications (for instance, oscillations and damped vibrations).

We choose the \textit{Ansatz} that the solutions of Equation \eqref{eq:homo_lin_2order_ode} is of the form
\[
	u(x) = e^{\alpha x},\qquad \alpha \in \mathbb{C}.
\]
With this choice, 
\[
	u''(x) + a_1u(x) + a_0u(x) = (\alpha^2 + a_2\alpha + a_0)u(x) = 0,
\]
so $u$ is a solution if and only if
\[
	\alpha^2 + a_1\alpha + a_0 = 0.
\]
The quadratic polynomial
\[
	p(t) = t^2 + a_1t + a_0
\]
is called the \textbf{characteristic polynomial}. Its roots determine the shape of the solutions. We distinguish three cases according to the discriminant $\Delta = a_1^2 - 4a_0$.
\begin{itemize}
	\item[$\bullet$] $\underline{\text{Case 1: } \Delta > 0.}$ The polynomial $p(t)$ has two distinct roots
	\begin{equation}
		\label{eq:root_case1}
		\alpha = \frac{-a_1 + \sqrt{\Delta}}{2}, \qquad \beta = \frac{-a_1 - \sqrt{\Delta}}{2}.
	\end{equation}
	Then $x \mapsto e^{\alpha x}$ and $x \mapsto e^{\beta x}$ are two linearly independent solutions, and therefore
	\[
		u(x) = Ae^{\alpha x} + Be^{\beta x}, \qquad A,B\in \mathbb{R},
	\]
	is a solutions of Equation \eqref{eq:homo_lin_2order_ode}
	\item[$\bullet$] $\underline{\text{Case 2: } \Delta < 0}.$ Then $p(t)$ has two complex-conjugate roots
	\begin{equation}
		\label{eq:roots_case2}
		\alpha + i \beta = -\frac{a_1}{2} + i \frac{\sqrt{-\Delta}}{2}, \qquad \alpha - i \beta = -\frac{a_1}{2} - i \frac{\sqrt{-\Delta}}{2},
	\end{equation}
	with $\beta > 0$. The complex-valued functions $x \mapsto e^{(\alpha \pm i\beta)x}$ solve Equation \eqref{eq:homo_lin_2order_ode}, and hence their real and imaginary parts are solutions. This gives
	\[
		u(x) = Ae^{\alpha x} \sin(\beta x) + Be^{\alpha x}\cos(\beta x), \qquad A,B\in \mathbb{R}.
	\]
	\item[$\bullet$] $\underline{\text{Case 3: }\Delta = 0}.$ Then $p(t)$ has a double root
	\begin{equation}
		\label{eq:root_case3}
		\alpha = -\frac{a_1}{2},
	\end{equation}
	so $x \mapsto e^{\alpha x}$ is a solution of Equation \eqref{eq:homo_lin_2order_ode}. To find another independent solution, recall the special case $u'' = 0$, where two linearly independent solutions are 1 and $x$, which can be written as $e^{\gamma x}$ and $xe^{\gamma x}$ with $\gamma = 0$. This suggests that $x \mapsto xe^{\alpha x}$ might be a solution. Indeed,
	\[
		\big(xe^{\alpha x}\big)'' + a_1 \big(xe^{\alpha x}\big)' + a_0xe^{\alpha x} = (\underbrace{\alpha^2 + a_1\alpha + a_0}_{\displaystyle =0})xe^{\alpha x} + (\underbrace{2\alpha + a_1}_{\displaystyle =0})e^{\alpha x} = 0,
 	\]
 	where the first term vanishes because $\alpha$ is a root of $p$, and the second vanishes by \eqref{eq:root_case3}. Hence
 	\[
 		u(x) = Ae^{\alpha x} + Bxe^{\alpha x}, \qquad A,B\in \mathbb{R},
 	\]
 	solves Equation \eqref{eq:homo_lin_2order_ode}.
\end{itemize}

For second-order ODEs it is customary to prescribe both the value of $u$ and the value of its derivative at some point (for instance, $u(0) = 1$ and $u'(0) = 0$). The two constants $A,B$ in the formulas above are precisely what we need in order to satisfy two such conditions.

\begin{theorem}{Existence and Uniqueness: The Homogeneous Case}{exist_unique_homo}
	Given $a_0, a_1 \in \mathbb{R}$. let $\Delta = a_1^2 - 4a_0$ and consider the following solutions of Equation \eqref{eq:homo_lin_2order_ode}:
	\begin{align*}
		\underline{\Delta > 0:} \qquad &u_1(x) = e^{\alpha x}, \qquad &u_2(x) = e^{\beta x}, \qquad &\alpha, \beta \text{ as in \eqref{eq:root_case1}}\\
		\underline{\Delta < 0:} \qquad &u_1(x) = e^{\alpha x}\sin(\beta x), \qquad &u_2(x) = e^{\alpha x}\cos(\beta x), \qquad &\alpha,\beta \text{ as in \eqref{eq:roots_case2}}\\
		\underline{\Delta = 0:} \qquad &u_1(x) = e^{\alpha x}, \qquad  &u_2(x) = xe^{\alpha x}, \qquad &\alpha,\beta \text{ as in \eqref{eq:root_case3}}.
	\end{align*}
	If $u \in C^2(I)$ solves Equation \eqref{eq:homo_lin_2order_ode}, then there exists $A,B \in \mathbb{R}$ such that
	\[
		u = Au_1 + Bu_2.
	\]
	In other words, the set of solutions of Equation \eqref{eq:homo_lin_2order_ode} forms a two-dimensional linear subspace of $C^2(I)$.
\end{theorem}
We will skip the proof of this theorem. (It's in the skript)

\begin{corollary}{Wronskian and Linear Dependence}{wronskian_lin_dep}
	Let $v_1, v_2 \in C^2(I)$ be solutions of Equation \eqref{eq:homo_lin_2order_ode} on an interval $I \subseteq \mathbb{R}$, and let
	\[
		W(x) = v_1(x)v_2'(x) - v_1'(x)v_2(x)
	\]
	be their Wronskian. If $W(x_0) = 0$ for some $x_0 \in I$, then $v_1$ and $v_2$ are linearly dependent on $I$.
\end{corollary}
\begin{proof}
	Consider the $2 \times 2$ matrix
	\[
		M(x_0) = \begin{pmatrix}
			v_1(x_0) & v_2(x_0)\\
			v_1'(x_0) & v_2'(x_0)
		\end{pmatrix}.
	\]
	The condition $W(x_0) = 0$ means precisely that $\det M(x_0) = 0$, so there exists a non-zero vector $(A,B) \in \mathbb{R}^2$ such that
	\[
		Av_1(x_0) + Bv_2(x_0) = 0,\qquad Av_1'(x_0) + Bv_2'(x_0) = 0.
	\]
	Define
	\[
		w(x) = Av_1(x) + Bv_2(x).
	\]
	Then $w$ solves Equation \eqref{eq:homo_lin_2order_ode} and satisfies
	\[
		w(x_0) = 0, \qquad w'(x_0) = 0.
	\]
	The only solution with these initial data is the trivial one, so $w = 0$ on $I$. This proves that $Av_1 + Bv_2 = 0$, thus $v_1$ and $v_2$ are linearly dependent on $I$.\qedhere
\end{proof}

\subsubsection{Non-Homogeneous Linear Second Order ODEs with Constant Coefficients}
We now add a forcing term and consider the non-homogeneous linear second-order ODE with constant coefficients
\begin{equation}
	\label{eq:non_homo_lin_2order_ode}
	u''(x) + a_1u'(x) + a_0u(x) = g(x),
\end{equation}
where $a_0, a_1 \in \mathbb{R}$ are constants and $g \in C^0(I)$ is a given function. The unknown is again $u$.

With the notation of Paragraph \ref{sec:homo_lin_2order_ode}, let $u_1,u_2$ be two linearly independent solutions of the homogeneous Equation \eqref{eq:homo_lin_2order_ode}:
\begin{align*}
	\underline{\Delta > 0:} \qquad &u_1(x) = e^{\alpha x}, \qquad &u_2(x) = e^{\beta x}, \qquad &\alpha, \beta \text{ as in \eqref{eq:root_case1}}\\
	\underline{\Delta < 0:} \qquad &u_1(x) = e^{\alpha x}\sin(\beta x), \qquad &u_2(x) = e^{\alpha x}\cos(\beta x), \qquad &\alpha,\beta \text{ as in \eqref{eq:roots_case2}}\\
	\underline{\Delta = 0:} \qquad &u_1(x) = e^{\alpha x}, \qquad  &u_2(x) = xe^{\alpha x}, \qquad &\alpha,\beta \text{ as in \eqref{eq:root_case3}}.
\end{align*}
We look for a solution $u$ of Equation \eqref{eq:non_homo_lin_2order_ode} of the form
\[
	u(x) = H_1(x)u_1(x) + H_2(x)u_2(x),
\]
where $H_1, H_2$ are unknown functions. This is the method of \textbf{variation of constants} in the second-order setting. We will spare you the details of the computation and state the theorem directly:

\begin{theorem}{Existence and Uniqueness: The Non-Homogeneous Case}{exist_unique_non_homo}
	Given $a_0, a_1 \in \mathbb{R}$, let $\Delta = a_1^2 - 4a_0$ and consider the following solutions of Equation \eqref{eq:homo_lin_2order_ode}:
	\begin{align*}
		\underline{\Delta > 0:} \qquad &u_1(x) = e^{\alpha x}, \qquad &u_2(x) = e^{\beta x}, \qquad &\alpha, \beta \text{ as in \eqref{eq:root_case1}}\\
		\underline{\Delta < 0:} \qquad &u_1(x) = e^{\alpha x}\sin(\beta x), \qquad &u_2(x) = e^{\alpha x}\cos(\beta x), \qquad &\alpha,\beta \text{ as in \eqref{eq:roots_case2}}\\
		\underline{\Delta = 0:} \qquad &u_1(x) = e^{\alpha x}, \qquad  &u_2(x) = xe^{\alpha x}, \qquad &\alpha,\beta \text{ as in \eqref{eq:root_case3}}.
	\end{align*}
	Let $H_1$ and $H_2$ be primitives of $\dfrac{u_2g}{u_1'u_2 - u_2'u_1}$ and $\dfrac{u_1g}{u_2'u_1 - u_1'u_2}$, respectively. If $u \in C^2(I)$ solves Equation \eqref{eq:non_homo_lin_2order_ode}, then there exists $A,B \in \mathbb{R}$ such that
	\[
		u = Au_1 + Bu_2 + H_1u_1 + H_2u_2.
	\]
	In other words, the set of solutions of Equation \eqref{eq:non_homo_lin_2order_ode} forms a two-dimensional affine subspace of $C^2(I)$.
\end{theorem}
\begin{proof}
	First we show existence. By the computation above, that was skipped (but is in the skript), if $H_1$ and $H_2$ are primitives of 
	\[
		\frac{u_2g}{u_1'u_2 - u_2'u_1} \quad \text{and} \quad \frac{u_1g}{u_2'u_1 - u_1'u_2},
	\]
	then the function
	\[
		u_p = H_1u_1 + H_2u_2
	\]
	satisfies
	\[
		u_p'' + a_1u_p' + a_0u_p = g,
	\]
	so $u_p$ is a particular solution of Equation \eqref{eq:non_homo_lin_2order_ode}. Since the equation is linear, for any $A,B \in \mathbb{R}$ the function
	\[
		u = Au_1+ Bu_2 + u_p
	\]
	also solves Equation \eqref{eq:non_homo_lin_2order_ode}. This proves the existence of solutions of the stated form.
	
	For uniqueness, let $u \in C^2(I)$ be any solution of Equation \eqref{eq:homo_lin_2order_ode}, and define
	\[
		v= u - u_p = u - (H_1u_1 + H_2u_2).
	\]
	Then $v$ solves the homogeneous Equation \eqref{eq:homo_lin_2order_ode} and therefore, by Theorem \ref{theo*exist_unique_homo}, there exist $A,B \in \mathbb{R}$ such that
	\[
		v= Au_1 + Bu_2.
	\]
	Hence, $u = v + u_p = Au_1 + Bu_2 + H_1u_1 + H_2u_2$, as desired.\qedhere
\end{proof}

\subsection{Existence and Uniqueness for ODEs}
\subsubsection{Existence and Uniqueness for First Order ODEs}
Our goal now is to present the general theory of first-order ODEs for real-valued functions on the real line. In general, a first-order ODE is an equation relating $x, u(x), u'(x)$, i.e.,
\[
	G(x, u(x), u'(x)) = 0.
\]
In this section we restrict to equations for which one can ''isolate'' $u'$, so that one can write the ODE in \textit{normal form}, i.e.,
\[
	u'(x) = f(x, u(x)).
\]

\begin{definition}{First-Order ODEs in Normal Form}
	A first-order ODE in \textbf{normal form} is an equation of the type
	\[
		u'(x) = f(x, u(x)),
	\]
	where $f:\mathbb{R}^2 \to \mathbb{R}$ is a given function and $u:\mathbb{R} \to \mathbb{R}$ is the unknown.
\end{definition}

The Cauchy-Lipschitz Theorem, also known as Picard-Lindelöf Theorem, is a fundamental result in the theory of ODEs. It ensures the existence and uniqueness of solutions under suitable conditions on $f$.

In the next theorem we need to assume that $f$ is continuous as a function of the two variables $x$ and $y$. This means that, for any point $(x_0,y_0)$ in the domain and for any $\varepsilon > 0$, there exists $\delta > 0$ such that
\[
	|x-x_0| < \delta \quad \text{and} \quad |y-y_0| < \delta \qquad \Rightarrow \qquad |f(x,y) - f(x_0,y_0)| < \varepsilon.
\]

\begin{theorem}{Cauchy-Lipschitz: Global Version}{cauchy_lipschitz_global}
	Let $f:\mathbb{R}\times \mathbb{R} \to \mathbb{R}$ satisfy the following conditions:
	\begin{enumerate}
		\item $f$ is continuous in $\mathbb{R}\times \mathbb{R}$;
		\item $f$ is Lipschitz continuous with respect to the second variable, i.e., there exists a constant $L > 0$ such that
		\[
			|f(x,y_1) - f(x,y_2)| \leq L|y_1 - y_2| \qquad \forall x,y_1,y_2\in \mathbb{R}.
		\]
		Then, for any point $(x_0,y_0) \in \mathbb{R}\times \mathbb{R}$ there exists a unique $C^1$ function $u:\mathbb{R}\to \mathbb{R}$ such that
		\begin{equation}
			\begin{cases}
				u'(x) = f(x,u(x)) \qquad \forall x \in\mathbb{R},\\
				u(x_0) = y_0.
			\end{cases}
		\end{equation}
	\end{enumerate}
\end{theorem}

\begin{theorem}{Cauchy-Lipschitz: Local Version}{cauchy_lipschitz_local}
	Let $I \subseteq \mathbb{R}$ be an interval, and let $f:I \times \mathbb{R} \to \mathbb{R}$ satisfy the following conditions:
	\begin{enumerate}
		\item $f$ is continuous in $I \times \mathbb{R}$;
		\item $f$ is locally Lipschitz continuous with respect to the second variable, i.e., for every pair of compact intervals $[a,b]\subseteq I$ and $[c,d]\subseteq \mathbb{R}$ there exists a constant such that
		\[
			|f(x,y_1) - f(x,y_2)| \leq L|y_1 - y_2| \qquad \forall x \in [a,b], \;\forall y_1,y_2 \in [c,d].
		\]
	\end{enumerate}
	Then, for any point $(x_0,y_0) \in I \times \mathbb{R}$ there exists an interval $I' \subseteq I$ containing $x_0$ and a unique function $u:I' \to \mathbb{R}$ such that
	\begin{equation}
		\begin{cases}
			u'(x) = f(x,u(x)) \qquad \forall x \in I',\\
			u(x_0) = y_0.
		\end{cases}
	\end{equation}
\end{theorem}

In other words, under a local Lipschitz assumption, one can only guarantee the existence and uniqueness of a solution on some interval around $x_0$. Moreover, as long as the solution $u(x)$ remains bounded within $I'$, one can continue applying Theorem \ref{theo*cauchy_lipschitz_local} to extend the interval $I'$ as much as possible.

\subsubsection{Higher Order ODEs}
Suppose we are given an $n$-th order ODE of the form
\[
	G(x, u'(x), u''(x), \hdots , u^{(n)}(x)) = 0,
\]
and assume that the highest derivative can be isolated and written as
\begin{equation}
	\label{eq:norder_iso}
	u^{(n)}(x) = f(x, u(x), u'(x), \hdots , u^{(n-1)}(x)),
\end{equation}
where $f:\mathbb{R}\times \mathbb{R}^n \to \mathbb{R}$ is a given function

We introduce the variables
\[
	U_1 = u, \qquad U_2 = u', \qquad U_3 = u'', \;\hdots \;, U_n = u^{(n-1)}.
\]
By definition, we have
\[
	U_1' = U_2, \qquad U_2' = U_3, \;\hdots \;, U_{n-1}' = U_n,
\]
and Equation \eqref{eq:norder_iso} becomes
\[
	U_n' = u^{(n)} = f(x, U_1, U_2,\hdots, U_n).
\]
Therefore, the $n$-th order Equation \eqref{eq:norder_iso} is equivalent to the first-order system
\[
	\begin{cases}
		U_1' &= U_2,\\
		U_2' &= U_3,\\
		&\vdots \\
		U_{n-1}' &= U_n,\\
		U_n' &= f(x,U_1, U_2,\hdots, U_n).
	\end{cases}
\]
The Cauchy-Lipschitz Theorem (both its global and local versions) extends to systems of first-order ODEs and ensures existence and uniqueness of solutions whenever the right-hand side is continuous and (locally) Lipschitz continuous with respect to the variable $(U_1, \hdots , U_n)$. In particular, once the initial conditions
\[
	U_1(X_0) = u(x_0), \qquad U_2(x_0) = u'(x_0), \;\hdots \;, U_n(x_0) = u^{(n-1)}(x_0)
\]
are prescribed at some $x_0 \in I$, there exists a unique (local) solution to the system, and hence a unique solution to the original $n$-th order ODE.

\begin{theorem}{Existence and Uniqueness For Linear Second-Order ODEs}{exist_unique_2order_ode}
	Let $a_0, a_1: I \to \mathbb{R}$ be continuous and bounded functions. Then the set of solutions of the linear homogeneous equation
	\[
		u''(x) + a_1(x)u'(x) + a_0(x)u(x) = 0\qquad \forall x \in I
	\]
	is a two-dimensional linear subspace of $C^2(I)$.
\end{theorem}
proof can be found in the skript.
	
	
	
	
\end{document}
